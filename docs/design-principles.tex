\documentclass[]{article}
\usepackage{lmodern}
\usepackage{amssymb,amsmath}
\usepackage{ifxetex,ifluatex}
\usepackage{fixltx2e} % provides \textsubscript
\ifnum 0\ifxetex 1\fi\ifluatex 1\fi=0 % if pdftex
  \usepackage[T1]{fontenc}
  \usepackage[utf8]{inputenc}
\else % if luatex or xelatex
  \ifxetex
    \usepackage{mathspec}
  \else
    \usepackage{fontspec}
  \fi
  \defaultfontfeatures{Ligatures=TeX,Scale=MatchLowercase}
\fi
% use upquote if available, for straight quotes in verbatim environments
\IfFileExists{upquote.sty}{\usepackage{upquote}}{}
% use microtype if available
\IfFileExists{microtype.sty}{%
\usepackage{microtype}
\UseMicrotypeSet[protrusion]{basicmath} % disable protrusion for tt fonts
}{}
\usepackage[margin=1in]{geometry}
\usepackage{hyperref}
\hypersetup{unicode=true,
            pdftitle={Design Principles, Comparisons and Limitations},
            pdfauthor={David Hugh-Jones},
            pdfborder={0 0 0},
            breaklinks=true}
\urlstyle{same}  % don't use monospace font for urls
\usepackage{graphicx,grffile}
\makeatletter
\def\maxwidth{\ifdim\Gin@nat@width>\linewidth\linewidth\else\Gin@nat@width\fi}
\def\maxheight{\ifdim\Gin@nat@height>\textheight\textheight\else\Gin@nat@height\fi}
\makeatother
% Scale images if necessary, so that they will not overflow the page
% margins by default, and it is still possible to overwrite the defaults
% using explicit options in \includegraphics[width, height, ...]{}
\setkeys{Gin}{width=\maxwidth,height=\maxheight,keepaspectratio}
\IfFileExists{parskip.sty}{%
\usepackage{parskip}
}{% else
\setlength{\parindent}{0pt}
\setlength{\parskip}{6pt plus 2pt minus 1pt}
}
\setlength{\emergencystretch}{3em}  % prevent overfull lines
\providecommand{\tightlist}{%
  \setlength{\itemsep}{0pt}\setlength{\parskip}{0pt}}
\setcounter{secnumdepth}{0}
% Redefines (sub)paragraphs to behave more like sections
\ifx\paragraph\undefined\else
\let\oldparagraph\paragraph
\renewcommand{\paragraph}[1]{\oldparagraph{#1}\mbox{}}
\fi
\ifx\subparagraph\undefined\else
\let\oldsubparagraph\subparagraph
\renewcommand{\subparagraph}[1]{\oldsubparagraph{#1}\mbox{}}
\fi

%%% Use protect on footnotes to avoid problems with footnotes in titles
\let\rmarkdownfootnote\footnote%
\def\footnote{\protect\rmarkdownfootnote}

%%% Change title format to be more compact
\usepackage{titling}

% Create subtitle command for use in maketitle
\newcommand{\subtitle}[1]{
  \posttitle{
    \begin{center}\large#1\end{center}
    }
}

\setlength{\droptitle}{-2em}

  \title{Design Principles, Comparisons and Limitations}
    \pretitle{\vspace{\droptitle}\centering\huge}
  \posttitle{\par}
    \author{David Hugh-Jones}
    \preauthor{\centering\large\emph}
  \postauthor{\par}
      \predate{\centering\large\emph}
  \postdate{\par}
    \date{2018-11-07}

\usepackage{array}
\usepackage{caption}
\usepackage{graphicx}
\usepackage{siunitx}
\usepackage[table]{xcolor}
\usepackage{multirow}
\usepackage{hhline}
\usepackage{calc}
\usepackage{tabularx}
\usepackage{threeparttable}

\begin{document}
\maketitle

This document briefly describes the design of
\href{https://hughjonesd.github.io/huxtable}{huxtable}, and compares it
with other R packages for creating tables. A current version is on the
web in
\href{http://hughjonesd.github.io/huxtable/design-principles.html}{HTML}
or
\href{http://hughjonesd.github.io/huxtable/design-principles.pdf}{PDF}
formats.

\hypertarget{design-principles}{%
\subsection{Design principles}\label{design-principles}}

I wrote this package because I wanted a simple way to create tables in
my LaTeX documents. At the same time, I wanted to be able to output HTML
or Markdown for use in RStudio. And, I wanted to be able to edit tables
intuitively using standard R features. My typical use case is creating
tables of regression outputs, but I also wanted to be able to represent
arbitrary data, like a table of descriptive statistics or of plain text.

The idea behind huxtable is to store data in a normal data frame, along
with properties that describe how to display the data, at cell, column,
row or table level. Operations on the data frame work as normal, and
they also affect the display properties. Then, the data can be output in
an appropriate format. At the moment, those formats are LaTeX, HTML,
Markdown, Word/Excel/Powerpoint, RTF and on-screen pretty printing. More
could be added.

Another design choice was to have separate functions per feature. Many
existing packages use a single function with a large number of options.
For instance, \texttt{print.xtable} in the \texttt{xtable} package has
34 options, and \texttt{texreg} in the \texttt{texreg} package has 41.
Having one function per feature should make life easier for the end
user. It should also lead to clearer code: each function starts with a
valid huxtable, changes one thing, and returns a valid huxtable.

The output formats are very different, and decisions have to be made as
to what any package will support. My background is more in HTML. This is
reflected in some of the huxtable properties, like per-cell borders and
padding. The package tries to keep output reasonably similar between
LaTeX and HTML, but there are inevitably some differences and
limitations (see below). For Markdown and on-screen output, obviously,
only a few basic properties are supported.

The package makes no attempt to output beautiful HTML or LaTeX source
code. In fact, in the case of LaTeX, it's pretty ugly. The approach is
``do what it takes to get the job done''.

\hypertarget{comparing-huxtable-with-other-packages}{%
\subsection{Comparing Huxtable With Other
Packages}\label{comparing-huxtable-with-other-packages}}

R has many different packages to create LaTeX and HTML tables. The
table(s) below list those I know and the features they have. The table
is produced with huxtable, of course ;-)

  \providecommand{\huxb}[2]{\arrayrulecolor[RGB]{#1}\global\arrayrulewidth=#2pt}
  \providecommand{\huxvb}[2]{\color[RGB]{#1}\vrule width #2pt}
  \providecommand{\huxtpad}[1]{\rule{0pt}{\baselineskip+#1}}
  \providecommand{\huxbpad}[1]{\rule[-#1]{0pt}{#1}}

\begin{table}[h]
\begin{raggedright}
\begin{threeparttable}
\captionsetup{justification=raggedright,singlelinecheck=off}
\caption{Comparison table, part 2}
\resizebox*{!}{0.95\textheight}{\begin{tabularx}{0.5\textwidth}{p{120pt} p{36pt} p{36pt} p{36pt} p{36pt} p{36pt} p{36pt} p{36pt} p{36pt} p{36pt} p{36pt}}


\hhline{}
\arrayrulecolor{black}

\multicolumn{1}{!{\huxvb{0, 0, 0}{0}}m{120pt}!{\huxvb{0, 0, 0}{0}}}{\parbox[c]{120pt-4pt-4pt}{\huxtpad{4pt}\raggedright \rotatebox{270}{\textbf{{\fontsize{10pt}{12pt}\selectfont }}}\huxbpad{4pt}}\rule{0pt}{20pt}} &
\multicolumn{1}{m{36pt}!{\huxvb{0, 0, 0}{0}}}{\cellcolor[RGB]{242, 242, 242}\parbox[c]{36pt-4pt-4pt}{\huxtpad{4pt}\raggedright \rotatebox{270}{\textbf{{\fontsize{10pt}{12pt}\selectfont Hmisc::latex}}}\huxbpad{4pt}}\rule{0pt}{20pt}} &
\multicolumn{1}{m{36pt}!{\huxvb{0, 0, 0}{0}}}{\parbox[c]{36pt-4pt-4pt}{\huxtpad{4pt}\raggedright \rotatebox{270}{\textbf{{\fontsize{10pt}{12pt}\selectfont ztable}}}\huxbpad{4pt}}\rule{0pt}{20pt}} &
\multicolumn{1}{m{36pt}!{\huxvb{0, 0, 0}{0}}}{\cellcolor[RGB]{242, 242, 242}\parbox[c]{36pt-4pt-4pt}{\huxtpad{4pt}\raggedright \rotatebox{270}{\textbf{{\fontsize{10pt}{12pt}\selectfont kable}}}\huxbpad{4pt}}\rule{0pt}{20pt}} &
\multicolumn{1}{m{36pt}!{\huxvb{0, 0, 0}{0}}}{\parbox[c]{36pt-4pt-4pt}{\huxtpad{4pt}\raggedright \rotatebox{270}{\textbf{{\fontsize{10pt}{12pt}\selectfont texreg}}}\huxbpad{4pt}}\rule{0pt}{20pt}} &
\multicolumn{1}{m{36pt}!{\huxvb{0, 0, 0}{0}}}{\cellcolor[RGB]{242, 242, 242}\parbox[c]{36pt-4pt-4pt}{\huxtpad{4pt}\raggedright \rotatebox{270}{\textbf{{\fontsize{10pt}{12pt}\selectfont stargazer}}}\huxbpad{4pt}}\rule{0pt}{20pt}} &
\multicolumn{1}{m{36pt}!{\huxvb{0, 0, 0}{0}}}{\parbox[c]{36pt-4pt-4pt}{\huxtpad{4pt}\raggedright \rotatebox{270}{\textbf{{\fontsize{10pt}{12pt}\selectfont tables}}}\huxbpad{4pt}}\rule{0pt}{20pt}} &
\multicolumn{1}{m{36pt}!{\huxvb{0, 0, 0}{0}}}{\cellcolor[RGB]{242, 242, 242}\parbox[c]{36pt-4pt-4pt}{\huxtpad{4pt}\raggedright \rotatebox{270}{\textbf{{\fontsize{10pt}{12pt}\selectfont DT}}}\huxbpad{4pt}}\rule{0pt}{20pt}} &
\multicolumn{1}{m{36pt}!{\huxvb{0, 0, 0}{0}}}{\parbox[c]{36pt-4pt-4pt}{\huxtpad{4pt}\raggedright \rotatebox{270}{\textbf{{\fontsize{10pt}{12pt}\selectfont pander}}}\huxbpad{4pt}}\rule{0pt}{20pt}} &
\multicolumn{1}{m{36pt}!{\huxvb{0, 0, 0}{0}}}{\cellcolor[RGB]{242, 242, 242}\parbox[c]{36pt-4pt-4pt}{\huxtpad{4pt}\raggedright \rotatebox{270}{\textbf{{\fontsize{10pt}{12pt}\selectfont flextable}}}\huxbpad{4pt}}\rule{0pt}{20pt}} &
\multicolumn{1}{m{36pt}!{\huxvb{0, 0, 0}{0}}}{\parbox[c]{36pt-4pt-4pt}{\huxtpad{4pt}\raggedright \rotatebox{270}{\textbf{{\fontsize{10pt}{12pt}\selectfont pixiedust}}}\huxbpad{4pt}}\rule{0pt}{20pt}} \tabularnewline[-0.5pt]


\multicolumn{11}{!{\huxvb{0, 0, 0}{0}}p{120pt+36pt+36pt+36pt+36pt+36pt+36pt+36pt+36pt+36pt+36pt+20\tabcolsep}!{\huxvb{0, 0, 0}{0}}}{\parbox[b]{120pt+36pt+36pt+36pt+36pt+36pt+36pt+36pt+36pt+36pt+36pt+20\tabcolsep-0pt-0pt}{\huxtpad{0pt}\raggedright {\fontsize{8pt}{9.6pt}\selectfont A (Y) means that there is limited support for the feature.
                    For example, multirow cells may only be supported in headers, or only horizontal
                    border lines may work.}\huxbpad{0pt}}\rule{0pt}{10pt}} \tabularnewline[-0.5pt]


\end{tabularx}}\end{threeparttable}
\par\end{raggedright}

\end{table}
\end{document}
