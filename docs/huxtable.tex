\documentclass[]{article}
\usepackage{lmodern}
\usepackage{amssymb,amsmath}
\usepackage{ifxetex,ifluatex}
\usepackage{fixltx2e} % provides \textsubscript
\ifnum 0\ifxetex 1\fi\ifluatex 1\fi=0 % if pdftex
  \usepackage[T1]{fontenc}
  \usepackage[utf8]{inputenc}
\else % if luatex or xelatex
  \ifxetex
    \usepackage{mathspec}
  \else
    \usepackage{fontspec}
  \fi
  \defaultfontfeatures{Ligatures=TeX,Scale=MatchLowercase}
\fi
% use upquote if available, for straight quotes in verbatim environments
\IfFileExists{upquote.sty}{\usepackage{upquote}}{}
% use microtype if available
\IfFileExists{microtype.sty}{%
\usepackage{microtype}
\UseMicrotypeSet[protrusion]{basicmath} % disable protrusion for tt fonts
}{}
\usepackage[margin=1in]{geometry}
\usepackage{hyperref}
\hypersetup{unicode=true,
            pdftitle={Introduction to Huxtable},
            pdfauthor={David Hugh-Jones},
            pdfborder={0 0 0},
            breaklinks=true}
\urlstyle{same}  % don't use monospace font for urls
\usepackage{color}
\usepackage{fancyvrb}
\newcommand{\VerbBar}{|}
\newcommand{\VERB}{\Verb[commandchars=\\\{\}]}
\DefineVerbatimEnvironment{Highlighting}{Verbatim}{commandchars=\\\{\}}
% Add ',fontsize=\small' for more characters per line
\usepackage{framed}
\definecolor{shadecolor}{RGB}{248,248,248}
\newenvironment{Shaded}{\begin{snugshade}}{\end{snugshade}}
\newcommand{\AlertTok}[1]{\textcolor[rgb]{0.94,0.16,0.16}{#1}}
\newcommand{\AnnotationTok}[1]{\textcolor[rgb]{0.56,0.35,0.01}{\textbf{\textit{#1}}}}
\newcommand{\AttributeTok}[1]{\textcolor[rgb]{0.77,0.63,0.00}{#1}}
\newcommand{\BaseNTok}[1]{\textcolor[rgb]{0.00,0.00,0.81}{#1}}
\newcommand{\BuiltInTok}[1]{#1}
\newcommand{\CharTok}[1]{\textcolor[rgb]{0.31,0.60,0.02}{#1}}
\newcommand{\CommentTok}[1]{\textcolor[rgb]{0.56,0.35,0.01}{\textit{#1}}}
\newcommand{\CommentVarTok}[1]{\textcolor[rgb]{0.56,0.35,0.01}{\textbf{\textit{#1}}}}
\newcommand{\ConstantTok}[1]{\textcolor[rgb]{0.00,0.00,0.00}{#1}}
\newcommand{\ControlFlowTok}[1]{\textcolor[rgb]{0.13,0.29,0.53}{\textbf{#1}}}
\newcommand{\DataTypeTok}[1]{\textcolor[rgb]{0.13,0.29,0.53}{#1}}
\newcommand{\DecValTok}[1]{\textcolor[rgb]{0.00,0.00,0.81}{#1}}
\newcommand{\DocumentationTok}[1]{\textcolor[rgb]{0.56,0.35,0.01}{\textbf{\textit{#1}}}}
\newcommand{\ErrorTok}[1]{\textcolor[rgb]{0.64,0.00,0.00}{\textbf{#1}}}
\newcommand{\ExtensionTok}[1]{#1}
\newcommand{\FloatTok}[1]{\textcolor[rgb]{0.00,0.00,0.81}{#1}}
\newcommand{\FunctionTok}[1]{\textcolor[rgb]{0.00,0.00,0.00}{#1}}
\newcommand{\ImportTok}[1]{#1}
\newcommand{\InformationTok}[1]{\textcolor[rgb]{0.56,0.35,0.01}{\textbf{\textit{#1}}}}
\newcommand{\KeywordTok}[1]{\textcolor[rgb]{0.13,0.29,0.53}{\textbf{#1}}}
\newcommand{\NormalTok}[1]{#1}
\newcommand{\OperatorTok}[1]{\textcolor[rgb]{0.81,0.36,0.00}{\textbf{#1}}}
\newcommand{\OtherTok}[1]{\textcolor[rgb]{0.56,0.35,0.01}{#1}}
\newcommand{\PreprocessorTok}[1]{\textcolor[rgb]{0.56,0.35,0.01}{\textit{#1}}}
\newcommand{\RegionMarkerTok}[1]{#1}
\newcommand{\SpecialCharTok}[1]{\textcolor[rgb]{0.00,0.00,0.00}{#1}}
\newcommand{\SpecialStringTok}[1]{\textcolor[rgb]{0.31,0.60,0.02}{#1}}
\newcommand{\StringTok}[1]{\textcolor[rgb]{0.31,0.60,0.02}{#1}}
\newcommand{\VariableTok}[1]{\textcolor[rgb]{0.00,0.00,0.00}{#1}}
\newcommand{\VerbatimStringTok}[1]{\textcolor[rgb]{0.31,0.60,0.02}{#1}}
\newcommand{\WarningTok}[1]{\textcolor[rgb]{0.56,0.35,0.01}{\textbf{\textit{#1}}}}
\usepackage{graphicx,grffile}
\makeatletter
\def\maxwidth{\ifdim\Gin@nat@width>\linewidth\linewidth\else\Gin@nat@width\fi}
\def\maxheight{\ifdim\Gin@nat@height>\textheight\textheight\else\Gin@nat@height\fi}
\makeatother
% Scale images if necessary, so that they will not overflow the page
% margins by default, and it is still possible to overwrite the defaults
% using explicit options in \includegraphics[width, height, ...]{}
\setkeys{Gin}{width=\maxwidth,height=\maxheight,keepaspectratio}
\IfFileExists{parskip.sty}{%
\usepackage{parskip}
}{% else
\setlength{\parindent}{0pt}
\setlength{\parskip}{6pt plus 2pt minus 1pt}
}
\setlength{\emergencystretch}{3em}  % prevent overfull lines
\providecommand{\tightlist}{%
  \setlength{\itemsep}{0pt}\setlength{\parskip}{0pt}}
\setcounter{secnumdepth}{0}
% Redefines (sub)paragraphs to behave more like sections
\ifx\paragraph\undefined\else
\let\oldparagraph\paragraph
\renewcommand{\paragraph}[1]{\oldparagraph{#1}\mbox{}}
\fi
\ifx\subparagraph\undefined\else
\let\oldsubparagraph\subparagraph
\renewcommand{\subparagraph}[1]{\oldsubparagraph{#1}\mbox{}}
\fi

%%% Use protect on footnotes to avoid problems with footnotes in titles
\let\rmarkdownfootnote\footnote%
\def\footnote{\protect\rmarkdownfootnote}

%%% Change title format to be more compact
\usepackage{titling}

% Create subtitle command for use in maketitle
\newcommand{\subtitle}[1]{
  \posttitle{
    \begin{center}\large#1\end{center}
    }
}

\setlength{\droptitle}{-2em}

  \title{Introduction to Huxtable}
    \pretitle{\vspace{\droptitle}\centering\huge}
  \posttitle{\par}
    \author{David Hugh-Jones}
    \preauthor{\centering\large\emph}
  \postauthor{\par}
      \predate{\centering\large\emph}
  \postdate{\par}
    \date{2018-10-26}

\usepackage{placeins}
\usepackage{pmboxdraw}
\usepackage{array}
\usepackage{caption}
\usepackage{graphicx}
\usepackage{siunitx}
\usepackage[table]{xcolor}
\usepackage{multirow}
\usepackage{hhline}
\usepackage{calc}
\usepackage{tabularx}

\begin{document}
\maketitle

{
\setcounter{tocdepth}{2}
\tableofcontents
}
 \begin{table}[h]
\centering
    \providecommand{\huxb}[2][0,0,0]{\arrayrulecolor[RGB]{#1}\global\arrayrulewidth=#2pt}
    \providecommand{\huxvb}[2][0,0,0]{\color[RGB]{#1}\vrule width #2pt}
    \providecommand{\huxtpad}[1]{\rule{0pt}{\baselineskip+#1}}
    \providecommand{\huxbpad}[1]{\rule[-#1]{0pt}{#1}}
  \begin{tabularx}{0.5\textwidth}{p{0.0833333333333333\textwidth} p{0.0833333333333333\textwidth} p{0.0833333333333333\textwidth} p{0.0833333333333333\textwidth} p{0.0833333333333333\textwidth} p{0.0833333333333333\textwidth}}


\hhline{>{\huxb[0, 0, 0]{1.2}}->{\huxb[0, 0, 0]{1.2}}->{\huxb[0, 0, 0]{1.2}}->{\huxb[0, 0, 0]{1.2}}->{\huxb[0, 0, 0]{1.2}}->{\huxb[0, 0, 0]{1.2}}-}
\arrayrulecolor{black}

\multicolumn{1}{!{\huxvb[0, 0, 0]{1.2}}c!{\huxvb[0, 0, 0]{1.2}}}{\huxtpad{0pt}\centering {\fontfamily{cmss}\selectfont }\huxbpad{0pt}} &
\multicolumn{1}{c!{\huxvb[0, 0, 0]{1.2}}}{\cellcolor[RGB]{255, 255, 0}\huxtpad{0pt}\centering {\fontfamily{cmss}\selectfont }\huxbpad{0pt}} &
\multicolumn{1}{c!{\huxvb[0, 0, 0]{1.2}}}{\cellcolor[RGB]{0, 0, 255}\huxtpad{0pt}\centering {\fontfamily{cmss}\selectfont \textcolor[RGB]{255, 255, 255}{}}\huxbpad{0pt}} &
\multicolumn{2}{c!{\huxvb[0, 0, 0]{1.2}}}{\cellcolor[RGB]{255, 255, 0}\huxtpad{0pt}\centering {\fontfamily{cmss}\selectfont }\huxbpad{0pt}} &
\multicolumn{1}{c!{\huxvb[0, 0, 0]{1.2}}}{\cellcolor[RGB]{0, 0, 255}} \tabularnewline[-0.5pt]


\hhline{>{\huxb[0, 0, 0]{1.2}}->{\huxb[0, 0, 0]{1.2}}->{\huxb[0, 0, 0]{1.2}}->{\huxb[0, 0, 0]{1.2}}->{\huxb[0, 0, 0]{1.2}}->{\huxb[0, 0, 255]{1.2}}->{\huxb[0, 0, 0]{1.2}}|}
\arrayrulecolor{black}

\multicolumn{1}{!{\huxvb[0, 0, 0]{1.2}}c!{\huxvb[0, 0, 0]{1.2}}}{\huxtpad{0pt}\centering {\fontfamily{cmss}\selectfont \textbf{h}}\huxbpad{0pt}} &
\multicolumn{1}{c!{\huxvb[0, 0, 0]{1.2}}}{\cellcolor[RGB]{0, 0, 255}\huxtpad{0pt}\centering {\fontfamily{cmss}\selectfont \textcolor[RGB]{255, 255, 255}{}}\huxbpad{0pt}} &
\multicolumn{1}{c!{\huxvb[0, 0, 0]{1.2}}}{\cellcolor[RGB]{255, 0, 0}\huxtpad{0pt}\centering {\fontfamily{cmss}\selectfont u}\huxbpad{0pt}} &
\multicolumn{1}{c!{\huxvb[0, 0, 0]{1.2}}}{\huxtpad{0pt}\centering {\fontfamily{cmss}\selectfont }\huxbpad{0pt}} &
\multicolumn{1}{c!{\huxvb[0, 0, 0]{1.2}}}{\huxtpad{0pt}\centering {\fontfamily{cmss}\selectfont }\huxbpad{0pt}} &
\multicolumn{1}{c!{\huxvb[0, 0, 0]{1.2}}}{\multirow{-2}{*}[0ex]{\cellcolor[RGB]{0, 0, 255}\huxtpad{0pt}\centering {\fontfamily{cmss}\selectfont \textcolor[RGB]{255, 255, 255}{}}\huxbpad{0pt}}} \tabularnewline[-0.5pt]


\hhline{>{\huxb[0, 0, 0]{1.2}}->{\huxb[0, 0, 0]{1.2}}->{\huxb[0, 0, 0]{1.2}}->{\huxb[0, 0, 0]{1.2}}->{\huxb[0, 0, 0]{1.2}}->{\huxb[0, 0, 0]{1.2}}-}
\arrayrulecolor{black}

\multicolumn{1}{!{\huxvb[0, 0, 0]{1.2}}c!{\huxvb[0, 0, 0]{1.2}}}{\huxtpad{0pt}\centering {\fontfamily{cmss}\selectfont x}\huxbpad{0pt}} &
\multicolumn{1}{c!{\huxvb[0, 0, 0]{1.2}}}{\cellcolor[RGB]{0, 0, 255}} &
\multicolumn{1}{c!{\huxvb[0, 0, 0]{1.2}}}{\huxtpad{0pt}\centering {\fontfamily{cmss}\selectfont }\huxbpad{0pt}} &
\multicolumn{1}{c!{\huxvb[0, 0, 0]{1.2}}}{\huxtpad{0pt}\centering {\fontfamily{cmss}\selectfont }\huxbpad{0pt}} &
\multicolumn{1}{c!{\huxvb[0, 0, 0]{1.2}}}{\huxtpad{0pt}\centering {\fontfamily{cmss}\selectfont }\huxbpad{0pt}} &
\multicolumn{1}{c!{\huxvb[0, 0, 0]{1.2}}}{\huxtpad{0pt}\centering {\fontfamily{cmss}\selectfont }\huxbpad{0pt}} \tabularnewline[-0.5pt]


\hhline{>{\huxb[0, 0, 0]{1.2}}->{\huxb[0, 0, 255]{1.2}}->{\huxb[0, 0, 0]{1.2}}|>{\huxb[0, 0, 0]{1.2}}->{\huxb[0, 0, 0]{1.2}}->{\huxb[0, 0, 0]{1.2}}->{\huxb[0, 0, 0]{1.2}}-}
\arrayrulecolor{black}

\multicolumn{1}{!{\huxvb[0, 0, 0]{1.2}}c!{\huxvb[0, 0, 0]{1.2}}}{\huxtpad{0pt}\centering {\fontfamily{cmss}\selectfont t}\huxbpad{0pt}} &
\multicolumn{1}{c!{\huxvb[0, 0, 0]{1.2}}}{\multirow{-2}{*}[0ex]{\cellcolor[RGB]{0, 0, 255}\huxtpad{0pt}\centering {\fontfamily{cmss}\selectfont \textcolor[RGB]{255, 255, 255}{}}\huxbpad{0pt}}} &
\multicolumn{1}{c!{\huxvb[0, 0, 0]{1.2}}}{\huxtpad{0pt}\centering {\fontfamily{cmss}\selectfont }\huxbpad{0pt}} &
\multicolumn{1}{c!{\huxvb[0, 0, 0]{1.2}}}{\huxtpad{0pt}\centering {\fontfamily{cmss}\selectfont a}\huxbpad{0pt}} &
\multicolumn{1}{c!{\huxvb[0, 0, 0]{1.2}}}{} &
\multicolumn{1}{c!{\huxvb[0, 0, 0]{1.2}}}{\huxtpad{0pt}\centering {\fontfamily{cmss}\selectfont b}\huxbpad{0pt}} \tabularnewline[-0.5pt]


\hhline{>{\huxb[0, 0, 0]{1.2}}->{\huxb[0, 0, 0]{1.2}}->{\huxb[0, 0, 0]{1.2}}->{\huxb[0, 0, 0]{1.2}}->{\huxb[255, 255, 255]{1.2}}->{\huxb[0, 0, 0]{1.2}}|>{\huxb[0, 0, 0]{1.2}}-}
\arrayrulecolor{black}

\multicolumn{1}{!{\huxvb[0, 0, 0]{1.2}}c!{\huxvb[0, 0, 0]{1.2}}}{\huxtpad{0pt}\centering {\fontfamily{cmss}\selectfont }\huxbpad{0pt}} &
\multicolumn{1}{c!{\huxvb[0, 0, 0]{1.2}}}{\huxtpad{0pt}\centering {\fontfamily{cmss}\selectfont }\huxbpad{0pt}} &
\multicolumn{1}{c!{\huxvb[0, 0, 0]{1.2}}}{\huxtpad{0pt}\centering {\fontfamily{cmss}\selectfont }\huxbpad{0pt}} &
\multicolumn{1}{c!{\huxvb[0, 0, 0]{1.2}}}{\huxtpad{0pt}\centering {\fontfamily{cmss}\selectfont }\huxbpad{0pt}} &
\multicolumn{1}{c!{\huxvb[0, 0, 0]{1.2}}}{\multirow{-2}{*}[0ex]{\huxtpad{0pt}\centering {\fontfamily{cmss}\selectfont }\huxbpad{0pt}}} &
\multicolumn{1}{c!{\huxvb[0, 0, 0]{1.2}}}{\huxtpad{0pt}\centering {\fontfamily{cmss}\selectfont }\huxbpad{0pt}} \tabularnewline[-0.5pt]


\hhline{>{\huxb[0, 0, 0]{1.2}}->{\huxb[0, 0, 0]{1.2}}->{\huxb[0, 0, 0]{1.2}}->{\huxb[0, 0, 0]{1.2}}->{\huxb[0, 0, 0]{1.2}}->{\huxb[0, 0, 0]{1.2}}-}
\arrayrulecolor{black}

\multicolumn{1}{!{\huxvb[0, 0, 0]{1.2}}c!{\huxvb[0, 0, 0]{1.2}}}{\huxtpad{0pt}\centering {\fontfamily{cmss}\selectfont l}\huxbpad{0pt}} &
\multicolumn{1}{c!{\huxvb[0, 0, 0]{1.2}}}{\cellcolor[RGB]{255, 0, 0}\huxtpad{0pt}\centering {\fontfamily{cmss}\selectfont }\huxbpad{0pt}} &
\multicolumn{1}{c!{\huxvb[0, 0, 0]{1.2}}}{\huxtpad{0pt}\centering {\fontfamily{cmss}\selectfont }\huxbpad{0pt}} &
\multicolumn{2}{c!{\huxvb[0, 0, 0]{1.2}}}{\huxtpad{0pt}\centering {\fontfamily{cmss}\selectfont }\huxbpad{0pt}} &
\multicolumn{1}{c!{\huxvb[0, 0, 0]{1.2}}}{\huxtpad{0pt}\centering {\fontfamily{cmss}\selectfont e}\huxbpad{0pt}} \tabularnewline[-0.5pt]


\hhline{>{\huxb[0, 0, 0]{1.2}}->{\huxb[0, 0, 0]{1.2}}->{\huxb[0, 0, 0]{1.2}}->{\huxb[0, 0, 0]{1.2}}->{\huxb[0, 0, 0]{1.2}}->{\huxb[0, 0, 0]{1.2}}-}
\arrayrulecolor{black}
\end{tabularx}
\end{table}
 

\FloatBarrier

\hypertarget{introduction}{%
\section{Introduction}\label{introduction}}

\hypertarget{about-this-document}{%
\subsection{About this document}\label{about-this-document}}

This is the introductory vignette for the R package `huxtable', version
4.2.1. A current version is available on the web in
\href{https://hughjonesd.github.io/huxtable/huxtable.html}{HTML} or
\href{https://hughjonesd.github.io/huxtable/huxtable.pdf}{PDF} format.

\hypertarget{huxtable}{%
\subsection{Huxtable}\label{huxtable}}

Huxtable is a package for creating \emph{text tables}. It is powerful,
but easy to use. It is meant to be a replacement for packages like
xtable, which is useful but not always very user-friendly. Huxtable's
features include:

\begin{itemize}
\tightlist
\item
  Export to LaTeX, HTML, Microsoft Word, Microsoft Excel, Microsoft
  Powerpoint, RTF and Markdown
\item
  Easy integration with knitr and rmarkdown documents
\item
  Formatted on-screen display
\item
  Multirow and multicolumn cells
\item
  Fine-grained control over cell background, spacing, alignment, size
  and borders
\item
  Control over text font, style, size, colour, alignment, number format
  and rotation
\item
  Table manipulation using standard R subsetting, or dplyr functions
  like \texttt{filter} and \texttt{select}
\item
  Easy conditional formatting based on table contents
\item
  Quick table themes
\item
  Automatic creation of regression output tables with the
  \texttt{huxreg} function
\end{itemize}

We will cover all of these features below.

\hypertarget{installation}{%
\subsection{Installation}\label{installation}}

If you haven't already installed huxtable, you can do so from the R
command line:

\begin{Shaded}
\begin{Highlighting}[]
\KeywordTok{install.packages}\NormalTok{(}\StringTok{'huxtable'}\NormalTok{)}
\end{Highlighting}
\end{Shaded}

\FloatBarrier

\hypertarget{getting-started}{%
\subsection{Getting started}\label{getting-started}}

A huxtable is a way of representing a table of text data in R. You
already know that R can represent a table of data in a data frame. For
example, if \texttt{mydata} is a data frame, then
\texttt{mydata{[}1,\ 2{]}} represents the the data in row 1, column 2,
and \texttt{mydata\$start\_time} is all the data in the column called
\texttt{start\_time}.

A huxtable is just a data frame with some extra properties. So, if
\texttt{myhux} is a huxtable, then \texttt{myhux{[}1,\ 2{]}} represents
the data in row 1 column 2, as before. But this cell will also have some
other properties - for example, the font size of the text, or the colour
of the cell border.

To create a table with huxtable, use the function \texttt{huxtable}, or
\texttt{hux} for short. This works very much like \texttt{data.frame}.

\begin{Shaded}
\begin{Highlighting}[]
\KeywordTok{library}\NormalTok{(huxtable)}
\NormalTok{ht <-}\StringTok{ }\KeywordTok{hux}\NormalTok{(}
        \DataTypeTok{Employee     =} \KeywordTok{c}\NormalTok{(}\StringTok{'John Smith'}\NormalTok{, }\StringTok{'Jane Doe'}\NormalTok{, }\StringTok{'David Hugh-Jones'}\NormalTok{), }
        \DataTypeTok{Salary       =} \KeywordTok{c}\NormalTok{(}\DecValTok{50000}\NormalTok{, }\DecValTok{50000}\NormalTok{, }\DecValTok{40000}\NormalTok{),}
        \DataTypeTok{add_colnames =} \OtherTok{TRUE}
\NormalTok{      )}
\end{Highlighting}
\end{Shaded}

\FloatBarrier

If you already have your data in a data frame, you can convert it to a
huxtable with \texttt{as\_hux}.

\begin{Shaded}
\begin{Highlighting}[]
\KeywordTok{data}\NormalTok{(mtcars)}
\NormalTok{car_ht <-}\StringTok{ }\KeywordTok{as_hux}\NormalTok{(mtcars)}
\end{Highlighting}
\end{Shaded}

\FloatBarrier

You can input your data row-by-row using the \texttt{tribble\_hux}
function. This can be more readable:

\begin{Shaded}
\begin{Highlighting}[]
\KeywordTok{tribble_hux}\NormalTok{(}
  \OperatorTok{~}\NormalTok{Employee,          }\OperatorTok{~}\NormalTok{Salary,}
  \StringTok{"John Smith"}\NormalTok{,       }\DecValTok{50000}\NormalTok{,}
  \StringTok{"Jane Doe"}\NormalTok{,         }\DecValTok{50000}\NormalTok{,}
  \StringTok{"David Hugh-Jones"}\NormalTok{, }\DecValTok{40000}\NormalTok{,}
  \DataTypeTok{add_colnames =} \OtherTok{TRUE}
\NormalTok{)}
\end{Highlighting}
\end{Shaded}

 \begin{table}[h]
\centering
    \providecommand{\huxb}[2][0,0,0]{\arrayrulecolor[RGB]{#1}\global\arrayrulewidth=#2pt}
    \providecommand{\huxvb}[2][0,0,0]{\color[RGB]{#1}\vrule width #2pt}
    \providecommand{\huxtpad}[1]{\rule{0pt}{\baselineskip+#1}}
    \providecommand{\huxbpad}[1]{\rule[-#1]{0pt}{#1}}
  \begin{tabularx}{0.5\textwidth}{p{0.25\textwidth} p{0.25\textwidth}}


\hhline{}
\arrayrulecolor{black}

\multicolumn{1}{!{\huxvb{0}}l!{\huxvb{0}}}{\huxtpad{4pt}\raggedright Employee\huxbpad{4pt}} &
\multicolumn{1}{r!{\huxvb{0}}}{\huxtpad{4pt}\raggedleft Salary\huxbpad{4pt}} \tabularnewline[-0.5pt]


\hhline{}
\arrayrulecolor{black}

\multicolumn{1}{!{\huxvb{0}}l!{\huxvb{0}}}{\huxtpad{4pt}\raggedright John Smith\huxbpad{4pt}} &
\multicolumn{1}{r!{\huxvb{0}}}{\huxtpad{4pt}\raggedleft 5e+04\huxbpad{4pt}} \tabularnewline[-0.5pt]


\hhline{}
\arrayrulecolor{black}

\multicolumn{1}{!{\huxvb{0}}l!{\huxvb{0}}}{\huxtpad{4pt}\raggedright Jane Doe\huxbpad{4pt}} &
\multicolumn{1}{r!{\huxvb{0}}}{\huxtpad{4pt}\raggedleft 5e+04\huxbpad{4pt}} \tabularnewline[-0.5pt]


\hhline{}
\arrayrulecolor{black}

\multicolumn{1}{!{\huxvb{0}}l!{\huxvb{0}}}{\huxtpad{4pt}\raggedright David Hugh-Jones\huxbpad{4pt}} &
\multicolumn{1}{r!{\huxvb{0}}}{\huxtpad{4pt}\raggedleft 4e+04\huxbpad{4pt}} \tabularnewline[-0.5pt]


\hhline{}
\arrayrulecolor{black}
\end{tabularx}
\end{table}
 

\FloatBarrier

If you look at a huxtable in R, it will print out a simple
representation of the data. Notice that we've added the column names to
the data frame itself, using the \texttt{add\_colnames} argument to
\texttt{hux}. We're going to print them out, so they need to be part of
the actual table. \textbf{NB:} This means that row 1 of your data will
be row 2 of the huxtable, and the column names of your data will be the
new row 1.

\begin{Shaded}
\begin{Highlighting}[]
\KeywordTok{print_screen}\NormalTok{(ht)     }\CommentTok{# on the R command line, you can just type "ht"}
\end{Highlighting}
\end{Shaded}

\begin{verbatim}
##   Employee            Salary  
##   John Smith           5e+04  
##   Jane Doe             5e+04  
##   David Hugh-Jones     4e+04  
## 
## Column names: Employee, Salary
\end{verbatim}

\FloatBarrier

To print a huxtable out using LaTeX or HTML, just call
\texttt{print\_latex} or \texttt{print\_html}. In knitr documents, like
this one, you can simply evaluate the hux. It will know what format to
print itself in.

\begin{Shaded}
\begin{Highlighting}[]
\NormalTok{ht}
\end{Highlighting}
\end{Shaded}

 \begin{table}[h]
\centering
    \providecommand{\huxb}[2][0,0,0]{\arrayrulecolor[RGB]{#1}\global\arrayrulewidth=#2pt}
    \providecommand{\huxvb}[2][0,0,0]{\color[RGB]{#1}\vrule width #2pt}
    \providecommand{\huxtpad}[1]{\rule{0pt}{\baselineskip+#1}}
    \providecommand{\huxbpad}[1]{\rule[-#1]{0pt}{#1}}
  \begin{tabularx}{0.5\textwidth}{p{0.25\textwidth} p{0.25\textwidth}}


\hhline{}
\arrayrulecolor{black}

\multicolumn{1}{!{\huxvb{0}}l!{\huxvb{0}}}{\huxtpad{4pt}\raggedright Employee\huxbpad{4pt}} &
\multicolumn{1}{r!{\huxvb{0}}}{\huxtpad{4pt}\raggedleft Salary\huxbpad{4pt}} \tabularnewline[-0.5pt]


\hhline{}
\arrayrulecolor{black}

\multicolumn{1}{!{\huxvb{0}}l!{\huxvb{0}}}{\huxtpad{4pt}\raggedright John Smith\huxbpad{4pt}} &
\multicolumn{1}{r!{\huxvb{0}}}{\huxtpad{4pt}\raggedleft 5e+04\huxbpad{4pt}} \tabularnewline[-0.5pt]


\hhline{}
\arrayrulecolor{black}

\multicolumn{1}{!{\huxvb{0}}l!{\huxvb{0}}}{\huxtpad{4pt}\raggedright Jane Doe\huxbpad{4pt}} &
\multicolumn{1}{r!{\huxvb{0}}}{\huxtpad{4pt}\raggedleft 5e+04\huxbpad{4pt}} \tabularnewline[-0.5pt]


\hhline{}
\arrayrulecolor{black}

\multicolumn{1}{!{\huxvb{0}}l!{\huxvb{0}}}{\huxtpad{4pt}\raggedright David Hugh-Jones\huxbpad{4pt}} &
\multicolumn{1}{r!{\huxvb{0}}}{\huxtpad{4pt}\raggedleft 4e+04\huxbpad{4pt}} \tabularnewline[-0.5pt]


\hhline{}
\arrayrulecolor{black}
\end{tabularx}
\end{table}
 

\FloatBarrier

\hypertarget{changing-the-look-and-feel}{%
\section{Changing the look and feel}\label{changing-the-look-and-feel}}

\hypertarget{huxtable-properties}{%
\subsection{Huxtable properties}\label{huxtable-properties}}

The default output is a very plain table. Let's make it a bit smarter.
We'll make the table headings bold, draw a line under the header row,
and add some horizontal space to the cells. We also need to change that
default number formatting to look less scientific.

To do this, we need to set some \textbf{properties} on the table cells.
You set properties by assigning to the property name, just as you assign
\texttt{names(x)\ \textless{}-\ new\_names} in base R. The following
commands assign the value 10 to the \texttt{right\_padding} and
\texttt{left\_padding} properties, for all cells in \texttt{ht}:

\begin{Shaded}
\begin{Highlighting}[]
\KeywordTok{right_padding}\NormalTok{(ht) <-}\StringTok{ }\DecValTok{10}
\KeywordTok{left_padding}\NormalTok{(ht)  <-}\StringTok{ }\DecValTok{10}
\end{Highlighting}
\end{Shaded}

\FloatBarrier

Similarly, we can set the \texttt{number\_format} property to change how
numbers are displayed in cells:

\begin{Shaded}
\begin{Highlighting}[]
\KeywordTok{number_format}\NormalTok{(ht) <-}\StringTok{ }\DecValTok{2}    \CommentTok{# 2 decimal places}
\end{Highlighting}
\end{Shaded}

\FloatBarrier

To assign properties to just some cells, you use subsetting, as in base
R. So, to make the first row of the table \textbf{bold} and give it a
bottom border, we do:

\begin{Shaded}
\begin{Highlighting}[]
\KeywordTok{bold}\NormalTok{(ht)[}\DecValTok{1}\NormalTok{, ]          <-}\StringTok{ }\OtherTok{TRUE}
\KeywordTok{bottom_border}\NormalTok{(ht)[}\DecValTok{1}\NormalTok{, ] <-}\StringTok{ }\DecValTok{1}
\end{Highlighting}
\end{Shaded}

\FloatBarrier

After these changes, our table looks smarter:

\begin{Shaded}
\begin{Highlighting}[]
\NormalTok{ht}
\end{Highlighting}
\end{Shaded}

 \begin{table}[h]
\centering
    \providecommand{\huxb}[2][0,0,0]{\arrayrulecolor[RGB]{#1}\global\arrayrulewidth=#2pt}
    \providecommand{\huxvb}[2][0,0,0]{\color[RGB]{#1}\vrule width #2pt}
    \providecommand{\huxtpad}[1]{\rule{0pt}{\baselineskip+#1}}
    \providecommand{\huxbpad}[1]{\rule[-#1]{0pt}{#1}}
  \begin{tabularx}{0.5\textwidth}{p{0.25\textwidth} p{0.25\textwidth}}


\hhline{}
\arrayrulecolor{black}

\multicolumn{1}{!{\huxvb{0}}l!{\huxvb{0}}}{\huxtpad{4pt}\raggedright \textbf{Employee}\huxbpad{4pt}} &
\multicolumn{1}{r!{\huxvb{0}}}{\huxtpad{4pt}\raggedleft \textbf{Salary}\huxbpad{4pt}} \tabularnewline[-0.5pt]


\hhline{>{\huxb{1}}->{\huxb{1}}-}
\arrayrulecolor{black}

\multicolumn{1}{!{\huxvb{0}}l!{\huxvb{0}}}{\huxtpad{4pt}\raggedright John Smith\huxbpad{4pt}} &
\multicolumn{1}{r!{\huxvb{0}}}{\huxtpad{4pt}\raggedleft 50000.00\huxbpad{4pt}} \tabularnewline[-0.5pt]


\hhline{}
\arrayrulecolor{black}

\multicolumn{1}{!{\huxvb{0}}l!{\huxvb{0}}}{\huxtpad{4pt}\raggedright Jane Doe\huxbpad{4pt}} &
\multicolumn{1}{r!{\huxvb{0}}}{\huxtpad{4pt}\raggedleft 50000.00\huxbpad{4pt}} \tabularnewline[-0.5pt]


\hhline{}
\arrayrulecolor{black}

\multicolumn{1}{!{\huxvb{0}}l!{\huxvb{0}}}{\huxtpad{4pt}\raggedright David Hugh-Jones\huxbpad{4pt}} &
\multicolumn{1}{r!{\huxvb{0}}}{\huxtpad{4pt}\raggedleft 40000.00\huxbpad{4pt}} \tabularnewline[-0.5pt]


\hhline{}
\arrayrulecolor{black}
\end{tabularx}
\end{table}
 

\FloatBarrier

So far, all these properties have been set at cell level. Different
cells can have different alignment, text formatting and so on. By
contrast, \texttt{caption} is a table-level property. It only takes one
value, which sets a table caption.

\begin{Shaded}
\begin{Highlighting}[]
\KeywordTok{caption}\NormalTok{(ht) <-}\StringTok{ 'Employee table'}
\NormalTok{ht}
\end{Highlighting}
\end{Shaded}

 \begin{table}[h]
\centering\captionsetup{justification=centering,singlelinecheck=off}
\caption{Employee table}

    \providecommand{\huxb}[2][0,0,0]{\arrayrulecolor[RGB]{#1}\global\arrayrulewidth=#2pt}
    \providecommand{\huxvb}[2][0,0,0]{\color[RGB]{#1}\vrule width #2pt}
    \providecommand{\huxtpad}[1]{\rule{0pt}{\baselineskip+#1}}
    \providecommand{\huxbpad}[1]{\rule[-#1]{0pt}{#1}}
  \begin{tabularx}{0.5\textwidth}{p{0.25\textwidth} p{0.25\textwidth}}


\hhline{}
\arrayrulecolor{black}

\multicolumn{1}{!{\huxvb{0}}l!{\huxvb{0}}}{\huxtpad{4pt}\raggedright \textbf{Employee}\huxbpad{4pt}} &
\multicolumn{1}{r!{\huxvb{0}}}{\huxtpad{4pt}\raggedleft \textbf{Salary}\huxbpad{4pt}} \tabularnewline[-0.5pt]


\hhline{>{\huxb{1}}->{\huxb{1}}-}
\arrayrulecolor{black}

\multicolumn{1}{!{\huxvb{0}}l!{\huxvb{0}}}{\huxtpad{4pt}\raggedright John Smith\huxbpad{4pt}} &
\multicolumn{1}{r!{\huxvb{0}}}{\huxtpad{4pt}\raggedleft 50000.00\huxbpad{4pt}} \tabularnewline[-0.5pt]


\hhline{}
\arrayrulecolor{black}

\multicolumn{1}{!{\huxvb{0}}l!{\huxvb{0}}}{\huxtpad{4pt}\raggedright Jane Doe\huxbpad{4pt}} &
\multicolumn{1}{r!{\huxvb{0}}}{\huxtpad{4pt}\raggedleft 50000.00\huxbpad{4pt}} \tabularnewline[-0.5pt]


\hhline{}
\arrayrulecolor{black}

\multicolumn{1}{!{\huxvb{0}}l!{\huxvb{0}}}{\huxtpad{4pt}\raggedright David Hugh-Jones\huxbpad{4pt}} &
\multicolumn{1}{r!{\huxvb{0}}}{\huxtpad{4pt}\raggedleft 40000.00\huxbpad{4pt}} \tabularnewline[-0.5pt]


\hhline{}
\arrayrulecolor{black}
\end{tabularx}
\end{table}
 

\FloatBarrier

As well as cell properties and table properties, there is also one row
property, \texttt{row\_height}, and one column property,
\texttt{col\_width}.

The table below shows a complete list of properties. Most properties
work the same for LaTeX and HTML, though there are some exceptions.

 \begin{table}[h]
\begin{raggedright}\captionsetup{justification=raggedright,singlelinecheck=off}
\caption{Huxtable properties}

    \providecommand{\huxb}[2][0,0,0]{\arrayrulecolor[RGB]{#1}\global\arrayrulewidth=#2pt}
    \providecommand{\huxvb}[2][0,0,0]{\color[RGB]{#1}\vrule width #2pt}
    \providecommand{\huxtpad}[1]{\rule{0pt}{\baselineskip+#1}}
    \providecommand{\huxbpad}[1]{\rule[-#1]{0pt}{#1}}
  \begin{tabularx}{0.9\textwidth}{p{0.18\textwidth} p{0.225\textwidth} p{0.135\textwidth} p{0.135\textwidth} p{0.225\textwidth}}


\hhline{>{\huxb{1}}->{\huxb{1}}->{\huxb{1}}->{\huxb{1}}->{\huxb{1}}-}
\arrayrulecolor{black}

\multicolumn{1}{!{\huxvb{1}}l!{\huxvb{0}}}{\huxtpad{2pt}\raggedright \textbf{Cell Text}\huxbpad{4pt}} &
\multicolumn{1}{l!{\huxvb{0}}}{\cellcolor[RGB]{230, 230, 230}\huxtpad{2pt}\raggedright \textbf{Cell}\huxbpad{4pt}} &
\multicolumn{1}{l!{\huxvb{0}}}{\huxtpad{2pt}\raggedright \textbf{Row}\huxbpad{4pt}} &
\multicolumn{1}{l!{\huxvb{0}}}{\cellcolor[RGB]{230, 230, 230}\huxtpad{2pt}\raggedright \textbf{Column}\huxbpad{4pt}} &
\multicolumn{1}{l!{\huxvb{1}}}{\huxtpad{2pt}\raggedright \textbf{Table}\huxbpad{4pt}} \tabularnewline[-0.5pt]


\hhline{>{\huxb{1}}->{\huxb{1}}->{\huxb{1}}->{\huxb{1}}->{\huxb{1}}-}
\arrayrulecolor{black}

\multicolumn{1}{!{\huxvb{1}}l!{\huxvb{0}}}{\huxtpad{2pt}\raggedright {\fontfamily{cmtt}\selectfont bold}\huxbpad{4pt}} &
\multicolumn{1}{l!{\huxvb{0}}}{\cellcolor[RGB]{230, 230, 230}\huxtpad{2pt}\raggedright {\fontfamily{cmtt}\selectfont align}\huxbpad{4pt}} &
\multicolumn{1}{l!{\huxvb{0}}}{\huxtpad{2pt}\raggedright {\fontfamily{cmtt}\selectfont row\_height}\huxbpad{4pt}} &
\multicolumn{1}{l!{\huxvb{0}}}{\cellcolor[RGB]{230, 230, 230}\huxtpad{2pt}\raggedright {\fontfamily{cmtt}\selectfont col\_width}\huxbpad{4pt}} &
\multicolumn{1}{l!{\huxvb{1}}}{\huxtpad{2pt}\raggedright {\fontfamily{cmtt}\selectfont caption}\huxbpad{4pt}} \tabularnewline[-0.5pt]


\hhline{>{\huxb{1}}|>{\huxb{1}}|}
\arrayrulecolor{black}

\multicolumn{1}{!{\huxvb{1}}l!{\huxvb{0}}}{\huxtpad{2pt}\raggedright {\fontfamily{cmtt}\selectfont escape\_contents}\huxbpad{4pt}} &
\multicolumn{1}{l!{\huxvb{0}}}{\cellcolor[RGB]{230, 230, 230}\huxtpad{2pt}\raggedright {\fontfamily{cmtt}\selectfont background\_color}\huxbpad{4pt}} &
\multicolumn{1}{l!{\huxvb{0}}}{\huxtpad{2pt}\raggedright {\fontfamily{cmtt}\selectfont }\huxbpad{4pt}} &
\multicolumn{1}{l!{\huxvb{0}}}{\cellcolor[RGB]{230, 230, 230}\huxtpad{2pt}\raggedright {\fontfamily{cmtt}\selectfont }\huxbpad{4pt}} &
\multicolumn{1}{l!{\huxvb{1}}}{\huxtpad{2pt}\raggedright {\fontfamily{cmtt}\selectfont caption\_pos}\huxbpad{4pt}} \tabularnewline[-0.5pt]


\hhline{>{\huxb{1}}|>{\huxb{1}}|}
\arrayrulecolor{black}

\multicolumn{1}{!{\huxvb{1}}l!{\huxvb{0}}}{\huxtpad{2pt}\raggedright {\fontfamily{cmtt}\selectfont font}\huxbpad{4pt}} &
\multicolumn{1}{l!{\huxvb{0}}}{\cellcolor[RGB]{230, 230, 230}\huxtpad{2pt}\raggedright {\fontfamily{cmtt}\selectfont bottom\_border}\huxbpad{4pt}} &
\multicolumn{1}{l!{\huxvb{0}}}{\huxtpad{2pt}\raggedright {\fontfamily{cmtt}\selectfont }\huxbpad{4pt}} &
\multicolumn{1}{l!{\huxvb{0}}}{\cellcolor[RGB]{230, 230, 230}\huxtpad{2pt}\raggedright {\fontfamily{cmtt}\selectfont }\huxbpad{4pt}} &
\multicolumn{1}{l!{\huxvb{1}}}{\huxtpad{2pt}\raggedright {\fontfamily{cmtt}\selectfont height}\huxbpad{4pt}} \tabularnewline[-0.5pt]


\hhline{>{\huxb{1}}|>{\huxb{1}}|}
\arrayrulecolor{black}

\multicolumn{1}{!{\huxvb{1}}l!{\huxvb{0}}}{\huxtpad{2pt}\raggedright {\fontfamily{cmtt}\selectfont font}\huxbpad{4pt}} &
\multicolumn{1}{l!{\huxvb{0}}}{\cellcolor[RGB]{230, 230, 230}\huxtpad{2pt}\raggedright {\fontfamily{cmtt}\selectfont bottom\_border\_color}\huxbpad{4pt}} &
\multicolumn{1}{l!{\huxvb{0}}}{\huxtpad{2pt}\raggedright {\fontfamily{cmtt}\selectfont }\huxbpad{4pt}} &
\multicolumn{1}{l!{\huxvb{0}}}{\cellcolor[RGB]{230, 230, 230}\huxtpad{2pt}\raggedright {\fontfamily{cmtt}\selectfont }\huxbpad{4pt}} &
\multicolumn{1}{l!{\huxvb{1}}}{\huxtpad{2pt}\raggedright {\fontfamily{cmtt}\selectfont label}\huxbpad{4pt}} \tabularnewline[-0.5pt]


\hhline{>{\huxb{1}}|>{\huxb{1}}|}
\arrayrulecolor{black}

\multicolumn{1}{!{\huxvb{1}}l!{\huxvb{0}}}{\huxtpad{2pt}\raggedright {\fontfamily{cmtt}\selectfont font\_size}\huxbpad{4pt}} &
\multicolumn{1}{l!{\huxvb{0}}}{\cellcolor[RGB]{230, 230, 230}\huxtpad{2pt}\raggedright {\fontfamily{cmtt}\selectfont bottom\_border\_style}\huxbpad{4pt}} &
\multicolumn{1}{l!{\huxvb{0}}}{\huxtpad{2pt}\raggedright {\fontfamily{cmtt}\selectfont }\huxbpad{4pt}} &
\multicolumn{1}{l!{\huxvb{0}}}{\cellcolor[RGB]{230, 230, 230}\huxtpad{2pt}\raggedright {\fontfamily{cmtt}\selectfont }\huxbpad{4pt}} &
\multicolumn{1}{l!{\huxvb{1}}}{\huxtpad{2pt}\raggedright {\fontfamily{cmtt}\selectfont latex\_float}\huxbpad{4pt}} \tabularnewline[-0.5pt]


\hhline{>{\huxb{1}}|>{\huxb{1}}|}
\arrayrulecolor{black}

\multicolumn{1}{!{\huxvb{1}}l!{\huxvb{0}}}{\huxtpad{2pt}\raggedright {\fontfamily{cmtt}\selectfont italic}\huxbpad{4pt}} &
\multicolumn{1}{l!{\huxvb{0}}}{\cellcolor[RGB]{230, 230, 230}\huxtpad{2pt}\raggedright {\fontfamily{cmtt}\selectfont bottom\_padding}\huxbpad{4pt}} &
\multicolumn{1}{l!{\huxvb{0}}}{\huxtpad{2pt}\raggedright {\fontfamily{cmtt}\selectfont }\huxbpad{4pt}} &
\multicolumn{1}{l!{\huxvb{0}}}{\cellcolor[RGB]{230, 230, 230}\huxtpad{2pt}\raggedright {\fontfamily{cmtt}\selectfont }\huxbpad{4pt}} &
\multicolumn{1}{l!{\huxvb{1}}}{\huxtpad{2pt}\raggedright {\fontfamily{cmtt}\selectfont position}\huxbpad{4pt}} \tabularnewline[-0.5pt]


\hhline{>{\huxb{1}}|>{\huxb{1}}|}
\arrayrulecolor{black}

\multicolumn{1}{!{\huxvb{1}}l!{\huxvb{0}}}{\huxtpad{2pt}\raggedright {\fontfamily{cmtt}\selectfont na\_string}\huxbpad{4pt}} &
\multicolumn{1}{l!{\huxvb{0}}}{\cellcolor[RGB]{230, 230, 230}\huxtpad{2pt}\raggedright {\fontfamily{cmtt}\selectfont colspan}\huxbpad{4pt}} &
\multicolumn{1}{l!{\huxvb{0}}}{\huxtpad{2pt}\raggedright {\fontfamily{cmtt}\selectfont }\huxbpad{4pt}} &
\multicolumn{1}{l!{\huxvb{0}}}{\cellcolor[RGB]{230, 230, 230}\huxtpad{2pt}\raggedright {\fontfamily{cmtt}\selectfont }\huxbpad{4pt}} &
\multicolumn{1}{l!{\huxvb{1}}}{\huxtpad{2pt}\raggedright {\fontfamily{cmtt}\selectfont tabular\_environment}\huxbpad{4pt}} \tabularnewline[-0.5pt]


\hhline{>{\huxb{1}}|>{\huxb{1}}|}
\arrayrulecolor{black}

\multicolumn{1}{!{\huxvb{1}}l!{\huxvb{0}}}{\huxtpad{2pt}\raggedright {\fontfamily{cmtt}\selectfont number\_format}\huxbpad{4pt}} &
\multicolumn{1}{l!{\huxvb{0}}}{\cellcolor[RGB]{230, 230, 230}\huxtpad{2pt}\raggedright {\fontfamily{cmtt}\selectfont left\_border}\huxbpad{4pt}} &
\multicolumn{1}{l!{\huxvb{0}}}{\huxtpad{2pt}\raggedright {\fontfamily{cmtt}\selectfont }\huxbpad{4pt}} &
\multicolumn{1}{l!{\huxvb{0}}}{\cellcolor[RGB]{230, 230, 230}\huxtpad{2pt}\raggedright {\fontfamily{cmtt}\selectfont }\huxbpad{4pt}} &
\multicolumn{1}{l!{\huxvb{1}}}{\huxtpad{2pt}\raggedright {\fontfamily{cmtt}\selectfont width}\huxbpad{4pt}} \tabularnewline[-0.5pt]


\hhline{>{\huxb{1}}|>{\huxb{1}}|}
\arrayrulecolor{black}

\multicolumn{1}{!{\huxvb{1}}l!{\huxvb{0}}}{\huxtpad{2pt}\raggedright {\fontfamily{cmtt}\selectfont rotation}\huxbpad{4pt}} &
\multicolumn{1}{l!{\huxvb{0}}}{\cellcolor[RGB]{230, 230, 230}\huxtpad{2pt}\raggedright {\fontfamily{cmtt}\selectfont left\_border\_color}\huxbpad{4pt}} &
\multicolumn{1}{l!{\huxvb{0}}}{\huxtpad{2pt}\raggedright {\fontfamily{cmtt}\selectfont }\huxbpad{4pt}} &
\multicolumn{1}{l!{\huxvb{0}}}{\cellcolor[RGB]{230, 230, 230}\huxtpad{2pt}\raggedright {\fontfamily{cmtt}\selectfont }\huxbpad{4pt}} &
\multicolumn{1}{l!{\huxvb{1}}}{\huxtpad{2pt}\raggedright {\fontfamily{cmtt}\selectfont }\huxbpad{4pt}} \tabularnewline[-0.5pt]


\hhline{>{\huxb{1}}|>{\huxb{1}}|}
\arrayrulecolor{black}

\multicolumn{1}{!{\huxvb{1}}l!{\huxvb{0}}}{\huxtpad{2pt}\raggedright {\fontfamily{cmtt}\selectfont text\_color}\huxbpad{4pt}} &
\multicolumn{1}{l!{\huxvb{0}}}{\cellcolor[RGB]{230, 230, 230}\huxtpad{2pt}\raggedright {\fontfamily{cmtt}\selectfont left\_border\_style}\huxbpad{4pt}} &
\multicolumn{1}{l!{\huxvb{0}}}{\huxtpad{2pt}\raggedright {\fontfamily{cmtt}\selectfont }\huxbpad{4pt}} &
\multicolumn{1}{l!{\huxvb{0}}}{\cellcolor[RGB]{230, 230, 230}\huxtpad{2pt}\raggedright {\fontfamily{cmtt}\selectfont }\huxbpad{4pt}} &
\multicolumn{1}{l!{\huxvb{1}}}{\huxtpad{2pt}\raggedright {\fontfamily{cmtt}\selectfont }\huxbpad{4pt}} \tabularnewline[-0.5pt]


\hhline{>{\huxb{1}}|>{\huxb{1}}|}
\arrayrulecolor{black}

\multicolumn{1}{!{\huxvb{1}}l!{\huxvb{0}}}{\huxtpad{2pt}\raggedright {\fontfamily{cmtt}\selectfont wrap}\huxbpad{4pt}} &
\multicolumn{1}{l!{\huxvb{0}}}{\cellcolor[RGB]{230, 230, 230}\huxtpad{2pt}\raggedright {\fontfamily{cmtt}\selectfont left\_padding}\huxbpad{4pt}} &
\multicolumn{1}{l!{\huxvb{0}}}{\huxtpad{2pt}\raggedright {\fontfamily{cmtt}\selectfont }\huxbpad{4pt}} &
\multicolumn{1}{l!{\huxvb{0}}}{\cellcolor[RGB]{230, 230, 230}\huxtpad{2pt}\raggedright {\fontfamily{cmtt}\selectfont }\huxbpad{4pt}} &
\multicolumn{1}{l!{\huxvb{1}}}{\huxtpad{2pt}\raggedright {\fontfamily{cmtt}\selectfont }\huxbpad{4pt}} \tabularnewline[-0.5pt]


\hhline{>{\huxb{1}}|>{\huxb{1}}|}
\arrayrulecolor{black}

\multicolumn{1}{!{\huxvb{1}}l!{\huxvb{0}}}{\huxtpad{2pt}\raggedright {\fontfamily{cmtt}\selectfont }\huxbpad{4pt}} &
\multicolumn{1}{l!{\huxvb{0}}}{\cellcolor[RGB]{230, 230, 230}\huxtpad{2pt}\raggedright {\fontfamily{cmtt}\selectfont right\_border}\huxbpad{4pt}} &
\multicolumn{1}{l!{\huxvb{0}}}{\huxtpad{2pt}\raggedright {\fontfamily{cmtt}\selectfont }\huxbpad{4pt}} &
\multicolumn{1}{l!{\huxvb{0}}}{\cellcolor[RGB]{230, 230, 230}\huxtpad{2pt}\raggedright {\fontfamily{cmtt}\selectfont }\huxbpad{4pt}} &
\multicolumn{1}{l!{\huxvb{1}}}{\huxtpad{2pt}\raggedright {\fontfamily{cmtt}\selectfont }\huxbpad{4pt}} \tabularnewline[-0.5pt]


\hhline{>{\huxb{1}}|>{\huxb{1}}|}
\arrayrulecolor{black}

\multicolumn{1}{!{\huxvb{1}}l!{\huxvb{0}}}{\huxtpad{2pt}\raggedright {\fontfamily{cmtt}\selectfont }\huxbpad{4pt}} &
\multicolumn{1}{l!{\huxvb{0}}}{\cellcolor[RGB]{230, 230, 230}\huxtpad{2pt}\raggedright {\fontfamily{cmtt}\selectfont right\_border\_color}\huxbpad{4pt}} &
\multicolumn{1}{l!{\huxvb{0}}}{\huxtpad{2pt}\raggedright {\fontfamily{cmtt}\selectfont }\huxbpad{4pt}} &
\multicolumn{1}{l!{\huxvb{0}}}{\cellcolor[RGB]{230, 230, 230}\huxtpad{2pt}\raggedright {\fontfamily{cmtt}\selectfont }\huxbpad{4pt}} &
\multicolumn{1}{l!{\huxvb{1}}}{\huxtpad{2pt}\raggedright {\fontfamily{cmtt}\selectfont }\huxbpad{4pt}} \tabularnewline[-0.5pt]


\hhline{>{\huxb{1}}|>{\huxb{1}}|}
\arrayrulecolor{black}

\multicolumn{1}{!{\huxvb{1}}l!{\huxvb{0}}}{\huxtpad{2pt}\raggedright {\fontfamily{cmtt}\selectfont }\huxbpad{4pt}} &
\multicolumn{1}{l!{\huxvb{0}}}{\cellcolor[RGB]{230, 230, 230}\huxtpad{2pt}\raggedright {\fontfamily{cmtt}\selectfont right\_border\_style}\huxbpad{4pt}} &
\multicolumn{1}{l!{\huxvb{0}}}{\huxtpad{2pt}\raggedright {\fontfamily{cmtt}\selectfont }\huxbpad{4pt}} &
\multicolumn{1}{l!{\huxvb{0}}}{\cellcolor[RGB]{230, 230, 230}\huxtpad{2pt}\raggedright {\fontfamily{cmtt}\selectfont }\huxbpad{4pt}} &
\multicolumn{1}{l!{\huxvb{1}}}{\huxtpad{2pt}\raggedright {\fontfamily{cmtt}\selectfont }\huxbpad{4pt}} \tabularnewline[-0.5pt]


\hhline{>{\huxb{1}}|>{\huxb{1}}|}
\arrayrulecolor{black}

\multicolumn{1}{!{\huxvb{1}}l!{\huxvb{0}}}{\huxtpad{2pt}\raggedright {\fontfamily{cmtt}\selectfont }\huxbpad{4pt}} &
\multicolumn{1}{l!{\huxvb{0}}}{\cellcolor[RGB]{230, 230, 230}\huxtpad{2pt}\raggedright {\fontfamily{cmtt}\selectfont right\_padding}\huxbpad{4pt}} &
\multicolumn{1}{l!{\huxvb{0}}}{\huxtpad{2pt}\raggedright {\fontfamily{cmtt}\selectfont }\huxbpad{4pt}} &
\multicolumn{1}{l!{\huxvb{0}}}{\cellcolor[RGB]{230, 230, 230}\huxtpad{2pt}\raggedright {\fontfamily{cmtt}\selectfont }\huxbpad{4pt}} &
\multicolumn{1}{l!{\huxvb{1}}}{\huxtpad{2pt}\raggedright {\fontfamily{cmtt}\selectfont }\huxbpad{4pt}} \tabularnewline[-0.5pt]


\hhline{>{\huxb{1}}|>{\huxb{1}}|}
\arrayrulecolor{black}

\multicolumn{1}{!{\huxvb{1}}l!{\huxvb{0}}}{\huxtpad{2pt}\raggedright {\fontfamily{cmtt}\selectfont }\huxbpad{4pt}} &
\multicolumn{1}{l!{\huxvb{0}}}{\cellcolor[RGB]{230, 230, 230}\huxtpad{2pt}\raggedright {\fontfamily{cmtt}\selectfont rowspan}\huxbpad{4pt}} &
\multicolumn{1}{l!{\huxvb{0}}}{\huxtpad{2pt}\raggedright {\fontfamily{cmtt}\selectfont }\huxbpad{4pt}} &
\multicolumn{1}{l!{\huxvb{0}}}{\cellcolor[RGB]{230, 230, 230}\huxtpad{2pt}\raggedright {\fontfamily{cmtt}\selectfont }\huxbpad{4pt}} &
\multicolumn{1}{l!{\huxvb{1}}}{\huxtpad{2pt}\raggedright {\fontfamily{cmtt}\selectfont }\huxbpad{4pt}} \tabularnewline[-0.5pt]


\hhline{>{\huxb{1}}|>{\huxb{1}}|}
\arrayrulecolor{black}

\multicolumn{1}{!{\huxvb{1}}l!{\huxvb{0}}}{\huxtpad{2pt}\raggedright {\fontfamily{cmtt}\selectfont }\huxbpad{4pt}} &
\multicolumn{1}{l!{\huxvb{0}}}{\cellcolor[RGB]{230, 230, 230}\huxtpad{2pt}\raggedright {\fontfamily{cmtt}\selectfont top\_border}\huxbpad{4pt}} &
\multicolumn{1}{l!{\huxvb{0}}}{\huxtpad{2pt}\raggedright {\fontfamily{cmtt}\selectfont }\huxbpad{4pt}} &
\multicolumn{1}{l!{\huxvb{0}}}{\cellcolor[RGB]{230, 230, 230}\huxtpad{2pt}\raggedright {\fontfamily{cmtt}\selectfont }\huxbpad{4pt}} &
\multicolumn{1}{l!{\huxvb{1}}}{\huxtpad{2pt}\raggedright {\fontfamily{cmtt}\selectfont }\huxbpad{4pt}} \tabularnewline[-0.5pt]


\hhline{>{\huxb{1}}|>{\huxb{1}}|}
\arrayrulecolor{black}

\multicolumn{1}{!{\huxvb{1}}l!{\huxvb{0}}}{\huxtpad{2pt}\raggedright {\fontfamily{cmtt}\selectfont }\huxbpad{4pt}} &
\multicolumn{1}{l!{\huxvb{0}}}{\cellcolor[RGB]{230, 230, 230}\huxtpad{2pt}\raggedright {\fontfamily{cmtt}\selectfont top\_border\_color}\huxbpad{4pt}} &
\multicolumn{1}{l!{\huxvb{0}}}{\huxtpad{2pt}\raggedright {\fontfamily{cmtt}\selectfont }\huxbpad{4pt}} &
\multicolumn{1}{l!{\huxvb{0}}}{\cellcolor[RGB]{230, 230, 230}\huxtpad{2pt}\raggedright {\fontfamily{cmtt}\selectfont }\huxbpad{4pt}} &
\multicolumn{1}{l!{\huxvb{1}}}{\huxtpad{2pt}\raggedright {\fontfamily{cmtt}\selectfont }\huxbpad{4pt}} \tabularnewline[-0.5pt]


\hhline{>{\huxb{1}}|>{\huxb{1}}|}
\arrayrulecolor{black}

\multicolumn{1}{!{\huxvb{1}}l!{\huxvb{0}}}{\huxtpad{2pt}\raggedright {\fontfamily{cmtt}\selectfont }\huxbpad{4pt}} &
\multicolumn{1}{l!{\huxvb{0}}}{\cellcolor[RGB]{230, 230, 230}\huxtpad{2pt}\raggedright {\fontfamily{cmtt}\selectfont top\_border\_style}\huxbpad{4pt}} &
\multicolumn{1}{l!{\huxvb{0}}}{\huxtpad{2pt}\raggedright {\fontfamily{cmtt}\selectfont }\huxbpad{4pt}} &
\multicolumn{1}{l!{\huxvb{0}}}{\cellcolor[RGB]{230, 230, 230}\huxtpad{2pt}\raggedright {\fontfamily{cmtt}\selectfont }\huxbpad{4pt}} &
\multicolumn{1}{l!{\huxvb{1}}}{\huxtpad{2pt}\raggedright {\fontfamily{cmtt}\selectfont }\huxbpad{4pt}} \tabularnewline[-0.5pt]


\hhline{>{\huxb{1}}|>{\huxb{1}}|}
\arrayrulecolor{black}

\multicolumn{1}{!{\huxvb{1}}l!{\huxvb{0}}}{\huxtpad{2pt}\raggedright {\fontfamily{cmtt}\selectfont }\huxbpad{4pt}} &
\multicolumn{1}{l!{\huxvb{0}}}{\cellcolor[RGB]{230, 230, 230}\huxtpad{2pt}\raggedright {\fontfamily{cmtt}\selectfont top\_padding}\huxbpad{4pt}} &
\multicolumn{1}{l!{\huxvb{0}}}{\huxtpad{2pt}\raggedright {\fontfamily{cmtt}\selectfont }\huxbpad{4pt}} &
\multicolumn{1}{l!{\huxvb{0}}}{\cellcolor[RGB]{230, 230, 230}\huxtpad{2pt}\raggedright {\fontfamily{cmtt}\selectfont }\huxbpad{4pt}} &
\multicolumn{1}{l!{\huxvb{1}}}{\huxtpad{2pt}\raggedright {\fontfamily{cmtt}\selectfont }\huxbpad{4pt}} \tabularnewline[-0.5pt]


\hhline{>{\huxb{1}}|>{\huxb{1}}|}
\arrayrulecolor{black}

\multicolumn{1}{!{\huxvb{1}}l!{\huxvb{0}}}{\huxtpad{2pt}\raggedright {\fontfamily{cmtt}\selectfont }\huxbpad{4pt}} &
\multicolumn{1}{l!{\huxvb{0}}}{\cellcolor[RGB]{230, 230, 230}\huxtpad{2pt}\raggedright {\fontfamily{cmtt}\selectfont valign}\huxbpad{4pt}} &
\multicolumn{1}{l!{\huxvb{0}}}{\huxtpad{2pt}\raggedright {\fontfamily{cmtt}\selectfont }\huxbpad{4pt}} &
\multicolumn{1}{l!{\huxvb{0}}}{\cellcolor[RGB]{230, 230, 230}\huxtpad{2pt}\raggedright {\fontfamily{cmtt}\selectfont }\huxbpad{4pt}} &
\multicolumn{1}{l!{\huxvb{1}}}{\huxtpad{2pt}\raggedright {\fontfamily{cmtt}\selectfont }\huxbpad{4pt}} \tabularnewline[-0.5pt]


\hhline{>{\huxb{1}}->{\huxb{1}}->{\huxb{1}}->{\huxb{1}}->{\huxb{1}}-}
\arrayrulecolor{black}
\end{tabularx}\par\end{raggedright}
\end{table}
 

\FloatBarrier

\hypertarget{tidyverse-syntax}{%
\subsection{Tidyverse syntax}\label{tidyverse-syntax}}

If you prefer a tidyverse style of code, using the pipe operator
\texttt{\%\textgreater{}\%}, then you can use \texttt{set\_*} functions
to set properties. These have the same name as the property, with
\texttt{set\_} prepended. For example, to set the \texttt{bold}
property, you use the \texttt{set\_bold} function.

\texttt{set\_*} functions return the modified huxtable, so you can chain
them together like this:

\begin{Shaded}
\begin{Highlighting}[]
\KeywordTok{library}\NormalTok{(dplyr)}
\KeywordTok{hux}\NormalTok{(}
        \DataTypeTok{Employee     =} \KeywordTok{c}\NormalTok{(}\StringTok{'John Smith'}\NormalTok{, }\StringTok{'Jane Doe'}\NormalTok{, }\StringTok{'David Hugh-Jones'}\NormalTok{), }
        \DataTypeTok{Salary       =} \KeywordTok{c}\NormalTok{(}\DecValTok{50000}\NormalTok{, }\DecValTok{50000}\NormalTok{, }\DecValTok{40000}\NormalTok{),}
        \DataTypeTok{add_colnames =} \OtherTok{TRUE}
\NormalTok{      )                               }\OperatorTok
\StringTok{      }\KeywordTok{set_right_padding}\NormalTok{(}\DecValTok{10}\NormalTok{)           }\OperatorTok
\StringTok{      }\KeywordTok{set_left_padding}\NormalTok{(}\DecValTok{10}\NormalTok{)            }\OperatorTok\StringTok{ }
\StringTok{      }\KeywordTok{set_bold}\NormalTok{(}\DecValTok{1}\NormalTok{, }\DecValTok{1}\OperatorTok{:}\DecValTok{2}\NormalTok{, }\OtherTok{TRUE}\NormalTok{)          }\OperatorTok\StringTok{ }
\StringTok{      }\KeywordTok{set_bottom_border}\NormalTok{(}\DecValTok{1}\NormalTok{, }\DecValTok{1}\OperatorTok{:}\DecValTok{2}\NormalTok{, }\DecValTok{1}\NormalTok{)    }\OperatorTok
\StringTok{      }\KeywordTok{set_align}\NormalTok{(}\DecValTok{1}\OperatorTok{:}\DecValTok{4}\NormalTok{, }\DecValTok{2}\NormalTok{, }\StringTok{'right'}\NormalTok{)      }\OperatorTok
\StringTok{      }\KeywordTok{set_number_format}\NormalTok{(}\DecValTok{2}\NormalTok{)            }\OperatorTok\StringTok{ }
\StringTok{      }\KeywordTok{set_caption}\NormalTok{(}\StringTok{'Employee table'}\NormalTok{)}
\end{Highlighting}
\end{Shaded}

 \begin{table}[h]
\centering\captionsetup{justification=centering,singlelinecheck=off}
\caption{Employee table}

    \providecommand{\huxb}[2][0,0,0]{\arrayrulecolor[RGB]{#1}\global\arrayrulewidth=#2pt}
    \providecommand{\huxvb}[2][0,0,0]{\color[RGB]{#1}\vrule width #2pt}
    \providecommand{\huxtpad}[1]{\rule{0pt}{\baselineskip+#1}}
    \providecommand{\huxbpad}[1]{\rule[-#1]{0pt}{#1}}
  \begin{tabularx}{0.5\textwidth}{p{0.25\textwidth} p{0.25\textwidth}}


\hhline{}
\arrayrulecolor{black}

\multicolumn{1}{!{\huxvb{0}}l!{\huxvb{0}}}{\huxtpad{4pt}\raggedright \textbf{Employee}\huxbpad{4pt}} &
\multicolumn{1}{r!{\huxvb{0}}}{\huxtpad{4pt}\raggedleft \textbf{Salary}\huxbpad{4pt}} \tabularnewline[-0.5pt]


\hhline{>{\huxb{1}}->{\huxb{1}}-}
\arrayrulecolor{black}

\multicolumn{1}{!{\huxvb{0}}l!{\huxvb{0}}}{\huxtpad{4pt}\raggedright John Smith\huxbpad{4pt}} &
\multicolumn{1}{r!{\huxvb{0}}}{\huxtpad{4pt}\raggedleft 50000.00\huxbpad{4pt}} \tabularnewline[-0.5pt]


\hhline{}
\arrayrulecolor{black}

\multicolumn{1}{!{\huxvb{0}}l!{\huxvb{0}}}{\huxtpad{4pt}\raggedright Jane Doe\huxbpad{4pt}} &
\multicolumn{1}{r!{\huxvb{0}}}{\huxtpad{4pt}\raggedleft 50000.00\huxbpad{4pt}} \tabularnewline[-0.5pt]


\hhline{}
\arrayrulecolor{black}

\multicolumn{1}{!{\huxvb{0}}l!{\huxvb{0}}}{\huxtpad{4pt}\raggedright David Hugh-Jones\huxbpad{4pt}} &
\multicolumn{1}{r!{\huxvb{0}}}{\huxtpad{4pt}\raggedleft 40000.00\huxbpad{4pt}} \tabularnewline[-0.5pt]


\hhline{}
\arrayrulecolor{black}
\end{tabularx}
\end{table}
 

\FloatBarrier

\texttt{set\_*} functions for cell properties are called like this:
\texttt{set\_xxx(ht,\ row,\ col,\ value)} or like this:
\texttt{set\_xxx(ht,\ value)}. If you use the second form, then the
value is set for all cells. \texttt{set\_*} functions for table
properties are always called like \texttt{set\_xxx(ht,\ value)}. We'll
learn more about this interface in a moment.

There are also some useful convenience functions:

\begin{itemize}
\tightlist
\item
  \texttt{set\_all\_borders} sets left, right, top and bottom borders
  for selected cells;
\item
  \texttt{set\_all\_border\_colors} sets left, right, top and bottom
  border colors;
\item
  \texttt{set\_all\_border\_styles} sets left, right, top and bottom
  border styles;
\item
  \texttt{set\_all\_padding} sets left, right, top and bottom padding
  (the amount of space between the content and the border);
\item
  \texttt{set\_outer\_borders} sets an outer border around a rectangle
  of cells.
\end{itemize}

\hypertarget{getting-properties}{%
\subsection{Getting properties}\label{getting-properties}}

To get the current properties of a huxtable, just use the properties
function without the left arrow:

\begin{Shaded}
\begin{Highlighting}[]
\KeywordTok{italic}\NormalTok{(ht)}
\end{Highlighting}
\end{Shaded}

\begin{verbatim}
##   Employee Salary
## 1    FALSE  FALSE
## 2    FALSE  FALSE
## 3    FALSE  FALSE
## 4    FALSE  FALSE
\end{verbatim}

\begin{Shaded}
\begin{Highlighting}[]
\KeywordTok{position}\NormalTok{(ht)}
\end{Highlighting}
\end{Shaded}

\begin{verbatim}
## [1] "center"
\end{verbatim}

\FloatBarrier

As before, you can use subsetting to get particular rows or columns:

\begin{Shaded}
\begin{Highlighting}[]
\KeywordTok{bottom_border}\NormalTok{(ht)[}\DecValTok{1}\OperatorTok{:}\DecValTok{2}\NormalTok{,]}
\end{Highlighting}
\end{Shaded}

\begin{verbatim}
##   Employee Salary
## 1        1      1
## 2        0      0
\end{verbatim}

\FloatBarrier

\hypertarget{editing-content}{%
\section{Editing content}\label{editing-content}}

\hypertarget{changing-text-in-a-huxtable}{%
\subsection{Changing text in a
huxtable}\label{changing-text-in-a-huxtable}}

You can treat a huxtable just like a data frame. If you want to change
data in a cell, assign to that cell:

\begin{Shaded}
\begin{Highlighting}[]
\NormalTok{ht[}\DecValTok{3}\NormalTok{, }\DecValTok{1}\NormalTok{] <-}\StringTok{ 'Jane Jones'}
\NormalTok{ht}
\end{Highlighting}
\end{Shaded}

 \begin{table}[h]
\centering\captionsetup{justification=centering,singlelinecheck=off}
\caption{Employee table}

    \providecommand{\huxb}[2][0,0,0]{\arrayrulecolor[RGB]{#1}\global\arrayrulewidth=#2pt}
    \providecommand{\huxvb}[2][0,0,0]{\color[RGB]{#1}\vrule width #2pt}
    \providecommand{\huxtpad}[1]{\rule{0pt}{\baselineskip+#1}}
    \providecommand{\huxbpad}[1]{\rule[-#1]{0pt}{#1}}
  \begin{tabularx}{0.5\textwidth}{p{0.25\textwidth} p{0.25\textwidth}}


\hhline{}
\arrayrulecolor{black}

\multicolumn{1}{!{\huxvb{0}}l!{\huxvb{0}}}{\huxtpad{4pt}\raggedright \textbf{Employee}\huxbpad{4pt}} &
\multicolumn{1}{r!{\huxvb{0}}}{\huxtpad{4pt}\raggedleft \textbf{Salary}\huxbpad{4pt}} \tabularnewline[-0.5pt]


\hhline{>{\huxb{1}}->{\huxb{1}}-}
\arrayrulecolor{black}

\multicolumn{1}{!{\huxvb{0}}l!{\huxvb{0}}}{\huxtpad{4pt}\raggedright John Smith\huxbpad{4pt}} &
\multicolumn{1}{r!{\huxvb{0}}}{\huxtpad{4pt}\raggedleft 50000.00\huxbpad{4pt}} \tabularnewline[-0.5pt]


\hhline{}
\arrayrulecolor{black}

\multicolumn{1}{!{\huxvb{0}}l!{\huxvb{0}}}{\huxtpad{4pt}\raggedright Jane Jones\huxbpad{4pt}} &
\multicolumn{1}{r!{\huxvb{0}}}{\huxtpad{4pt}\raggedleft 50000.00\huxbpad{4pt}} \tabularnewline[-0.5pt]


\hhline{}
\arrayrulecolor{black}

\multicolumn{1}{!{\huxvb{0}}l!{\huxvb{0}}}{\huxtpad{4pt}\raggedright David Hugh-Jones\huxbpad{4pt}} &
\multicolumn{1}{r!{\huxvb{0}}}{\huxtpad{4pt}\raggedleft 40000.00\huxbpad{4pt}} \tabularnewline[-0.5pt]


\hhline{}
\arrayrulecolor{black}
\end{tabularx}
\end{table}
 

\FloatBarrier

To add a column, do, e.g.:

\begin{Shaded}
\begin{Highlighting}[]
\NormalTok{ht_with_roles <-}\StringTok{ }\NormalTok{ht}
\NormalTok{ht_with_roles}\OperatorTok{$}\NormalTok{Role <-}\StringTok{ }\KeywordTok{c}\NormalTok{(}\StringTok{"Role"}\NormalTok{, }\StringTok{"Admin"}\NormalTok{, }\StringTok{"CEO"}\NormalTok{, }\StringTok{"Dogsbody"}\NormalTok{)}
\NormalTok{ht_with_roles}
\end{Highlighting}
\end{Shaded}

 \begin{table}[h]
\centering\captionsetup{justification=centering,singlelinecheck=off}
\caption{Employee table}

    \providecommand{\huxb}[2][0,0,0]{\arrayrulecolor[RGB]{#1}\global\arrayrulewidth=#2pt}
    \providecommand{\huxvb}[2][0,0,0]{\color[RGB]{#1}\vrule width #2pt}
    \providecommand{\huxtpad}[1]{\rule{0pt}{\baselineskip+#1}}
    \providecommand{\huxbpad}[1]{\rule[-#1]{0pt}{#1}}
  \begin{tabularx}{0.5\textwidth}{p{0.166666666666667\textwidth} p{0.166666666666667\textwidth} p{0.166666666666667\textwidth}}


\hhline{}
\arrayrulecolor{black}

\multicolumn{1}{!{\huxvb{0}}l!{\huxvb{0}}}{\huxtpad{4pt}\raggedright \textbf{Employee}\huxbpad{4pt}} &
\multicolumn{1}{r!{\huxvb{0}}}{\huxtpad{4pt}\raggedleft \textbf{Salary}\huxbpad{4pt}} &
\multicolumn{1}{l!{\huxvb{0}}}{\huxtpad{4pt}\raggedright Role\huxbpad{4pt}} \tabularnewline[-0.5pt]


\hhline{>{\huxb{1}}->{\huxb{1}}->{\huxb[255, 255, 255]{1}}-}
\arrayrulecolor{black}

\multicolumn{1}{!{\huxvb{0}}l!{\huxvb{0}}}{\huxtpad{4pt}\raggedright John Smith\huxbpad{4pt}} &
\multicolumn{1}{r!{\huxvb{0}}}{\huxtpad{4pt}\raggedleft 50000.00\huxbpad{4pt}} &
\multicolumn{1}{l!{\huxvb{0}}}{\huxtpad{4pt}\raggedright Admin\huxbpad{4pt}} \tabularnewline[-0.5pt]


\hhline{}
\arrayrulecolor{black}

\multicolumn{1}{!{\huxvb{0}}l!{\huxvb{0}}}{\huxtpad{4pt}\raggedright Jane Jones\huxbpad{4pt}} &
\multicolumn{1}{r!{\huxvb{0}}}{\huxtpad{4pt}\raggedleft 50000.00\huxbpad{4pt}} &
\multicolumn{1}{l!{\huxvb{0}}}{\huxtpad{4pt}\raggedright CEO\huxbpad{4pt}} \tabularnewline[-0.5pt]


\hhline{}
\arrayrulecolor{black}

\multicolumn{1}{!{\huxvb{0}}l!{\huxvb{0}}}{\huxtpad{4pt}\raggedright David Hugh-Jones\huxbpad{4pt}} &
\multicolumn{1}{r!{\huxvb{0}}}{\huxtpad{4pt}\raggedleft 40000.00\huxbpad{4pt}} &
\multicolumn{1}{l!{\huxvb{0}}}{\huxtpad{4pt}\raggedright Dogsbody\huxbpad{4pt}} \tabularnewline[-0.5pt]


\hhline{}
\arrayrulecolor{black}
\end{tabularx}
\end{table}
 

\FloatBarrier

There are two things to notice here:

\begin{itemize}
\tightlist
\item
  When we added the column, we included the column name explicitly in
  the data so it would be printed out.
\item
  The third column doesn't have the properties we set on the first two
  columns, like the bold first row and the underlining.
\end{itemize}

If we want new columns to have the properties from their neighbours, we
can use \texttt{cbind}, a base R function that binds columns together.
When you \texttt{cbind} huxtable objects, by default, cell properties
are copied over from their neighbours:

\begin{Shaded}
\begin{Highlighting}[]
\NormalTok{ht_with_roles <-}\StringTok{ }\KeywordTok{cbind}\NormalTok{(ht, }\KeywordTok{c}\NormalTok{(}\StringTok{"Role"}\NormalTok{, }\StringTok{"Admin"}\NormalTok{, }\StringTok{"CEO"}\NormalTok{, }\StringTok{"Dogsbody"}\NormalTok{))}
\NormalTok{ht_with_roles}
\end{Highlighting}
\end{Shaded}

 \begin{table}[h]
\centering\captionsetup{justification=centering,singlelinecheck=off}
\caption{Employee table}

    \providecommand{\huxb}[2][0,0,0]{\arrayrulecolor[RGB]{#1}\global\arrayrulewidth=#2pt}
    \providecommand{\huxvb}[2][0,0,0]{\color[RGB]{#1}\vrule width #2pt}
    \providecommand{\huxtpad}[1]{\rule{0pt}{\baselineskip+#1}}
    \providecommand{\huxbpad}[1]{\rule[-#1]{0pt}{#1}}
  \begin{tabularx}{0.5\textwidth}{p{0.166666666666667\textwidth} p{0.166666666666667\textwidth} p{0.166666666666667\textwidth}}


\hhline{}
\arrayrulecolor{black}

\multicolumn{1}{!{\huxvb{0}}l!{\huxvb{0}}}{\huxtpad{4pt}\raggedright \textbf{Employee}\huxbpad{4pt}} &
\multicolumn{1}{r!{\huxvb{0}}}{\huxtpad{4pt}\raggedleft \textbf{Salary}\huxbpad{4pt}} &
\multicolumn{1}{r!{\huxvb{0}}}{\huxtpad{4pt}\raggedleft \textbf{Role}\huxbpad{4pt}} \tabularnewline[-0.5pt]


\hhline{>{\huxb{1}}->{\huxb{1}}->{\huxb{1}}-}
\arrayrulecolor{black}

\multicolumn{1}{!{\huxvb{0}}l!{\huxvb{0}}}{\huxtpad{4pt}\raggedright John Smith\huxbpad{4pt}} &
\multicolumn{1}{r!{\huxvb{0}}}{\huxtpad{4pt}\raggedleft 50000.00\huxbpad{4pt}} &
\multicolumn{1}{r!{\huxvb{0}}}{\huxtpad{4pt}\raggedleft Admin\huxbpad{4pt}} \tabularnewline[-0.5pt]


\hhline{}
\arrayrulecolor{black}

\multicolumn{1}{!{\huxvb{0}}l!{\huxvb{0}}}{\huxtpad{4pt}\raggedright Jane Jones\huxbpad{4pt}} &
\multicolumn{1}{r!{\huxvb{0}}}{\huxtpad{4pt}\raggedleft 50000.00\huxbpad{4pt}} &
\multicolumn{1}{r!{\huxvb{0}}}{\huxtpad{4pt}\raggedleft CEO\huxbpad{4pt}} \tabularnewline[-0.5pt]


\hhline{}
\arrayrulecolor{black}

\multicolumn{1}{!{\huxvb{0}}l!{\huxvb{0}}}{\huxtpad{4pt}\raggedright David Hugh-Jones\huxbpad{4pt}} &
\multicolumn{1}{r!{\huxvb{0}}}{\huxtpad{4pt}\raggedleft 40000.00\huxbpad{4pt}} &
\multicolumn{1}{r!{\huxvb{0}}}{\huxtpad{4pt}\raggedleft Dogsbody\huxbpad{4pt}} \tabularnewline[-0.5pt]


\hhline{}
\arrayrulecolor{black}
\end{tabularx}
\end{table}
 

\FloatBarrier

\texttt{rbind} works the same way:

\begin{Shaded}
\begin{Highlighting}[]
\KeywordTok{rbind}\NormalTok{(ht, }\KeywordTok{c}\NormalTok{(}\StringTok{"Yihui Xie"}\NormalTok{, }\DecValTok{100000}\NormalTok{))}
\end{Highlighting}
\end{Shaded}

 \begin{table}[h]
\centering\captionsetup{justification=centering,singlelinecheck=off}
\caption{Employee table}

    \providecommand{\huxb}[2][0,0,0]{\arrayrulecolor[RGB]{#1}\global\arrayrulewidth=#2pt}
    \providecommand{\huxvb}[2][0,0,0]{\color[RGB]{#1}\vrule width #2pt}
    \providecommand{\huxtpad}[1]{\rule{0pt}{\baselineskip+#1}}
    \providecommand{\huxbpad}[1]{\rule[-#1]{0pt}{#1}}
  \begin{tabularx}{0.5\textwidth}{p{0.25\textwidth} p{0.25\textwidth}}


\hhline{}
\arrayrulecolor{black}

\multicolumn{1}{!{\huxvb{0}}l!{\huxvb{0}}}{\huxtpad{4pt}\raggedright \textbf{Employee}\huxbpad{4pt}} &
\multicolumn{1}{r!{\huxvb{0}}}{\huxtpad{4pt}\raggedleft \textbf{Salary}\huxbpad{4pt}} \tabularnewline[-0.5pt]


\hhline{>{\huxb{1}}->{\huxb{1}}-}
\arrayrulecolor{black}

\multicolumn{1}{!{\huxvb{0}}l!{\huxvb{0}}}{\huxtpad{4pt}\raggedright John Smith\huxbpad{4pt}} &
\multicolumn{1}{r!{\huxvb{0}}}{\huxtpad{4pt}\raggedleft 50000.00\huxbpad{4pt}} \tabularnewline[-0.5pt]


\hhline{}
\arrayrulecolor{black}

\multicolumn{1}{!{\huxvb{0}}l!{\huxvb{0}}}{\huxtpad{4pt}\raggedright Jane Jones\huxbpad{4pt}} &
\multicolumn{1}{r!{\huxvb{0}}}{\huxtpad{4pt}\raggedleft 50000.00\huxbpad{4pt}} \tabularnewline[-0.5pt]


\hhline{}
\arrayrulecolor{black}

\multicolumn{1}{!{\huxvb{0}}l!{\huxvb{0}}}{\huxtpad{4pt}\raggedright David Hugh-Jones\huxbpad{4pt}} &
\multicolumn{1}{r!{\huxvb{0}}}{\huxtpad{4pt}\raggedleft 40000.00\huxbpad{4pt}} \tabularnewline[-0.5pt]


\hhline{}
\arrayrulecolor{black}

\multicolumn{1}{!{\huxvb{0}}l!{\huxvb{0}}}{\huxtpad{4pt}\raggedright Yihui Xie\huxbpad{4pt}} &
\multicolumn{1}{r!{\huxvb{0}}}{\huxtpad{4pt}\raggedleft 100000.00\huxbpad{4pt}} \tabularnewline[-0.5pt]


\hhline{}
\arrayrulecolor{black}
\end{tabularx}
\end{table}
 

\FloatBarrier

Notice how Yihui's salary has got the same number formatting as the
other employees. That's because cell properties for the new row were
copied from the row above.

If you want to avoid this behaviour, use
\texttt{copy\_cell\_props\ =\ FALSE}:

\begin{Shaded}
\begin{Highlighting}[]
\KeywordTok{rbind}\NormalTok{(ht, }\KeywordTok{c}\NormalTok{(}\StringTok{"Yihui Xie"}\NormalTok{, }\DecValTok{100000}\NormalTok{), }\DataTypeTok{copy_cell_props =} \OtherTok{FALSE}\NormalTok{)}
\end{Highlighting}
\end{Shaded}

 \begin{table}[h]
\centering\captionsetup{justification=centering,singlelinecheck=off}
\caption{Employee table}

    \providecommand{\huxb}[2][0,0,0]{\arrayrulecolor[RGB]{#1}\global\arrayrulewidth=#2pt}
    \providecommand{\huxvb}[2][0,0,0]{\color[RGB]{#1}\vrule width #2pt}
    \providecommand{\huxtpad}[1]{\rule{0pt}{\baselineskip+#1}}
    \providecommand{\huxbpad}[1]{\rule[-#1]{0pt}{#1}}
  \begin{tabularx}{0.5\textwidth}{p{0.25\textwidth} p{0.25\textwidth}}


\hhline{}
\arrayrulecolor{black}

\multicolumn{1}{!{\huxvb{0}}l!{\huxvb{0}}}{\huxtpad{4pt}\raggedright \textbf{Employee}\huxbpad{4pt}} &
\multicolumn{1}{r!{\huxvb{0}}}{\huxtpad{4pt}\raggedleft \textbf{Salary}\huxbpad{4pt}} \tabularnewline[-0.5pt]


\hhline{>{\huxb{1}}->{\huxb{1}}-}
\arrayrulecolor{black}

\multicolumn{1}{!{\huxvb{0}}l!{\huxvb{0}}}{\huxtpad{4pt}\raggedright John Smith\huxbpad{4pt}} &
\multicolumn{1}{r!{\huxvb{0}}}{\huxtpad{4pt}\raggedleft 50000.00\huxbpad{4pt}} \tabularnewline[-0.5pt]


\hhline{}
\arrayrulecolor{black}

\multicolumn{1}{!{\huxvb{0}}l!{\huxvb{0}}}{\huxtpad{4pt}\raggedright Jane Jones\huxbpad{4pt}} &
\multicolumn{1}{r!{\huxvb{0}}}{\huxtpad{4pt}\raggedleft 50000.00\huxbpad{4pt}} \tabularnewline[-0.5pt]


\hhline{}
\arrayrulecolor{black}

\multicolumn{1}{!{\huxvb{0}}l!{\huxvb{0}}}{\huxtpad{4pt}\raggedright David Hugh-Jones\huxbpad{4pt}} &
\multicolumn{1}{r!{\huxvb{0}}}{\huxtpad{4pt}\raggedleft 40000.00\huxbpad{4pt}} \tabularnewline[-0.5pt]


\hhline{}
\arrayrulecolor{black}

\multicolumn{1}{!{\huxvb{0}}l!{\huxvb{0}}}{\huxtpad{4pt}\raggedright Yihui Xie\huxbpad{4pt}} &
\multicolumn{1}{l!{\huxvb{0}}}{\huxtpad{4pt}\raggedright 1e+05\huxbpad{4pt}} \tabularnewline[-0.5pt]


\hhline{}
\arrayrulecolor{black}
\end{tabularx}
\end{table}
 

\FloatBarrier

\hypertarget{data-manipulation-the-base-r-way}{%
\subsection{Data manipulation the base R
way}\label{data-manipulation-the-base-r-way}}

You can subset, sort and generally data-wrangle a huxtable just like a
normal data frame. Cell and table properties will be carried over into
subsets.

\begin{Shaded}
\begin{Highlighting}[]
\CommentTok{# Select columns by name:}
\NormalTok{cars_mpg <-}\StringTok{ }\NormalTok{car_ht[, }\KeywordTok{c}\NormalTok{(}\StringTok{"mpg"}\NormalTok{, }\StringTok{"cyl"}\NormalTok{, }\StringTok{"am"}\NormalTok{)] }
\CommentTok{# Order by number of cylinders:}
\NormalTok{cars_mpg <-}\StringTok{ }\NormalTok{cars_mpg[}\KeywordTok{order}\NormalTok{(cars_mpg}\OperatorTok{$}\NormalTok{cyl),]}

\NormalTok{cars_mpg <-}\StringTok{ }\NormalTok{cars_mpg                          }\OperatorTok\StringTok{ }
\StringTok{      }\NormalTok{huxtable}\OperatorTok{::}\KeywordTok{add_rownames}\NormalTok{(}\DataTypeTok{colname =} \StringTok{"Car"}\NormalTok{) }\OperatorTok\StringTok{ }
\StringTok{      }\NormalTok{huxtable}\OperatorTok{::}\KeywordTok{add_colnames}\NormalTok{()                }\OperatorTok
\StringTok{      }\KeywordTok{set_right_border}\NormalTok{(}\FloatTok{0.4}\NormalTok{)                   }\OperatorTok\StringTok{ }
\StringTok{      }\KeywordTok{set_right_border_color}\NormalTok{(}\StringTok{"grey"}\NormalTok{)}

\CommentTok{# Show the first 5 rows:}
\NormalTok{cars_mpg[}\DecValTok{1}\OperatorTok{:}\DecValTok{5}\NormalTok{,]}
\end{Highlighting}
\end{Shaded}

 \begin{table}[h]
\centering
    \providecommand{\huxb}[2][0,0,0]{\arrayrulecolor[RGB]{#1}\global\arrayrulewidth=#2pt}
    \providecommand{\huxvb}[2][0,0,0]{\color[RGB]{#1}\vrule width #2pt}
    \providecommand{\huxtpad}[1]{\rule{0pt}{\baselineskip+#1}}
    \providecommand{\huxbpad}[1]{\rule[-#1]{0pt}{#1}}
  \begin{tabularx}{0.5\textwidth}{p{0.125\textwidth} p{0.125\textwidth} p{0.125\textwidth} p{0.125\textwidth}}


\hhline{>{\huxb[190, 190, 190]{0.4}}|>{\huxb[190, 190, 190]{0.4}}|>{\huxb[190, 190, 190]{0.4}}|>{\huxb[190, 190, 190]{0.4}}|}
\arrayrulecolor{black}

\multicolumn{1}{!{\huxvb{0}}l!{\huxvb[190, 190, 190]{0.4}}}{\huxtpad{4pt}\raggedright Car\huxbpad{4pt}} &
\multicolumn{1}{l!{\huxvb[190, 190, 190]{0.4}}}{\huxtpad{4pt}\raggedright mpg\huxbpad{4pt}} &
\multicolumn{1}{l!{\huxvb[190, 190, 190]{0.4}}}{\huxtpad{4pt}\raggedright cyl\huxbpad{4pt}} &
\multicolumn{1}{l!{\huxvb[190, 190, 190]{0.4}}}{\huxtpad{4pt}\raggedright am\huxbpad{4pt}} \tabularnewline[-0.5pt]


\hhline{>{\huxb[190, 190, 190]{0.4}}|>{\huxb[190, 190, 190]{0.4}}|>{\huxb[190, 190, 190]{0.4}}|>{\huxb[190, 190, 190]{0.4}}|}
\arrayrulecolor{black}

\multicolumn{1}{!{\huxvb{0}}l!{\huxvb[190, 190, 190]{0.4}}}{\huxtpad{4pt}\raggedright Datsun 710\huxbpad{4pt}} &
\multicolumn{1}{r!{\huxvb[190, 190, 190]{0.4}}}{\huxtpad{4pt}\raggedleft 22.8\huxbpad{4pt}} &
\multicolumn{1}{r!{\huxvb[190, 190, 190]{0.4}}}{\huxtpad{4pt}\raggedleft 4\huxbpad{4pt}} &
\multicolumn{1}{r!{\huxvb[190, 190, 190]{0.4}}}{\huxtpad{4pt}\raggedleft 1\huxbpad{4pt}} \tabularnewline[-0.5pt]


\hhline{>{\huxb[190, 190, 190]{0.4}}|>{\huxb[190, 190, 190]{0.4}}|>{\huxb[190, 190, 190]{0.4}}|>{\huxb[190, 190, 190]{0.4}}|}
\arrayrulecolor{black}

\multicolumn{1}{!{\huxvb{0}}l!{\huxvb[190, 190, 190]{0.4}}}{\huxtpad{4pt}\raggedright Merc 240D\huxbpad{4pt}} &
\multicolumn{1}{r!{\huxvb[190, 190, 190]{0.4}}}{\huxtpad{4pt}\raggedleft 24.4\huxbpad{4pt}} &
\multicolumn{1}{r!{\huxvb[190, 190, 190]{0.4}}}{\huxtpad{4pt}\raggedleft 4\huxbpad{4pt}} &
\multicolumn{1}{r!{\huxvb[190, 190, 190]{0.4}}}{\huxtpad{4pt}\raggedleft 0\huxbpad{4pt}} \tabularnewline[-0.5pt]


\hhline{>{\huxb[190, 190, 190]{0.4}}|>{\huxb[190, 190, 190]{0.4}}|>{\huxb[190, 190, 190]{0.4}}|>{\huxb[190, 190, 190]{0.4}}|}
\arrayrulecolor{black}

\multicolumn{1}{!{\huxvb{0}}l!{\huxvb[190, 190, 190]{0.4}}}{\huxtpad{4pt}\raggedright Merc 230\huxbpad{4pt}} &
\multicolumn{1}{r!{\huxvb[190, 190, 190]{0.4}}}{\huxtpad{4pt}\raggedleft 22.8\huxbpad{4pt}} &
\multicolumn{1}{r!{\huxvb[190, 190, 190]{0.4}}}{\huxtpad{4pt}\raggedleft 4\huxbpad{4pt}} &
\multicolumn{1}{r!{\huxvb[190, 190, 190]{0.4}}}{\huxtpad{4pt}\raggedleft 0\huxbpad{4pt}} \tabularnewline[-0.5pt]


\hhline{>{\huxb[190, 190, 190]{0.4}}|>{\huxb[190, 190, 190]{0.4}}|>{\huxb[190, 190, 190]{0.4}}|>{\huxb[190, 190, 190]{0.4}}|}
\arrayrulecolor{black}

\multicolumn{1}{!{\huxvb{0}}l!{\huxvb[190, 190, 190]{0.4}}}{\huxtpad{4pt}\raggedright Fiat 128\huxbpad{4pt}} &
\multicolumn{1}{r!{\huxvb[190, 190, 190]{0.4}}}{\huxtpad{4pt}\raggedleft 32.4\huxbpad{4pt}} &
\multicolumn{1}{r!{\huxvb[190, 190, 190]{0.4}}}{\huxtpad{4pt}\raggedleft 4\huxbpad{4pt}} &
\multicolumn{1}{r!{\huxvb[190, 190, 190]{0.4}}}{\huxtpad{4pt}\raggedleft 1\huxbpad{4pt}} \tabularnewline[-0.5pt]


\hhline{>{\huxb[190, 190, 190]{0.4}}|>{\huxb[190, 190, 190]{0.4}}|>{\huxb[190, 190, 190]{0.4}}|>{\huxb[190, 190, 190]{0.4}}|}
\arrayrulecolor{black}
\end{tabularx}
\end{table}
 

\FloatBarrier

\hypertarget{data-manipulation-the-dplyr-way}{%
\subsection{Data manipulation the dplyr
way}\label{data-manipulation-the-dplyr-way}}

You can also use \texttt{dplyr} functions to edit a huxtable:

\begin{Shaded}
\begin{Highlighting}[]
\NormalTok{car_ht <-}\StringTok{ }\NormalTok{car_ht                                          }\OperatorTok
\StringTok{      }\NormalTok{huxtable}\OperatorTok{::}\KeywordTok{add_rownames}\NormalTok{(}\DataTypeTok{colname =} \StringTok{"Car"}\NormalTok{)             }\OperatorTok
\StringTok{      }\KeywordTok{slice}\NormalTok{(}\DecValTok{1}\OperatorTok{:}\DecValTok{10}\NormalTok{)                                         }\OperatorTok\StringTok{ }
\StringTok{      }\KeywordTok{select}\NormalTok{(Car, mpg, cyl, hp)                           }\OperatorTok\StringTok{ }
\StringTok{      }\KeywordTok{arrange}\NormalTok{(hp)                                         }\OperatorTok\StringTok{ }
\StringTok{      }\KeywordTok{filter}\NormalTok{(cyl }\OperatorTok{>}\StringTok{ }\DecValTok{4}\NormalTok{)                                     }\OperatorTok\StringTok{ }
\StringTok{      }\KeywordTok{rename}\NormalTok{(}\DataTypeTok{MPG =}\NormalTok{ mpg, }\DataTypeTok{Cylinders =}\NormalTok{ cyl, }\DataTypeTok{Horsepower =}\NormalTok{ hp) }\OperatorTok\StringTok{ }
\StringTok{      }\KeywordTok{mutate}\NormalTok{(}\DataTypeTok{kml =}\NormalTok{ MPG}\OperatorTok{/}\FloatTok{2.82}\NormalTok{)                               }


\NormalTok{car_ht <-}\StringTok{ }\NormalTok{car_ht                               }\OperatorTok\StringTok{ }
\StringTok{      }\KeywordTok{set_number_format}\NormalTok{(}\DecValTok{1}\OperatorTok{:}\DecValTok{7}\NormalTok{, }\StringTok{"kml"}\NormalTok{, }\DecValTok{2}\NormalTok{)         }\OperatorTok\StringTok{ }
\StringTok{      }\KeywordTok{set_col_width}\NormalTok{(}\KeywordTok{c}\NormalTok{(.}\DecValTok{35}\NormalTok{, }\FloatTok{.15}\NormalTok{, }\FloatTok{.15}\NormalTok{, }\FloatTok{.15}\NormalTok{, }\FloatTok{.2}\NormalTok{)) }\OperatorTok\StringTok{ }
\StringTok{      }\KeywordTok{set_width}\NormalTok{(.}\DecValTok{6}\NormalTok{)                            }\OperatorTok\StringTok{ }
\StringTok{      }\NormalTok{huxtable}\OperatorTok{::}\KeywordTok{add_colnames}\NormalTok{()                 }\OperatorTok\StringTok{ }
\StringTok{      }\KeywordTok{set_right_border}\NormalTok{(}\FloatTok{0.4}\NormalTok{)                    }\OperatorTok\StringTok{ }
\StringTok{      }\KeywordTok{set_right_border_color}\NormalTok{(}\StringTok{"grey"}\NormalTok{)}

\NormalTok{car_ht}
\end{Highlighting}
\end{Shaded}

 \begin{table}[h]
\centering
    \providecommand{\huxb}[2][0,0,0]{\arrayrulecolor[RGB]{#1}\global\arrayrulewidth=#2pt}
    \providecommand{\huxvb}[2][0,0,0]{\color[RGB]{#1}\vrule width #2pt}
    \providecommand{\huxtpad}[1]{\rule{0pt}{\baselineskip+#1}}
    \providecommand{\huxbpad}[1]{\rule[-#1]{0pt}{#1}}
  \begin{tabularx}{0.6\textwidth}{p{0.21\textwidth} p{0.09\textwidth} p{0.09\textwidth} p{0.09\textwidth} p{0.12\textwidth}}


\hhline{>{\huxb[190, 190, 190]{0.4}}|>{\huxb[190, 190, 190]{0.4}}|>{\huxb[190, 190, 190]{0.4}}|>{\huxb[190, 190, 190]{0.4}}|>{\huxb[190, 190, 190]{0.4}}|}
\arrayrulecolor{black}

\multicolumn{1}{!{\huxvb{0}}l!{\huxvb[190, 190, 190]{0.4}}}{\huxtpad{4pt}\raggedright Car\huxbpad{4pt}} &
\multicolumn{1}{l!{\huxvb[190, 190, 190]{0.4}}}{\huxtpad{4pt}\raggedright MPG\huxbpad{4pt}} &
\multicolumn{1}{l!{\huxvb[190, 190, 190]{0.4}}}{\huxtpad{4pt}\raggedright Cylinders\huxbpad{4pt}} &
\multicolumn{1}{l!{\huxvb[190, 190, 190]{0.4}}}{\huxtpad{4pt}\raggedright Horsepower\huxbpad{4pt}} &
\multicolumn{1}{l!{\huxvb[190, 190, 190]{0.4}}}{\huxtpad{4pt}\raggedright kml\huxbpad{4pt}} \tabularnewline[-0.5pt]


\hhline{>{\huxb[190, 190, 190]{0.4}}|>{\huxb[190, 190, 190]{0.4}}|>{\huxb[190, 190, 190]{0.4}}|>{\huxb[190, 190, 190]{0.4}}|>{\huxb[190, 190, 190]{0.4}}|}
\arrayrulecolor{black}

\multicolumn{1}{!{\huxvb{0}}l!{\huxvb[190, 190, 190]{0.4}}}{\huxtpad{4pt}\raggedright Valiant\huxbpad{4pt}} &
\multicolumn{1}{r!{\huxvb[190, 190, 190]{0.4}}}{\huxtpad{4pt}\raggedleft 18.1\huxbpad{4pt}} &
\multicolumn{1}{r!{\huxvb[190, 190, 190]{0.4}}}{\huxtpad{4pt}\raggedleft 6\huxbpad{4pt}} &
\multicolumn{1}{r!{\huxvb[190, 190, 190]{0.4}}}{\huxtpad{4pt}\raggedleft 105\huxbpad{4pt}} &
\multicolumn{1}{r!{\huxvb[190, 190, 190]{0.4}}}{\huxtpad{4pt}\raggedleft 6.42\huxbpad{4pt}} \tabularnewline[-0.5pt]


\hhline{>{\huxb[190, 190, 190]{0.4}}|>{\huxb[190, 190, 190]{0.4}}|>{\huxb[190, 190, 190]{0.4}}|>{\huxb[190, 190, 190]{0.4}}|>{\huxb[190, 190, 190]{0.4}}|}
\arrayrulecolor{black}

\multicolumn{1}{!{\huxvb{0}}l!{\huxvb[190, 190, 190]{0.4}}}{\huxtpad{4pt}\raggedright Mazda RX4\huxbpad{4pt}} &
\multicolumn{1}{r!{\huxvb[190, 190, 190]{0.4}}}{\huxtpad{4pt}\raggedleft 21~~\huxbpad{4pt}} &
\multicolumn{1}{r!{\huxvb[190, 190, 190]{0.4}}}{\huxtpad{4pt}\raggedleft 6\huxbpad{4pt}} &
\multicolumn{1}{r!{\huxvb[190, 190, 190]{0.4}}}{\huxtpad{4pt}\raggedleft 110\huxbpad{4pt}} &
\multicolumn{1}{r!{\huxvb[190, 190, 190]{0.4}}}{\huxtpad{4pt}\raggedleft 7.45\huxbpad{4pt}} \tabularnewline[-0.5pt]


\hhline{>{\huxb[190, 190, 190]{0.4}}|>{\huxb[190, 190, 190]{0.4}}|>{\huxb[190, 190, 190]{0.4}}|>{\huxb[190, 190, 190]{0.4}}|>{\huxb[190, 190, 190]{0.4}}|}
\arrayrulecolor{black}

\multicolumn{1}{!{\huxvb{0}}l!{\huxvb[190, 190, 190]{0.4}}}{\huxtpad{4pt}\raggedright Mazda RX4 Wag\huxbpad{4pt}} &
\multicolumn{1}{r!{\huxvb[190, 190, 190]{0.4}}}{\huxtpad{4pt}\raggedleft 21~~\huxbpad{4pt}} &
\multicolumn{1}{r!{\huxvb[190, 190, 190]{0.4}}}{\huxtpad{4pt}\raggedleft 6\huxbpad{4pt}} &
\multicolumn{1}{r!{\huxvb[190, 190, 190]{0.4}}}{\huxtpad{4pt}\raggedleft 110\huxbpad{4pt}} &
\multicolumn{1}{r!{\huxvb[190, 190, 190]{0.4}}}{\huxtpad{4pt}\raggedleft 7.45\huxbpad{4pt}} \tabularnewline[-0.5pt]


\hhline{>{\huxb[190, 190, 190]{0.4}}|>{\huxb[190, 190, 190]{0.4}}|>{\huxb[190, 190, 190]{0.4}}|>{\huxb[190, 190, 190]{0.4}}|>{\huxb[190, 190, 190]{0.4}}|}
\arrayrulecolor{black}

\multicolumn{1}{!{\huxvb{0}}l!{\huxvb[190, 190, 190]{0.4}}}{\huxtpad{4pt}\raggedright Hornet 4 Drive\huxbpad{4pt}} &
\multicolumn{1}{r!{\huxvb[190, 190, 190]{0.4}}}{\huxtpad{4pt}\raggedleft 21.4\huxbpad{4pt}} &
\multicolumn{1}{r!{\huxvb[190, 190, 190]{0.4}}}{\huxtpad{4pt}\raggedleft 6\huxbpad{4pt}} &
\multicolumn{1}{r!{\huxvb[190, 190, 190]{0.4}}}{\huxtpad{4pt}\raggedleft 110\huxbpad{4pt}} &
\multicolumn{1}{r!{\huxvb[190, 190, 190]{0.4}}}{\huxtpad{4pt}\raggedleft 7.59\huxbpad{4pt}} \tabularnewline[-0.5pt]


\hhline{>{\huxb[190, 190, 190]{0.4}}|>{\huxb[190, 190, 190]{0.4}}|>{\huxb[190, 190, 190]{0.4}}|>{\huxb[190, 190, 190]{0.4}}|>{\huxb[190, 190, 190]{0.4}}|}
\arrayrulecolor{black}

\multicolumn{1}{!{\huxvb{0}}l!{\huxvb[190, 190, 190]{0.4}}}{\huxtpad{4pt}\raggedright Merc 280\huxbpad{4pt}} &
\multicolumn{1}{r!{\huxvb[190, 190, 190]{0.4}}}{\huxtpad{4pt}\raggedleft 19.2\huxbpad{4pt}} &
\multicolumn{1}{r!{\huxvb[190, 190, 190]{0.4}}}{\huxtpad{4pt}\raggedleft 6\huxbpad{4pt}} &
\multicolumn{1}{r!{\huxvb[190, 190, 190]{0.4}}}{\huxtpad{4pt}\raggedleft 123\huxbpad{4pt}} &
\multicolumn{1}{r!{\huxvb[190, 190, 190]{0.4}}}{\huxtpad{4pt}\raggedleft 6.81\huxbpad{4pt}} \tabularnewline[-0.5pt]


\hhline{>{\huxb[190, 190, 190]{0.4}}|>{\huxb[190, 190, 190]{0.4}}|>{\huxb[190, 190, 190]{0.4}}|>{\huxb[190, 190, 190]{0.4}}|>{\huxb[190, 190, 190]{0.4}}|}
\arrayrulecolor{black}

\multicolumn{1}{!{\huxvb{0}}l!{\huxvb[190, 190, 190]{0.4}}}{\huxtpad{4pt}\raggedright Hornet Sportabout\huxbpad{4pt}} &
\multicolumn{1}{r!{\huxvb[190, 190, 190]{0.4}}}{\huxtpad{4pt}\raggedleft 18.7\huxbpad{4pt}} &
\multicolumn{1}{r!{\huxvb[190, 190, 190]{0.4}}}{\huxtpad{4pt}\raggedleft 8\huxbpad{4pt}} &
\multicolumn{1}{r!{\huxvb[190, 190, 190]{0.4}}}{\huxtpad{4pt}\raggedleft 175\huxbpad{4pt}} &
\multicolumn{1}{r!{\huxvb[190, 190, 190]{0.4}}}{\huxtpad{4pt}\raggedleft 6.63\huxbpad{4pt}} \tabularnewline[-0.5pt]


\hhline{>{\huxb[190, 190, 190]{0.4}}|>{\huxb[190, 190, 190]{0.4}}|>{\huxb[190, 190, 190]{0.4}}|>{\huxb[190, 190, 190]{0.4}}|>{\huxb[190, 190, 190]{0.4}}|}
\arrayrulecolor{black}

\multicolumn{1}{!{\huxvb{0}}l!{\huxvb[190, 190, 190]{0.4}}}{\huxtpad{4pt}\raggedright Duster 360\huxbpad{4pt}} &
\multicolumn{1}{r!{\huxvb[190, 190, 190]{0.4}}}{\huxtpad{4pt}\raggedleft 14.3\huxbpad{4pt}} &
\multicolumn{1}{r!{\huxvb[190, 190, 190]{0.4}}}{\huxtpad{4pt}\raggedleft 8\huxbpad{4pt}} &
\multicolumn{1}{r!{\huxvb[190, 190, 190]{0.4}}}{\huxtpad{4pt}\raggedleft 245\huxbpad{4pt}} &
\multicolumn{1}{r!{\huxvb[190, 190, 190]{0.4}}}{\huxtpad{4pt}\raggedleft 5.07\huxbpad{4pt}} \tabularnewline[-0.5pt]


\hhline{>{\huxb[190, 190, 190]{0.4}}|>{\huxb[190, 190, 190]{0.4}}|>{\huxb[190, 190, 190]{0.4}}|>{\huxb[190, 190, 190]{0.4}}|>{\huxb[190, 190, 190]{0.4}}|}
\arrayrulecolor{black}
\end{tabularx}
\end{table}
 

\FloatBarrier

In general it is a good idea to prepare your data first, before styling
it. For example, it was easier to sort the \texttt{cars\_mpg} data by
cylinder, before adding column names to the data frame itself.

\hypertarget{functions-to-insert-rows-columns-and-footnotes}{%
\subsection{Functions to insert rows, columns and
footnotes}\label{functions-to-insert-rows-columns-and-footnotes}}

Huxtable has three convenience functions for adding a row or column to
your table: \texttt{insert\_row}, \texttt{insert\_column} and
\texttt{add\_footnote}. \texttt{insert\_row} and \texttt{insert\_column}
let you add a single row or column. The \texttt{after} parameter
specifies where in the table to do the insertion, i.e.~after what row or
column number. \texttt{add\_footnote} adds a single cell in a new row at
the bottom. The cell spans the whole table row, and has a border above.

\begin{Shaded}
\begin{Highlighting}[]
\NormalTok{ht <-}\StringTok{ }\KeywordTok{insert_row}\NormalTok{(ht, }\StringTok{"Hadley Wickham"}\NormalTok{, }\StringTok{"100000"}\NormalTok{, }\DataTypeTok{after =} \DecValTok{3}\NormalTok{)}
\NormalTok{ht <-}\StringTok{ }\KeywordTok{add_footnote}\NormalTok{(ht, }\StringTok{"DHJ deserves a pay rise"}\NormalTok{)}
\NormalTok{ht}
\end{Highlighting}
\end{Shaded}

 \begin{table}[h]
\centering\captionsetup{justification=centering,singlelinecheck=off}
\caption{Employee table}

    \providecommand{\huxb}[2][0,0,0]{\arrayrulecolor[RGB]{#1}\global\arrayrulewidth=#2pt}
    \providecommand{\huxvb}[2][0,0,0]{\color[RGB]{#1}\vrule width #2pt}
    \providecommand{\huxtpad}[1]{\rule{0pt}{\baselineskip+#1}}
    \providecommand{\huxbpad}[1]{\rule[-#1]{0pt}{#1}}
  \begin{tabularx}{0.5\textwidth}{p{0.25\textwidth} p{0.25\textwidth}}


\hhline{}
\arrayrulecolor{black}

\multicolumn{1}{!{\huxvb{0}}l!{\huxvb{0}}}{\huxtpad{4pt}\raggedright \textbf{Employee}\huxbpad{4pt}} &
\multicolumn{1}{r!{\huxvb{0}}}{\huxtpad{4pt}\raggedleft \textbf{Salary}\huxbpad{4pt}} \tabularnewline[-0.5pt]


\hhline{>{\huxb{1}}->{\huxb{1}}-}
\arrayrulecolor{black}

\multicolumn{1}{!{\huxvb{0}}l!{\huxvb{0}}}{\huxtpad{4pt}\raggedright John Smith\huxbpad{4pt}} &
\multicolumn{1}{r!{\huxvb{0}}}{\huxtpad{4pt}\raggedleft 50000.00\huxbpad{4pt}} \tabularnewline[-0.5pt]


\hhline{}
\arrayrulecolor{black}

\multicolumn{1}{!{\huxvb{0}}l!{\huxvb{0}}}{\huxtpad{4pt}\raggedright Jane Jones\huxbpad{4pt}} &
\multicolumn{1}{r!{\huxvb{0}}}{\huxtpad{4pt}\raggedleft 50000.00\huxbpad{4pt}} \tabularnewline[-0.5pt]


\hhline{}
\arrayrulecolor{black}

\multicolumn{1}{!{\huxvb{0}}l!{\huxvb{0}}}{\huxtpad{4pt}\raggedright Hadley Wickham\huxbpad{4pt}} &
\multicolumn{1}{r!{\huxvb{0}}}{\huxtpad{4pt}\raggedleft 100000.00\huxbpad{4pt}} \tabularnewline[-0.5pt]


\hhline{}
\arrayrulecolor{black}

\multicolumn{1}{!{\huxvb{0}}l!{\huxvb{0}}}{\huxtpad{4pt}\raggedright David Hugh-Jones\huxbpad{4pt}} &
\multicolumn{1}{r!{\huxvb{0}}}{\huxtpad{4pt}\raggedleft 40000.00\huxbpad{4pt}} \tabularnewline[-0.5pt]


\hhline{>{\huxb{0.8}}->{\huxb{0.8}}-}
\arrayrulecolor{black}

\multicolumn{2}{!{\huxvb{0}}p{0.5\textwidth+2\tabcolsep}!{\huxvb{0}}}{\parbox[b]{0.5\textwidth+2\tabcolsep-4pt-4pt}{\huxtpad{4pt}\raggedright DHJ deserves a pay rise\huxbpad{4pt}}} \tabularnewline[-0.5pt]


\hhline{}
\arrayrulecolor{black}
\end{tabularx}
\end{table}
 

\FloatBarrier

\hypertarget{more-formatting}{%
\section{More formatting}\label{more-formatting}}

\hypertarget{number-format}{%
\subsection{Number format}\label{number-format}}

You can change how huxtable formats numbers using
\texttt{number\_format}. Set \texttt{number\_format} to a number of
decimal places (for more advanced options, see the help files). This
doesn't just affect cells that are numbers: it works on any numbers
within the cells. So, for example, if you have a cell like ``12.00001
(3.0003)'', \texttt{number\_format} will affect both the numbers

\begin{Shaded}
\begin{Highlighting}[]
\NormalTok{pointy_ht <-}\StringTok{ }\KeywordTok{hux}\NormalTok{(}\KeywordTok{c}\NormalTok{(}\StringTok{"Column heading"}\NormalTok{, }\FloatTok{11.003}\NormalTok{, }\DecValTok{300}\NormalTok{, }\FloatTok{12.02}\NormalTok{, }\StringTok{"12.1 **"}\NormalTok{, }\StringTok{"mean 11.7 (se 2.3)"}\NormalTok{)) }\OperatorTok\StringTok{ }
\StringTok{    }\KeywordTok{set_all_borders}\NormalTok{(}\DecValTok{1}\NormalTok{)}

\KeywordTok{number_format}\NormalTok{(pointy_ht) <-}\StringTok{ }\DecValTok{3}
\NormalTok{pointy_ht}
\end{Highlighting}
\end{Shaded}

 \begin{table}[h]
\centering
    \providecommand{\huxb}[2][0,0,0]{\arrayrulecolor[RGB]{#1}\global\arrayrulewidth=#2pt}
    \providecommand{\huxvb}[2][0,0,0]{\color[RGB]{#1}\vrule width #2pt}
    \providecommand{\huxtpad}[1]{\rule{0pt}{\baselineskip+#1}}
    \providecommand{\huxbpad}[1]{\rule[-#1]{0pt}{#1}}
  \begin{tabularx}{0.5\textwidth}{p{0.5\textwidth}}


\hhline{>{\huxb{1}}-}
\arrayrulecolor{black}

\multicolumn{1}{!{\huxvb{1}}l!{\huxvb{1}}}{\huxtpad{4pt}\raggedright Column heading\huxbpad{4pt}} \tabularnewline[-0.5pt]


\hhline{>{\huxb{1}}-}
\arrayrulecolor{black}

\multicolumn{1}{!{\huxvb{1}}l!{\huxvb{1}}}{\huxtpad{4pt}\raggedright 11.003\huxbpad{4pt}} \tabularnewline[-0.5pt]


\hhline{>{\huxb{1}}-}
\arrayrulecolor{black}

\multicolumn{1}{!{\huxvb{1}}l!{\huxvb{1}}}{\huxtpad{4pt}\raggedright 300.000\huxbpad{4pt}} \tabularnewline[-0.5pt]


\hhline{>{\huxb{1}}-}
\arrayrulecolor{black}

\multicolumn{1}{!{\huxvb{1}}l!{\huxvb{1}}}{\huxtpad{4pt}\raggedright 12.020\huxbpad{4pt}} \tabularnewline[-0.5pt]


\hhline{>{\huxb{1}}-}
\arrayrulecolor{black}

\multicolumn{1}{!{\huxvb{1}}l!{\huxvb{1}}}{\huxtpad{4pt}\raggedright 12.100 **\huxbpad{4pt}} \tabularnewline[-0.5pt]


\hhline{>{\huxb{1}}-}
\arrayrulecolor{black}

\multicolumn{1}{!{\huxvb{1}}l!{\huxvb{1}}}{\huxtpad{4pt}\raggedright mean 11.700 (se 2.300)\huxbpad{4pt}} \tabularnewline[-0.5pt]


\hhline{>{\huxb{1}}-}
\arrayrulecolor{black}
\end{tabularx}
\end{table}
 

\FloatBarrier

You can also align columns by decimal places. If you want to do this for
a cell, just set the \texttt{align} property to `.' (or whatever you use
for a decimal point).

\begin{Shaded}
\begin{Highlighting}[]
\KeywordTok{align}\NormalTok{(pointy_ht)[}\DecValTok{2}\OperatorTok{:}\DecValTok{5}\NormalTok{, ] <-}\StringTok{ "."} \CommentTok{# not the first row}
\NormalTok{pointy_ht}
\end{Highlighting}
\end{Shaded}

 \begin{table}[h]
\centering
    \providecommand{\huxb}[2][0,0,0]{\arrayrulecolor[RGB]{#1}\global\arrayrulewidth=#2pt}
    \providecommand{\huxvb}[2][0,0,0]{\color[RGB]{#1}\vrule width #2pt}
    \providecommand{\huxtpad}[1]{\rule{0pt}{\baselineskip+#1}}
    \providecommand{\huxbpad}[1]{\rule[-#1]{0pt}{#1}}
  \begin{tabularx}{0.5\textwidth}{p{0.5\textwidth}}


\hhline{>{\huxb{1}}-}
\arrayrulecolor{black}

\multicolumn{1}{!{\huxvb{1}}l!{\huxvb{1}}}{\huxtpad{4pt}\raggedright Column heading\huxbpad{4pt}} \tabularnewline[-0.5pt]


\hhline{>{\huxb{1}}-}
\arrayrulecolor{black}

\multicolumn{1}{!{\huxvb{1}}r!{\huxvb{1}}}{\huxtpad{4pt}\raggedleft 11.003~~~\huxbpad{4pt}} \tabularnewline[-0.5pt]


\hhline{>{\huxb{1}}-}
\arrayrulecolor{black}

\multicolumn{1}{!{\huxvb{1}}r!{\huxvb{1}}}{\huxtpad{4pt}\raggedleft 300.000~~~\huxbpad{4pt}} \tabularnewline[-0.5pt]


\hhline{>{\huxb{1}}-}
\arrayrulecolor{black}

\multicolumn{1}{!{\huxvb{1}}r!{\huxvb{1}}}{\huxtpad{4pt}\raggedleft 12.020~~~\huxbpad{4pt}} \tabularnewline[-0.5pt]


\hhline{>{\huxb{1}}-}
\arrayrulecolor{black}

\multicolumn{1}{!{\huxvb{1}}r!{\huxvb{1}}}{\huxtpad{4pt}\raggedleft 12.100 **\huxbpad{4pt}} \tabularnewline[-0.5pt]


\hhline{>{\huxb{1}}-}
\arrayrulecolor{black}

\multicolumn{1}{!{\huxvb{1}}l!{\huxvb{1}}}{\huxtpad{4pt}\raggedright mean 11.700 (se 2.300)\huxbpad{4pt}} \tabularnewline[-0.5pt]


\hhline{>{\huxb{1}}-}
\arrayrulecolor{black}
\end{tabularx}
\end{table}
 

\FloatBarrier

There is currently no true way to align cells by the decimal point in
HTML, and only limited possibilities in TeX, so this works by
right-padding cells with spaces. The output may look better if you use a
fixed width font.

\hypertarget{automatic-formatting}{%
\subsection{Automatic formatting}\label{automatic-formatting}}

By default, when you create a huxtable using \texttt{huxtable} or
\texttt{as\_huxtable}, the package will guess defaults for number
formatting and alignment, based on the type of data in your columns.
Numeric data will be right-aligned or aligned on the decimal point;
character data will be left aligned; and the package will try to set
sensible defaults for number formatting. If you want to, you can turn
this off with \texttt{autoformat\ =\ FALSE}:

\begin{Shaded}
\begin{Highlighting}[]
\NormalTok{my_data <-}\StringTok{ }\KeywordTok{data.frame}\NormalTok{(}
        \DataTypeTok{Employee           =} \KeywordTok{c}\NormalTok{(}\StringTok{"John Smith"}\NormalTok{, }\StringTok{"Jane Doe"}\NormalTok{, }\StringTok{"David Hugh-Jones"}\NormalTok{), }
        \DataTypeTok{Salary             =} \KeywordTok{c}\NormalTok{(50000L, 50000L, 40000L),}
        \DataTypeTok{Performance_rating =} \KeywordTok{c}\NormalTok{(}\FloatTok{8.9}\NormalTok{, }\FloatTok{9.2}\NormalTok{, }\FloatTok{7.8}\NormalTok{)  }
\NormalTok{      )}
\KeywordTok{as_huxtable}\NormalTok{(my_data, }\DataTypeTok{add_colnames =} \OtherTok{TRUE}\NormalTok{) }\CommentTok{# with automatic formatting}
\end{Highlighting}
\end{Shaded}

 \begin{table}[h]
\centering
    \providecommand{\huxb}[2][0,0,0]{\arrayrulecolor[RGB]{#1}\global\arrayrulewidth=#2pt}
    \providecommand{\huxvb}[2][0,0,0]{\color[RGB]{#1}\vrule width #2pt}
    \providecommand{\huxtpad}[1]{\rule{0pt}{\baselineskip+#1}}
    \providecommand{\huxbpad}[1]{\rule[-#1]{0pt}{#1}}
  \begin{tabularx}{0.5\textwidth}{p{0.166666666666667\textwidth} p{0.166666666666667\textwidth} p{0.166666666666667\textwidth}}


\hhline{}
\arrayrulecolor{black}

\multicolumn{1}{!{\huxvb{0}}l!{\huxvb{0}}}{\huxtpad{4pt}\raggedright Employee\huxbpad{4pt}} &
\multicolumn{1}{r!{\huxvb{0}}}{\huxtpad{4pt}\raggedleft Salary\huxbpad{4pt}} &
\multicolumn{1}{r!{\huxvb{0}}}{\huxtpad{4pt}\raggedleft Performance\_rating\huxbpad{4pt}} \tabularnewline[-0.5pt]


\hhline{}
\arrayrulecolor{black}

\multicolumn{1}{!{\huxvb{0}}l!{\huxvb{0}}}{\huxtpad{4pt}\raggedright John Smith\huxbpad{4pt}} &
\multicolumn{1}{r!{\huxvb{0}}}{\huxtpad{4pt}\raggedleft 50000\huxbpad{4pt}} &
\multicolumn{1}{r!{\huxvb{0}}}{\huxtpad{4pt}\raggedleft 8.9\huxbpad{4pt}} \tabularnewline[-0.5pt]


\hhline{}
\arrayrulecolor{black}

\multicolumn{1}{!{\huxvb{0}}l!{\huxvb{0}}}{\huxtpad{4pt}\raggedright Jane Doe\huxbpad{4pt}} &
\multicolumn{1}{r!{\huxvb{0}}}{\huxtpad{4pt}\raggedleft 50000\huxbpad{4pt}} &
\multicolumn{1}{r!{\huxvb{0}}}{\huxtpad{4pt}\raggedleft 9.2\huxbpad{4pt}} \tabularnewline[-0.5pt]


\hhline{}
\arrayrulecolor{black}

\multicolumn{1}{!{\huxvb{0}}l!{\huxvb{0}}}{\huxtpad{4pt}\raggedright David Hugh-Jones\huxbpad{4pt}} &
\multicolumn{1}{r!{\huxvb{0}}}{\huxtpad{4pt}\raggedleft 40000\huxbpad{4pt}} &
\multicolumn{1}{r!{\huxvb{0}}}{\huxtpad{4pt}\raggedleft 7.8\huxbpad{4pt}} \tabularnewline[-0.5pt]


\hhline{}
\arrayrulecolor{black}
\end{tabularx}
\end{table}
 

\begin{Shaded}
\begin{Highlighting}[]
\KeywordTok{as_huxtable}\NormalTok{(my_data, }\DataTypeTok{add_colnames =} \OtherTok{TRUE}\NormalTok{, }\DataTypeTok{autoformat =} \OtherTok{FALSE}\NormalTok{) }\CommentTok{# no automatic formatting}
\end{Highlighting}
\end{Shaded}

 \begin{table}[h]
\centering
    \providecommand{\huxb}[2][0,0,0]{\arrayrulecolor[RGB]{#1}\global\arrayrulewidth=#2pt}
    \providecommand{\huxvb}[2][0,0,0]{\color[RGB]{#1}\vrule width #2pt}
    \providecommand{\huxtpad}[1]{\rule{0pt}{\baselineskip+#1}}
    \providecommand{\huxbpad}[1]{\rule[-#1]{0pt}{#1}}
  \begin{tabularx}{0.5\textwidth}{p{0.166666666666667\textwidth} p{0.166666666666667\textwidth} p{0.166666666666667\textwidth}}


\hhline{}
\arrayrulecolor{black}

\multicolumn{1}{!{\huxvb{0}}l!{\huxvb{0}}}{\huxtpad{4pt}\raggedright Employee\huxbpad{4pt}} &
\multicolumn{1}{l!{\huxvb{0}}}{\huxtpad{4pt}\raggedright Salary\huxbpad{4pt}} &
\multicolumn{1}{l!{\huxvb{0}}}{\huxtpad{4pt}\raggedright Performance\_rating\huxbpad{4pt}} \tabularnewline[-0.5pt]


\hhline{}
\arrayrulecolor{black}

\multicolumn{1}{!{\huxvb{0}}l!{\huxvb{0}}}{\huxtpad{4pt}\raggedright John Smith\huxbpad{4pt}} &
\multicolumn{1}{l!{\huxvb{0}}}{\huxtpad{4pt}\raggedright 5e+04\huxbpad{4pt}} &
\multicolumn{1}{l!{\huxvb{0}}}{\huxtpad{4pt}\raggedright 8.9\huxbpad{4pt}} \tabularnewline[-0.5pt]


\hhline{}
\arrayrulecolor{black}

\multicolumn{1}{!{\huxvb{0}}l!{\huxvb{0}}}{\huxtpad{4pt}\raggedright Jane Doe\huxbpad{4pt}} &
\multicolumn{1}{l!{\huxvb{0}}}{\huxtpad{4pt}\raggedright 5e+04\huxbpad{4pt}} &
\multicolumn{1}{l!{\huxvb{0}}}{\huxtpad{4pt}\raggedright 9.2\huxbpad{4pt}} \tabularnewline[-0.5pt]


\hhline{}
\arrayrulecolor{black}

\multicolumn{1}{!{\huxvb{0}}l!{\huxvb{0}}}{\huxtpad{4pt}\raggedright David Hugh-Jones\huxbpad{4pt}} &
\multicolumn{1}{l!{\huxvb{0}}}{\huxtpad{4pt}\raggedright 4e+04\huxbpad{4pt}} &
\multicolumn{1}{l!{\huxvb{0}}}{\huxtpad{4pt}\raggedright 7.8\huxbpad{4pt}} \tabularnewline[-0.5pt]


\hhline{}
\arrayrulecolor{black}
\end{tabularx}
\end{table}
 

\FloatBarrier

\hypertarget{escaping-html-or-latex}{%
\subsection{Escaping HTML or LaTeX}\label{escaping-html-or-latex}}

By default, HTML or LaTeX code will be escaped:

\begin{Shaded}
\begin{Highlighting}[]
\NormalTok{code_ht <-}\StringTok{ }\ControlFlowTok{if}\NormalTok{ (is_latex) }\KeywordTok{hux}\NormalTok{(}\KeywordTok{c}\NormalTok{(}\StringTok{"Some maths"}\NormalTok{, }\StringTok{"$a^b$"}\NormalTok{)) }\ControlFlowTok{else} 
      \KeywordTok{hux}\NormalTok{(}\KeywordTok{c}\NormalTok{(}\StringTok{"Copyright symbol"}\NormalTok{, }\StringTok{"&copy;"}\NormalTok{))}
\NormalTok{code_ht}
\end{Highlighting}
\end{Shaded}

 \begin{table}[h]
\centering
    \providecommand{\huxb}[2][0,0,0]{\arrayrulecolor[RGB]{#1}\global\arrayrulewidth=#2pt}
    \providecommand{\huxvb}[2][0,0,0]{\color[RGB]{#1}\vrule width #2pt}
    \providecommand{\huxtpad}[1]{\rule{0pt}{\baselineskip+#1}}
    \providecommand{\huxbpad}[1]{\rule[-#1]{0pt}{#1}}
  \begin{tabularx}{0.5\textwidth}{p{0.5\textwidth}}


\hhline{}
\arrayrulecolor{black}

\multicolumn{1}{!{\huxvb{0}}l!{\huxvb{0}}}{\huxtpad{4pt}\raggedright Some maths\huxbpad{4pt}} \tabularnewline[-0.5pt]


\hhline{}
\arrayrulecolor{black}

\multicolumn{1}{!{\huxvb{0}}l!{\huxvb{0}}}{\huxtpad{4pt}\raggedright \$a\verb|^|b\$\huxbpad{4pt}} \tabularnewline[-0.5pt]


\hhline{}
\arrayrulecolor{black}
\end{tabularx}
\end{table}
 

\FloatBarrier

To avoid this, set the \texttt{escape\_contents} property to
\texttt{FALSE}.

\begin{Shaded}
\begin{Highlighting}[]
\KeywordTok{escape_contents}\NormalTok{(code_ht)[}\DecValTok{2}\NormalTok{, }\DecValTok{1}\NormalTok{] <-}\StringTok{ }\OtherTok{FALSE}
\NormalTok{code_ht}
\end{Highlighting}
\end{Shaded}

 \begin{table}[h]
\centering
    \providecommand{\huxb}[2][0,0,0]{\arrayrulecolor[RGB]{#1}\global\arrayrulewidth=#2pt}
    \providecommand{\huxvb}[2][0,0,0]{\color[RGB]{#1}\vrule width #2pt}
    \providecommand{\huxtpad}[1]{\rule{0pt}{\baselineskip+#1}}
    \providecommand{\huxbpad}[1]{\rule[-#1]{0pt}{#1}}
  \begin{tabularx}{0.5\textwidth}{p{0.5\textwidth}}


\hhline{}
\arrayrulecolor{black}

\multicolumn{1}{!{\huxvb{0}}l!{\huxvb{0}}}{\huxtpad{4pt}\raggedright Some maths\huxbpad{4pt}} \tabularnewline[-0.5pt]


\hhline{}
\arrayrulecolor{black}

\multicolumn{1}{!{\huxvb{0}}l!{\huxvb{0}}}{\huxtpad{4pt}\raggedright $a^b$\huxbpad{4pt}} \tabularnewline[-0.5pt]


\hhline{}
\arrayrulecolor{black}
\end{tabularx}
\end{table}
 

\FloatBarrier

\hypertarget{width-and-cell-wrapping}{%
\subsection{Width and cell wrapping}\label{width-and-cell-wrapping}}

You can set table widths using the \texttt{width} property, and column
widths using the \texttt{col\_width} property. If you use numbers for
these, they will be interpreted as proportions of the table width (or
for \texttt{width}, a proportion of the width of the surrounding text).
If you use character vectors, they must be valid CSS or LaTeX widths.
The only unit both systems have in common is \texttt{pt} for points.

\begin{Shaded}
\begin{Highlighting}[]
\KeywordTok{width}\NormalTok{(ht) <-}\StringTok{ }\FloatTok{0.35}
\KeywordTok{col_width}\NormalTok{(ht) <-}\StringTok{ }\KeywordTok{c}\NormalTok{(.}\DecValTok{7}\NormalTok{, }\FloatTok{.3}\NormalTok{)}
\NormalTok{ht}
\end{Highlighting}
\end{Shaded}

 \begin{table}[h]
\centering\captionsetup{justification=centering,singlelinecheck=off}
\caption{Employee table}

    \providecommand{\huxb}[2][0,0,0]{\arrayrulecolor[RGB]{#1}\global\arrayrulewidth=#2pt}
    \providecommand{\huxvb}[2][0,0,0]{\color[RGB]{#1}\vrule width #2pt}
    \providecommand{\huxtpad}[1]{\rule{0pt}{\baselineskip+#1}}
    \providecommand{\huxbpad}[1]{\rule[-#1]{0pt}{#1}}
  \begin{tabularx}{0.35\textwidth}{p{0.245\textwidth} p{0.105\textwidth}}


\hhline{}
\arrayrulecolor{black}

\multicolumn{1}{!{\huxvb{0}}l!{\huxvb{0}}}{\huxtpad{4pt}\raggedright \textbf{Employee}\huxbpad{4pt}} &
\multicolumn{1}{r!{\huxvb{0}}}{\huxtpad{4pt}\raggedleft \textbf{Salary}\huxbpad{4pt}} \tabularnewline[-0.5pt]


\hhline{>{\huxb{1}}->{\huxb{1}}-}
\arrayrulecolor{black}

\multicolumn{1}{!{\huxvb{0}}l!{\huxvb{0}}}{\huxtpad{4pt}\raggedright John Smith\huxbpad{4pt}} &
\multicolumn{1}{r!{\huxvb{0}}}{\huxtpad{4pt}\raggedleft 50000.00\huxbpad{4pt}} \tabularnewline[-0.5pt]


\hhline{}
\arrayrulecolor{black}

\multicolumn{1}{!{\huxvb{0}}l!{\huxvb{0}}}{\huxtpad{4pt}\raggedright Jane Jones\huxbpad{4pt}} &
\multicolumn{1}{r!{\huxvb{0}}}{\huxtpad{4pt}\raggedleft 50000.00\huxbpad{4pt}} \tabularnewline[-0.5pt]


\hhline{}
\arrayrulecolor{black}

\multicolumn{1}{!{\huxvb{0}}l!{\huxvb{0}}}{\huxtpad{4pt}\raggedright Hadley Wickham\huxbpad{4pt}} &
\multicolumn{1}{r!{\huxvb{0}}}{\huxtpad{4pt}\raggedleft 100000.00\huxbpad{4pt}} \tabularnewline[-0.5pt]


\hhline{}
\arrayrulecolor{black}

\multicolumn{1}{!{\huxvb{0}}l!{\huxvb{0}}}{\huxtpad{4pt}\raggedright David Hugh-Jones\huxbpad{4pt}} &
\multicolumn{1}{r!{\huxvb{0}}}{\huxtpad{4pt}\raggedleft 40000.00\huxbpad{4pt}} \tabularnewline[-0.5pt]


\hhline{>{\huxb{0.8}}->{\huxb{0.8}}-}
\arrayrulecolor{black}

\multicolumn{2}{!{\huxvb{0}}p{0.35\textwidth+2\tabcolsep}!{\huxvb{0}}}{\parbox[b]{0.35\textwidth+2\tabcolsep-4pt-4pt}{\huxtpad{4pt}\raggedright DHJ deserves a pay rise\huxbpad{4pt}}} \tabularnewline[-0.5pt]


\hhline{}
\arrayrulecolor{black}
\end{tabularx}
\end{table}
 

\FloatBarrier

It is best to set table width explicitly, then set column widths as
proportions.

By default, if a cell contains long contents, it will be stretched. Use
the \texttt{wrap} property to allow cell contents to wrap over multiple
lines:

\begin{Shaded}
\begin{Highlighting}[]
\NormalTok{ht_wrapped <-}\StringTok{ }\NormalTok{ht}
\NormalTok{ht_wrapped[}\DecValTok{5}\NormalTok{, }\DecValTok{1}\NormalTok{] <-}\StringTok{ "David Arthur Shrimpton Hugh-Jones"}
\KeywordTok{wrap}\NormalTok{(ht_wrapped) <-}\StringTok{ }\OtherTok{TRUE}
\NormalTok{ht_wrapped}
\end{Highlighting}
\end{Shaded}

 \begin{table}[h]
\centering\captionsetup{justification=centering,singlelinecheck=off}
\caption{Employee table}

    \providecommand{\huxb}[2][0,0,0]{\arrayrulecolor[RGB]{#1}\global\arrayrulewidth=#2pt}
    \providecommand{\huxvb}[2][0,0,0]{\color[RGB]{#1}\vrule width #2pt}
    \providecommand{\huxtpad}[1]{\rule{0pt}{\baselineskip+#1}}
    \providecommand{\huxbpad}[1]{\rule[-#1]{0pt}{#1}}
  \begin{tabularx}{0.35\textwidth}{p{0.245\textwidth} p{0.105\textwidth}}


\hhline{}
\arrayrulecolor{black}

\multicolumn{1}{!{\huxvb{0}}p{0.245\textwidth}!{\huxvb{0}}}{\parbox[b]{0.245\textwidth-10pt-10pt}{\huxtpad{4pt}\raggedright \textbf{Employee}\huxbpad{4pt}}} &
\multicolumn{1}{p{0.105\textwidth}!{\huxvb{0}}}{\parbox[b]{0.105\textwidth-10pt-10pt}{\huxtpad{4pt}\raggedleft \textbf{Salary}\huxbpad{4pt}}} \tabularnewline[-0.5pt]


\hhline{>{\huxb{1}}->{\huxb{1}}-}
\arrayrulecolor{black}

\multicolumn{1}{!{\huxvb{0}}p{0.245\textwidth}!{\huxvb{0}}}{\parbox[b]{0.245\textwidth-10pt-10pt}{\huxtpad{4pt}\raggedright John Smith\huxbpad{4pt}}} &
\multicolumn{1}{p{0.105\textwidth}!{\huxvb{0}}}{\parbox[b]{0.105\textwidth-10pt-10pt}{\huxtpad{4pt}\raggedleft 50000.00\huxbpad{4pt}}} \tabularnewline[-0.5pt]


\hhline{}
\arrayrulecolor{black}

\multicolumn{1}{!{\huxvb{0}}p{0.245\textwidth}!{\huxvb{0}}}{\parbox[b]{0.245\textwidth-10pt-10pt}{\huxtpad{4pt}\raggedright Jane Jones\huxbpad{4pt}}} &
\multicolumn{1}{p{0.105\textwidth}!{\huxvb{0}}}{\parbox[b]{0.105\textwidth-10pt-10pt}{\huxtpad{4pt}\raggedleft 50000.00\huxbpad{4pt}}} \tabularnewline[-0.5pt]


\hhline{}
\arrayrulecolor{black}

\multicolumn{1}{!{\huxvb{0}}p{0.245\textwidth}!{\huxvb{0}}}{\parbox[b]{0.245\textwidth-10pt-10pt}{\huxtpad{4pt}\raggedright Hadley Wickham\huxbpad{4pt}}} &
\multicolumn{1}{p{0.105\textwidth}!{\huxvb{0}}}{\parbox[b]{0.105\textwidth-10pt-10pt}{\huxtpad{4pt}\raggedleft 100000.00\huxbpad{4pt}}} \tabularnewline[-0.5pt]


\hhline{}
\arrayrulecolor{black}

\multicolumn{1}{!{\huxvb{0}}p{0.245\textwidth}!{\huxvb{0}}}{\parbox[b]{0.245\textwidth-10pt-10pt}{\huxtpad{4pt}\raggedright David Arthur Shrimpton Hugh-Jones\huxbpad{4pt}}} &
\multicolumn{1}{p{0.105\textwidth}!{\huxvb{0}}}{\parbox[b]{0.105\textwidth-10pt-10pt}{\huxtpad{4pt}\raggedleft 40000.00\huxbpad{4pt}}} \tabularnewline[-0.5pt]


\hhline{>{\huxb{0.8}}->{\huxb{0.8}}-}
\arrayrulecolor{black}

\multicolumn{2}{!{\huxvb{0}}p{0.35\textwidth+2\tabcolsep}!{\huxvb{0}}}{\parbox[b]{0.35\textwidth+2\tabcolsep-4pt-4pt}{\huxtpad{4pt}\raggedright DHJ deserves a pay rise\huxbpad{4pt}}} \tabularnewline[-0.5pt]


\hhline{}
\arrayrulecolor{black}
\end{tabularx}
\end{table}
 

\FloatBarrier

\hypertarget{adding-row-and-column-names}{%
\subsection{Adding row and column
names}\label{adding-row-and-column-names}}

Just like data frames, huxtables can have row and column names. Often,
we want to add these to the final table. You can do this using either
the \texttt{add\_colnames}/\texttt{add\_rownames} arguments to
\texttt{as\_huxtable}, or the
\texttt{add\_colnames()}/\texttt{add\_rownames()} functions. (Note that
earlier versions of \texttt{dplyr} used to have functions with the same
name.)

\begin{Shaded}
\begin{Highlighting}[]
\KeywordTok{as_hux}\NormalTok{(mtcars[}\DecValTok{1}\OperatorTok{:}\DecValTok{4}\NormalTok{, }\DecValTok{1}\OperatorTok{:}\DecValTok{4}\NormalTok{])                           }\OperatorTok\StringTok{ }
\StringTok{      }\NormalTok{huxtable}\OperatorTok{::}\KeywordTok{add_rownames}\NormalTok{(}\DataTypeTok{colname =} \StringTok{"Car name"}\NormalTok{) }\OperatorTok\StringTok{ }
\StringTok{      }\NormalTok{huxtable}\OperatorTok{::}\KeywordTok{add_colnames}\NormalTok{()}
\end{Highlighting}
\end{Shaded}

 \begin{table}[h]
\centering
    \providecommand{\huxb}[2][0,0,0]{\arrayrulecolor[RGB]{#1}\global\arrayrulewidth=#2pt}
    \providecommand{\huxvb}[2][0,0,0]{\color[RGB]{#1}\vrule width #2pt}
    \providecommand{\huxtpad}[1]{\rule{0pt}{\baselineskip+#1}}
    \providecommand{\huxbpad}[1]{\rule[-#1]{0pt}{#1}}
  \begin{tabularx}{0.5\textwidth}{p{0.1\textwidth} p{0.1\textwidth} p{0.1\textwidth} p{0.1\textwidth} p{0.1\textwidth}}


\hhline{}
\arrayrulecolor{black}

\multicolumn{1}{!{\huxvb{0}}l!{\huxvb{0}}}{\huxtpad{4pt}\raggedright Car name\huxbpad{4pt}} &
\multicolumn{1}{l!{\huxvb{0}}}{\huxtpad{4pt}\raggedright mpg\huxbpad{4pt}} &
\multicolumn{1}{l!{\huxvb{0}}}{\huxtpad{4pt}\raggedright cyl\huxbpad{4pt}} &
\multicolumn{1}{l!{\huxvb{0}}}{\huxtpad{4pt}\raggedright disp\huxbpad{4pt}} &
\multicolumn{1}{l!{\huxvb{0}}}{\huxtpad{4pt}\raggedright hp\huxbpad{4pt}} \tabularnewline[-0.5pt]


\hhline{}
\arrayrulecolor{black}

\multicolumn{1}{!{\huxvb{0}}l!{\huxvb{0}}}{\huxtpad{4pt}\raggedright Mazda RX4\huxbpad{4pt}} &
\multicolumn{1}{r!{\huxvb{0}}}{\huxtpad{4pt}\raggedleft 21~~\huxbpad{4pt}} &
\multicolumn{1}{r!{\huxvb{0}}}{\huxtpad{4pt}\raggedleft 6\huxbpad{4pt}} &
\multicolumn{1}{r!{\huxvb{0}}}{\huxtpad{4pt}\raggedleft 160\huxbpad{4pt}} &
\multicolumn{1}{r!{\huxvb{0}}}{\huxtpad{4pt}\raggedleft 110\huxbpad{4pt}} \tabularnewline[-0.5pt]


\hhline{}
\arrayrulecolor{black}

\multicolumn{1}{!{\huxvb{0}}l!{\huxvb{0}}}{\huxtpad{4pt}\raggedright Mazda RX4 Wag\huxbpad{4pt}} &
\multicolumn{1}{r!{\huxvb{0}}}{\huxtpad{4pt}\raggedleft 21~~\huxbpad{4pt}} &
\multicolumn{1}{r!{\huxvb{0}}}{\huxtpad{4pt}\raggedleft 6\huxbpad{4pt}} &
\multicolumn{1}{r!{\huxvb{0}}}{\huxtpad{4pt}\raggedleft 160\huxbpad{4pt}} &
\multicolumn{1}{r!{\huxvb{0}}}{\huxtpad{4pt}\raggedleft 110\huxbpad{4pt}} \tabularnewline[-0.5pt]


\hhline{}
\arrayrulecolor{black}

\multicolumn{1}{!{\huxvb{0}}l!{\huxvb{0}}}{\huxtpad{4pt}\raggedright Datsun 710\huxbpad{4pt}} &
\multicolumn{1}{r!{\huxvb{0}}}{\huxtpad{4pt}\raggedleft 22.8\huxbpad{4pt}} &
\multicolumn{1}{r!{\huxvb{0}}}{\huxtpad{4pt}\raggedleft 4\huxbpad{4pt}} &
\multicolumn{1}{r!{\huxvb{0}}}{\huxtpad{4pt}\raggedleft 108\huxbpad{4pt}} &
\multicolumn{1}{r!{\huxvb{0}}}{\huxtpad{4pt}\raggedleft 93\huxbpad{4pt}} \tabularnewline[-0.5pt]


\hhline{}
\arrayrulecolor{black}

\multicolumn{1}{!{\huxvb{0}}l!{\huxvb{0}}}{\huxtpad{4pt}\raggedright Hornet 4 Drive\huxbpad{4pt}} &
\multicolumn{1}{r!{\huxvb{0}}}{\huxtpad{4pt}\raggedleft 21.4\huxbpad{4pt}} &
\multicolumn{1}{r!{\huxvb{0}}}{\huxtpad{4pt}\raggedleft 6\huxbpad{4pt}} &
\multicolumn{1}{r!{\huxvb{0}}}{\huxtpad{4pt}\raggedleft 258\huxbpad{4pt}} &
\multicolumn{1}{r!{\huxvb{0}}}{\huxtpad{4pt}\raggedleft 110\huxbpad{4pt}} \tabularnewline[-0.5pt]


\hhline{}
\arrayrulecolor{black}
\end{tabularx}
\end{table}
 

\FloatBarrier

\hypertarget{merging-cells}{%
\subsection{Merging cells}\label{merging-cells}}

Sometimes you want a single cell to spread over more than one row or
column: for example, if you want a heading that covers several different
rows.

You can do this by calling \texttt{merge\_cells(ht,\ rows,\ cols)}.
\texttt{rows} and \texttt{cols} should be a contiguous sequence of
numbers. The rectangle of cells \texttt{ht{[}rows,\ cols{]}} will be
merged.

When cells in a rectangle are merged, all cells apart from the top left
one are hidden, along with any properties they have. So if you want to
set cell properties, you have to target the top left cell.

Here, we'll add some row and column headings to the \texttt{mtcars}
dataset:

\begin{Shaded}
\begin{Highlighting}[]
\NormalTok{cars_mpg <-}\StringTok{ }\KeywordTok{cbind}\NormalTok{(}\DataTypeTok{cylinders =}\NormalTok{ cars_mpg}\OperatorTok{$}\NormalTok{cyl, cars_mpg)}
\NormalTok{cars_mpg}\OperatorTok{$}\NormalTok{cylinders[}\DecValTok{1}\NormalTok{]   <-}\StringTok{ ""}
\NormalTok{cars_mpg}\OperatorTok{$}\NormalTok{cylinders[}\DecValTok{2}\NormalTok{]   <-}\StringTok{ "Four cylinders"}
\NormalTok{cars_mpg}\OperatorTok{$}\NormalTok{cylinders[}\DecValTok{13}\NormalTok{]  <-}\StringTok{ "Six cylinders"}
\NormalTok{cars_mpg}\OperatorTok{$}\NormalTok{cylinders[}\DecValTok{20}\NormalTok{]  <-}\StringTok{ "Eight cylinders"}

\NormalTok{cars_mpg <-}\StringTok{ }\NormalTok{cars_mpg }\OperatorTok\StringTok{  }
\StringTok{  }\KeywordTok{merge_cells}\NormalTok{(}\DecValTok{2}\OperatorTok{:}\DecValTok{12}\NormalTok{, }\DecValTok{1}\NormalTok{) }\OperatorTok\StringTok{ }
\StringTok{  }\KeywordTok{merge_cells}\NormalTok{(}\DecValTok{13}\OperatorTok{:}\DecValTok{19}\NormalTok{, }\DecValTok{1}\NormalTok{) }\OperatorTok\StringTok{ }
\StringTok{  }\KeywordTok{merge_cells}\NormalTok{(}\DecValTok{20}\OperatorTok{:}\DecValTok{33}\NormalTok{, }\DecValTok{1}\NormalTok{)}

\NormalTok{cars_mpg <-}\StringTok{ }\KeywordTok{rbind}\NormalTok{(}\KeywordTok{c}\NormalTok{(}\StringTok{"List of cars"}\NormalTok{, }\StringTok{""}\NormalTok{, }\StringTok{""}\NormalTok{, }\StringTok{""}\NormalTok{, }\StringTok{""}\NormalTok{), cars_mpg)}
\NormalTok{cars_mpg <-}\StringTok{ }\KeywordTok{merge_cells}\NormalTok{(cars_mpg, }\DecValTok{1}\NormalTok{, }\DecValTok{1}\OperatorTok{:}\DecValTok{5}\NormalTok{)}
\KeywordTok{align}\NormalTok{(cars_mpg)[}\DecValTok{1}\NormalTok{, }\DecValTok{1}\NormalTok{] <-}\StringTok{ "center"}

\CommentTok{# a little more formatting:}

\NormalTok{cars_mpg <-}\StringTok{ }\KeywordTok{set_all_padding}\NormalTok{(cars_mpg, }\DecValTok{2}\NormalTok{)}
\NormalTok{cars_mpg <-}\StringTok{ }\KeywordTok{set_all_borders}\NormalTok{(cars_mpg, }\DecValTok{1}\NormalTok{)}
\KeywordTok{valign}\NormalTok{(cars_mpg)[}\DecValTok{1}\NormalTok{,] <-}\StringTok{ "top"}
\KeywordTok{col_width}\NormalTok{(cars_mpg) <-}\StringTok{ }\KeywordTok{c}\NormalTok{(.}\DecValTok{4}\NormalTok{ , }\FloatTok{.3}\NormalTok{ , }\FloatTok{.1}\NormalTok{, }\FloatTok{.1}\NormalTok{, }\FloatTok{.1}\NormalTok{)}
\KeywordTok{number_format}\NormalTok{(cars_mpg)[, }\DecValTok{4}\OperatorTok{:}\DecValTok{5}\NormalTok{] <-}\StringTok{ }\DecValTok{0}
\KeywordTok{bold}\NormalTok{(cars_mpg)[}\DecValTok{1}\OperatorTok{:}\DecValTok{2}\NormalTok{, ] <-}\StringTok{ }\OtherTok{TRUE}
\KeywordTok{bold}\NormalTok{(cars_mpg)[, }\DecValTok{1}\NormalTok{] <-}\StringTok{ }\OtherTok{TRUE}
\ControlFlowTok{if}\NormalTok{ (is_latex) }\KeywordTok{font_size}\NormalTok{(cars_mpg) <-}\StringTok{ }\DecValTok{10}
\NormalTok{cars_mpg}
\end{Highlighting}
\end{Shaded}

 \begin{table}[h]
\centering
    \providecommand{\huxb}[2][0,0,0]{\arrayrulecolor[RGB]{#1}\global\arrayrulewidth=#2pt}
    \providecommand{\huxvb}[2][0,0,0]{\color[RGB]{#1}\vrule width #2pt}
    \providecommand{\huxtpad}[1]{\rule{0pt}{\baselineskip+#1}}
    \providecommand{\huxbpad}[1]{\rule[-#1]{0pt}{#1}}
  \begin{tabularx}{0.5\textwidth}{p{0.2\textwidth} p{0.15\textwidth} p{0.05\textwidth} p{0.05\textwidth} p{0.05\textwidth}}


\hhline{>{\huxb{1}}->{\huxb{1}}->{\huxb{1}}->{\huxb{1}}->{\huxb{1}}-}
\arrayrulecolor{black}

\multicolumn{5}{!{\huxvb{1}}c!{\huxvb[190, 190, 190]{1}}}{\huxtpad{2pt}\centering \textbf{{\fontsize{10pt}{12pt}\selectfont List of cars}}\huxbpad{2pt}} \tabularnewline[-0.5pt]


\hhline{>{\huxb{1}}->{\huxb{1}}->{\huxb{1}}->{\huxb{1}}->{\huxb{1}}-}
\arrayrulecolor{black}

\multicolumn{1}{!{\huxvb{1}}l!{\huxvb[190, 190, 190]{1}}}{\huxtpad{2pt}\raggedright \textbf{{\fontsize{10pt}{12pt}\selectfont }}\huxbpad{2pt}} &
\multicolumn{1}{l!{\huxvb[190, 190, 190]{1}}}{\huxtpad{2pt}\raggedright \textbf{{\fontsize{10pt}{12pt}\selectfont Car}}\huxbpad{2pt}} &
\multicolumn{1}{l!{\huxvb[190, 190, 190]{1}}}{\huxtpad{2pt}\raggedright \textbf{{\fontsize{10pt}{12pt}\selectfont mpg}}\huxbpad{2pt}} &
\multicolumn{1}{l!{\huxvb[190, 190, 190]{1}}}{\huxtpad{2pt}\raggedright \textbf{{\fontsize{10pt}{12pt}\selectfont cyl}}\huxbpad{2pt}} &
\multicolumn{1}{l!{\huxvb[190, 190, 190]{1}}}{\huxtpad{2pt}\raggedright \textbf{{\fontsize{10pt}{12pt}\selectfont am}}\huxbpad{2pt}} \tabularnewline[-0.5pt]


\hhline{>{\huxb{1}}->{\huxb{1}}->{\huxb{1}}->{\huxb{1}}->{\huxb{1}}-}
\arrayrulecolor{black}

\multicolumn{1}{!{\huxvb{1}}l!{\huxvb[190, 190, 190]{1}}}{} &
\multicolumn{1}{l!{\huxvb[190, 190, 190]{1}}}{\huxtpad{2pt}\raggedright {\fontsize{10pt}{12pt}\selectfont Datsun 710}\huxbpad{2pt}} &
\multicolumn{1}{r!{\huxvb[190, 190, 190]{1}}}{\huxtpad{2pt}\raggedleft {\fontsize{10pt}{12pt}\selectfont 22.8}\huxbpad{2pt}} &
\multicolumn{1}{r!{\huxvb[190, 190, 190]{1}}}{\huxtpad{2pt}\raggedleft {\fontsize{10pt}{12pt}\selectfont 4}\huxbpad{2pt}} &
\multicolumn{1}{r!{\huxvb[190, 190, 190]{1}}}{\huxtpad{2pt}\raggedleft {\fontsize{10pt}{12pt}\selectfont 1}\huxbpad{2pt}} \tabularnewline[-0.5pt]


\hhline{>{\huxb{1}}|>{\huxb[255, 255, 255]{1}}->{\huxb{1}}|>{\huxb{1}}->{\huxb{1}}->{\huxb{1}}->{\huxb{1}}-}
\arrayrulecolor{black}

\multicolumn{1}{!{\huxvb{1}}l!{\huxvb[190, 190, 190]{1}}}{} &
\multicolumn{1}{l!{\huxvb[190, 190, 190]{1}}}{\huxtpad{2pt}\raggedright {\fontsize{10pt}{12pt}\selectfont Merc 240D}\huxbpad{2pt}} &
\multicolumn{1}{r!{\huxvb[190, 190, 190]{1}}}{\huxtpad{2pt}\raggedleft {\fontsize{10pt}{12pt}\selectfont 24.4}\huxbpad{2pt}} &
\multicolumn{1}{r!{\huxvb[190, 190, 190]{1}}}{\huxtpad{2pt}\raggedleft {\fontsize{10pt}{12pt}\selectfont 4}\huxbpad{2pt}} &
\multicolumn{1}{r!{\huxvb[190, 190, 190]{1}}}{\huxtpad{2pt}\raggedleft {\fontsize{10pt}{12pt}\selectfont 0}\huxbpad{2pt}} \tabularnewline[-0.5pt]


\hhline{>{\huxb{1}}|>{\huxb[255, 255, 255]{1}}->{\huxb{1}}|>{\huxb{1}}->{\huxb{1}}->{\huxb{1}}->{\huxb{1}}-}
\arrayrulecolor{black}

\multicolumn{1}{!{\huxvb{1}}l!{\huxvb[190, 190, 190]{1}}}{} &
\multicolumn{1}{l!{\huxvb[190, 190, 190]{1}}}{\huxtpad{2pt}\raggedright {\fontsize{10pt}{12pt}\selectfont Merc 230}\huxbpad{2pt}} &
\multicolumn{1}{r!{\huxvb[190, 190, 190]{1}}}{\huxtpad{2pt}\raggedleft {\fontsize{10pt}{12pt}\selectfont 22.8}\huxbpad{2pt}} &
\multicolumn{1}{r!{\huxvb[190, 190, 190]{1}}}{\huxtpad{2pt}\raggedleft {\fontsize{10pt}{12pt}\selectfont 4}\huxbpad{2pt}} &
\multicolumn{1}{r!{\huxvb[190, 190, 190]{1}}}{\huxtpad{2pt}\raggedleft {\fontsize{10pt}{12pt}\selectfont 0}\huxbpad{2pt}} \tabularnewline[-0.5pt]


\hhline{>{\huxb{1}}|>{\huxb[255, 255, 255]{1}}->{\huxb{1}}|>{\huxb{1}}->{\huxb{1}}->{\huxb{1}}->{\huxb{1}}-}
\arrayrulecolor{black}

\multicolumn{1}{!{\huxvb{1}}l!{\huxvb[190, 190, 190]{1}}}{} &
\multicolumn{1}{l!{\huxvb[190, 190, 190]{1}}}{\huxtpad{2pt}\raggedright {\fontsize{10pt}{12pt}\selectfont Fiat 128}\huxbpad{2pt}} &
\multicolumn{1}{r!{\huxvb[190, 190, 190]{1}}}{\huxtpad{2pt}\raggedleft {\fontsize{10pt}{12pt}\selectfont 32.4}\huxbpad{2pt}} &
\multicolumn{1}{r!{\huxvb[190, 190, 190]{1}}}{\huxtpad{2pt}\raggedleft {\fontsize{10pt}{12pt}\selectfont 4}\huxbpad{2pt}} &
\multicolumn{1}{r!{\huxvb[190, 190, 190]{1}}}{\huxtpad{2pt}\raggedleft {\fontsize{10pt}{12pt}\selectfont 1}\huxbpad{2pt}} \tabularnewline[-0.5pt]


\hhline{>{\huxb{1}}|>{\huxb[255, 255, 255]{1}}->{\huxb{1}}|>{\huxb{1}}->{\huxb{1}}->{\huxb{1}}->{\huxb{1}}-}
\arrayrulecolor{black}

\multicolumn{1}{!{\huxvb{1}}l!{\huxvb[190, 190, 190]{1}}}{} &
\multicolumn{1}{l!{\huxvb[190, 190, 190]{1}}}{\huxtpad{2pt}\raggedright {\fontsize{10pt}{12pt}\selectfont Honda Civic}\huxbpad{2pt}} &
\multicolumn{1}{r!{\huxvb[190, 190, 190]{1}}}{\huxtpad{2pt}\raggedleft {\fontsize{10pt}{12pt}\selectfont 30.4}\huxbpad{2pt}} &
\multicolumn{1}{r!{\huxvb[190, 190, 190]{1}}}{\huxtpad{2pt}\raggedleft {\fontsize{10pt}{12pt}\selectfont 4}\huxbpad{2pt}} &
\multicolumn{1}{r!{\huxvb[190, 190, 190]{1}}}{\huxtpad{2pt}\raggedleft {\fontsize{10pt}{12pt}\selectfont 1}\huxbpad{2pt}} \tabularnewline[-0.5pt]


\hhline{>{\huxb{1}}|>{\huxb[255, 255, 255]{1}}->{\huxb{1}}|>{\huxb{1}}->{\huxb{1}}->{\huxb{1}}->{\huxb{1}}-}
\arrayrulecolor{black}

\multicolumn{1}{!{\huxvb{1}}l!{\huxvb[190, 190, 190]{1}}}{} &
\multicolumn{1}{l!{\huxvb[190, 190, 190]{1}}}{\huxtpad{2pt}\raggedright {\fontsize{10pt}{12pt}\selectfont Toyota Corolla}\huxbpad{2pt}} &
\multicolumn{1}{r!{\huxvb[190, 190, 190]{1}}}{\huxtpad{2pt}\raggedleft {\fontsize{10pt}{12pt}\selectfont 33.9}\huxbpad{2pt}} &
\multicolumn{1}{r!{\huxvb[190, 190, 190]{1}}}{\huxtpad{2pt}\raggedleft {\fontsize{10pt}{12pt}\selectfont 4}\huxbpad{2pt}} &
\multicolumn{1}{r!{\huxvb[190, 190, 190]{1}}}{\huxtpad{2pt}\raggedleft {\fontsize{10pt}{12pt}\selectfont 1}\huxbpad{2pt}} \tabularnewline[-0.5pt]


\hhline{>{\huxb{1}}|>{\huxb[255, 255, 255]{1}}->{\huxb{1}}|>{\huxb{1}}->{\huxb{1}}->{\huxb{1}}->{\huxb{1}}-}
\arrayrulecolor{black}

\multicolumn{1}{!{\huxvb{1}}l!{\huxvb[190, 190, 190]{1}}}{} &
\multicolumn{1}{l!{\huxvb[190, 190, 190]{1}}}{\huxtpad{2pt}\raggedright {\fontsize{10pt}{12pt}\selectfont Toyota Corona}\huxbpad{2pt}} &
\multicolumn{1}{r!{\huxvb[190, 190, 190]{1}}}{\huxtpad{2pt}\raggedleft {\fontsize{10pt}{12pt}\selectfont 21.5}\huxbpad{2pt}} &
\multicolumn{1}{r!{\huxvb[190, 190, 190]{1}}}{\huxtpad{2pt}\raggedleft {\fontsize{10pt}{12pt}\selectfont 4}\huxbpad{2pt}} &
\multicolumn{1}{r!{\huxvb[190, 190, 190]{1}}}{\huxtpad{2pt}\raggedleft {\fontsize{10pt}{12pt}\selectfont 0}\huxbpad{2pt}} \tabularnewline[-0.5pt]


\hhline{>{\huxb{1}}|>{\huxb[255, 255, 255]{1}}->{\huxb{1}}|>{\huxb{1}}->{\huxb{1}}->{\huxb{1}}->{\huxb{1}}-}
\arrayrulecolor{black}

\multicolumn{1}{!{\huxvb{1}}l!{\huxvb[190, 190, 190]{1}}}{} &
\multicolumn{1}{l!{\huxvb[190, 190, 190]{1}}}{\huxtpad{2pt}\raggedright {\fontsize{10pt}{12pt}\selectfont Fiat X1-9}\huxbpad{2pt}} &
\multicolumn{1}{r!{\huxvb[190, 190, 190]{1}}}{\huxtpad{2pt}\raggedleft {\fontsize{10pt}{12pt}\selectfont 27.3}\huxbpad{2pt}} &
\multicolumn{1}{r!{\huxvb[190, 190, 190]{1}}}{\huxtpad{2pt}\raggedleft {\fontsize{10pt}{12pt}\selectfont 4}\huxbpad{2pt}} &
\multicolumn{1}{r!{\huxvb[190, 190, 190]{1}}}{\huxtpad{2pt}\raggedleft {\fontsize{10pt}{12pt}\selectfont 1}\huxbpad{2pt}} \tabularnewline[-0.5pt]


\hhline{>{\huxb{1}}|>{\huxb[255, 255, 255]{1}}->{\huxb{1}}|>{\huxb{1}}->{\huxb{1}}->{\huxb{1}}->{\huxb{1}}-}
\arrayrulecolor{black}

\multicolumn{1}{!{\huxvb{1}}l!{\huxvb[190, 190, 190]{1}}}{} &
\multicolumn{1}{l!{\huxvb[190, 190, 190]{1}}}{\huxtpad{2pt}\raggedright {\fontsize{10pt}{12pt}\selectfont Porsche 914-2}\huxbpad{2pt}} &
\multicolumn{1}{r!{\huxvb[190, 190, 190]{1}}}{\huxtpad{2pt}\raggedleft {\fontsize{10pt}{12pt}\selectfont 26~~}\huxbpad{2pt}} &
\multicolumn{1}{r!{\huxvb[190, 190, 190]{1}}}{\huxtpad{2pt}\raggedleft {\fontsize{10pt}{12pt}\selectfont 4}\huxbpad{2pt}} &
\multicolumn{1}{r!{\huxvb[190, 190, 190]{1}}}{\huxtpad{2pt}\raggedleft {\fontsize{10pt}{12pt}\selectfont 1}\huxbpad{2pt}} \tabularnewline[-0.5pt]


\hhline{>{\huxb{1}}|>{\huxb[255, 255, 255]{1}}->{\huxb{1}}|>{\huxb{1}}->{\huxb{1}}->{\huxb{1}}->{\huxb{1}}-}
\arrayrulecolor{black}

\multicolumn{1}{!{\huxvb{1}}l!{\huxvb[190, 190, 190]{1}}}{} &
\multicolumn{1}{l!{\huxvb[190, 190, 190]{1}}}{\huxtpad{2pt}\raggedright {\fontsize{10pt}{12pt}\selectfont Lotus Europa}\huxbpad{2pt}} &
\multicolumn{1}{r!{\huxvb[190, 190, 190]{1}}}{\huxtpad{2pt}\raggedleft {\fontsize{10pt}{12pt}\selectfont 30.4}\huxbpad{2pt}} &
\multicolumn{1}{r!{\huxvb[190, 190, 190]{1}}}{\huxtpad{2pt}\raggedleft {\fontsize{10pt}{12pt}\selectfont 4}\huxbpad{2pt}} &
\multicolumn{1}{r!{\huxvb[190, 190, 190]{1}}}{\huxtpad{2pt}\raggedleft {\fontsize{10pt}{12pt}\selectfont 1}\huxbpad{2pt}} \tabularnewline[-0.5pt]


\hhline{>{\huxb{1}}|>{\huxb[255, 255, 255]{1}}->{\huxb{1}}|>{\huxb{1}}->{\huxb{1}}->{\huxb{1}}->{\huxb{1}}-}
\arrayrulecolor{black}

\multicolumn{1}{!{\huxvb{1}}l!{\huxvb[190, 190, 190]{1}}}{\multirow{-11}{*}[0ex]{\huxtpad{2pt}\raggedright \textbf{{\fontsize{10pt}{12pt}\selectfont Four cylinders}}\huxbpad{2pt}}} &
\multicolumn{1}{l!{\huxvb[190, 190, 190]{1}}}{\huxtpad{2pt}\raggedright {\fontsize{10pt}{12pt}\selectfont Volvo 142E}\huxbpad{2pt}} &
\multicolumn{1}{r!{\huxvb[190, 190, 190]{1}}}{\huxtpad{2pt}\raggedleft {\fontsize{10pt}{12pt}\selectfont 21.4}\huxbpad{2pt}} &
\multicolumn{1}{r!{\huxvb[190, 190, 190]{1}}}{\huxtpad{2pt}\raggedleft {\fontsize{10pt}{12pt}\selectfont 4}\huxbpad{2pt}} &
\multicolumn{1}{r!{\huxvb[190, 190, 190]{1}}}{\huxtpad{2pt}\raggedleft {\fontsize{10pt}{12pt}\selectfont 1}\huxbpad{2pt}} \tabularnewline[-0.5pt]


\hhline{>{\huxb{1}}->{\huxb{1}}->{\huxb{1}}->{\huxb{1}}->{\huxb{1}}-}
\arrayrulecolor{black}

\multicolumn{1}{!{\huxvb{1}}l!{\huxvb[190, 190, 190]{1}}}{} &
\multicolumn{1}{l!{\huxvb[190, 190, 190]{1}}}{\huxtpad{2pt}\raggedright {\fontsize{10pt}{12pt}\selectfont Mazda RX4}\huxbpad{2pt}} &
\multicolumn{1}{r!{\huxvb[190, 190, 190]{1}}}{\huxtpad{2pt}\raggedleft {\fontsize{10pt}{12pt}\selectfont 21~~}\huxbpad{2pt}} &
\multicolumn{1}{r!{\huxvb[190, 190, 190]{1}}}{\huxtpad{2pt}\raggedleft {\fontsize{10pt}{12pt}\selectfont 6}\huxbpad{2pt}} &
\multicolumn{1}{r!{\huxvb[190, 190, 190]{1}}}{\huxtpad{2pt}\raggedleft {\fontsize{10pt}{12pt}\selectfont 1}\huxbpad{2pt}} \tabularnewline[-0.5pt]


\hhline{>{\huxb{1}}|>{\huxb[255, 255, 255]{1}}->{\huxb{1}}|>{\huxb{1}}->{\huxb{1}}->{\huxb{1}}->{\huxb{1}}-}
\arrayrulecolor{black}

\multicolumn{1}{!{\huxvb{1}}l!{\huxvb[190, 190, 190]{1}}}{} &
\multicolumn{1}{l!{\huxvb[190, 190, 190]{1}}}{\huxtpad{2pt}\raggedright {\fontsize{10pt}{12pt}\selectfont Mazda RX4 Wag}\huxbpad{2pt}} &
\multicolumn{1}{r!{\huxvb[190, 190, 190]{1}}}{\huxtpad{2pt}\raggedleft {\fontsize{10pt}{12pt}\selectfont 21~~}\huxbpad{2pt}} &
\multicolumn{1}{r!{\huxvb[190, 190, 190]{1}}}{\huxtpad{2pt}\raggedleft {\fontsize{10pt}{12pt}\selectfont 6}\huxbpad{2pt}} &
\multicolumn{1}{r!{\huxvb[190, 190, 190]{1}}}{\huxtpad{2pt}\raggedleft {\fontsize{10pt}{12pt}\selectfont 1}\huxbpad{2pt}} \tabularnewline[-0.5pt]


\hhline{>{\huxb{1}}|>{\huxb[255, 255, 255]{1}}->{\huxb{1}}|>{\huxb{1}}->{\huxb{1}}->{\huxb{1}}->{\huxb{1}}-}
\arrayrulecolor{black}

\multicolumn{1}{!{\huxvb{1}}l!{\huxvb[190, 190, 190]{1}}}{} &
\multicolumn{1}{l!{\huxvb[190, 190, 190]{1}}}{\huxtpad{2pt}\raggedright {\fontsize{10pt}{12pt}\selectfont Hornet 4 Drive}\huxbpad{2pt}} &
\multicolumn{1}{r!{\huxvb[190, 190, 190]{1}}}{\huxtpad{2pt}\raggedleft {\fontsize{10pt}{12pt}\selectfont 21.4}\huxbpad{2pt}} &
\multicolumn{1}{r!{\huxvb[190, 190, 190]{1}}}{\huxtpad{2pt}\raggedleft {\fontsize{10pt}{12pt}\selectfont 6}\huxbpad{2pt}} &
\multicolumn{1}{r!{\huxvb[190, 190, 190]{1}}}{\huxtpad{2pt}\raggedleft {\fontsize{10pt}{12pt}\selectfont 0}\huxbpad{2pt}} \tabularnewline[-0.5pt]


\hhline{>{\huxb{1}}|>{\huxb[255, 255, 255]{1}}->{\huxb{1}}|>{\huxb{1}}->{\huxb{1}}->{\huxb{1}}->{\huxb{1}}-}
\arrayrulecolor{black}

\multicolumn{1}{!{\huxvb{1}}l!{\huxvb[190, 190, 190]{1}}}{} &
\multicolumn{1}{l!{\huxvb[190, 190, 190]{1}}}{\huxtpad{2pt}\raggedright {\fontsize{10pt}{12pt}\selectfont Valiant}\huxbpad{2pt}} &
\multicolumn{1}{r!{\huxvb[190, 190, 190]{1}}}{\huxtpad{2pt}\raggedleft {\fontsize{10pt}{12pt}\selectfont 18.1}\huxbpad{2pt}} &
\multicolumn{1}{r!{\huxvb[190, 190, 190]{1}}}{\huxtpad{2pt}\raggedleft {\fontsize{10pt}{12pt}\selectfont 6}\huxbpad{2pt}} &
\multicolumn{1}{r!{\huxvb[190, 190, 190]{1}}}{\huxtpad{2pt}\raggedleft {\fontsize{10pt}{12pt}\selectfont 0}\huxbpad{2pt}} \tabularnewline[-0.5pt]


\hhline{>{\huxb{1}}|>{\huxb[255, 255, 255]{1}}->{\huxb{1}}|>{\huxb{1}}->{\huxb{1}}->{\huxb{1}}->{\huxb{1}}-}
\arrayrulecolor{black}

\multicolumn{1}{!{\huxvb{1}}l!{\huxvb[190, 190, 190]{1}}}{} &
\multicolumn{1}{l!{\huxvb[190, 190, 190]{1}}}{\huxtpad{2pt}\raggedright {\fontsize{10pt}{12pt}\selectfont Merc 280}\huxbpad{2pt}} &
\multicolumn{1}{r!{\huxvb[190, 190, 190]{1}}}{\huxtpad{2pt}\raggedleft {\fontsize{10pt}{12pt}\selectfont 19.2}\huxbpad{2pt}} &
\multicolumn{1}{r!{\huxvb[190, 190, 190]{1}}}{\huxtpad{2pt}\raggedleft {\fontsize{10pt}{12pt}\selectfont 6}\huxbpad{2pt}} &
\multicolumn{1}{r!{\huxvb[190, 190, 190]{1}}}{\huxtpad{2pt}\raggedleft {\fontsize{10pt}{12pt}\selectfont 0}\huxbpad{2pt}} \tabularnewline[-0.5pt]


\hhline{>{\huxb{1}}|>{\huxb[255, 255, 255]{1}}->{\huxb{1}}|>{\huxb{1}}->{\huxb{1}}->{\huxb{1}}->{\huxb{1}}-}
\arrayrulecolor{black}

\multicolumn{1}{!{\huxvb{1}}l!{\huxvb[190, 190, 190]{1}}}{} &
\multicolumn{1}{l!{\huxvb[190, 190, 190]{1}}}{\huxtpad{2pt}\raggedright {\fontsize{10pt}{12pt}\selectfont Merc 280C}\huxbpad{2pt}} &
\multicolumn{1}{r!{\huxvb[190, 190, 190]{1}}}{\huxtpad{2pt}\raggedleft {\fontsize{10pt}{12pt}\selectfont 17.8}\huxbpad{2pt}} &
\multicolumn{1}{r!{\huxvb[190, 190, 190]{1}}}{\huxtpad{2pt}\raggedleft {\fontsize{10pt}{12pt}\selectfont 6}\huxbpad{2pt}} &
\multicolumn{1}{r!{\huxvb[190, 190, 190]{1}}}{\huxtpad{2pt}\raggedleft {\fontsize{10pt}{12pt}\selectfont 0}\huxbpad{2pt}} \tabularnewline[-0.5pt]


\hhline{>{\huxb{1}}|>{\huxb[255, 255, 255]{1}}->{\huxb{1}}|>{\huxb{1}}->{\huxb{1}}->{\huxb{1}}->{\huxb{1}}-}
\arrayrulecolor{black}

\multicolumn{1}{!{\huxvb{1}}l!{\huxvb[190, 190, 190]{1}}}{\multirow{-7}{*}[0ex]{\huxtpad{2pt}\raggedright \textbf{{\fontsize{10pt}{12pt}\selectfont Six cylinders}}\huxbpad{2pt}}} &
\multicolumn{1}{l!{\huxvb[190, 190, 190]{1}}}{\huxtpad{2pt}\raggedright {\fontsize{10pt}{12pt}\selectfont Ferrari Dino}\huxbpad{2pt}} &
\multicolumn{1}{r!{\huxvb[190, 190, 190]{1}}}{\huxtpad{2pt}\raggedleft {\fontsize{10pt}{12pt}\selectfont 19.7}\huxbpad{2pt}} &
\multicolumn{1}{r!{\huxvb[190, 190, 190]{1}}}{\huxtpad{2pt}\raggedleft {\fontsize{10pt}{12pt}\selectfont 6}\huxbpad{2pt}} &
\multicolumn{1}{r!{\huxvb[190, 190, 190]{1}}}{\huxtpad{2pt}\raggedleft {\fontsize{10pt}{12pt}\selectfont 1}\huxbpad{2pt}} \tabularnewline[-0.5pt]


\hhline{>{\huxb{1}}->{\huxb{1}}->{\huxb{1}}->{\huxb{1}}->{\huxb{1}}-}
\arrayrulecolor{black}

\multicolumn{1}{!{\huxvb{1}}l!{\huxvb[190, 190, 190]{1}}}{} &
\multicolumn{1}{l!{\huxvb[190, 190, 190]{1}}}{\huxtpad{2pt}\raggedright {\fontsize{10pt}{12pt}\selectfont Hornet Sportabout}\huxbpad{2pt}} &
\multicolumn{1}{r!{\huxvb[190, 190, 190]{1}}}{\huxtpad{2pt}\raggedleft {\fontsize{10pt}{12pt}\selectfont 18.7}\huxbpad{2pt}} &
\multicolumn{1}{r!{\huxvb[190, 190, 190]{1}}}{\huxtpad{2pt}\raggedleft {\fontsize{10pt}{12pt}\selectfont 8}\huxbpad{2pt}} &
\multicolumn{1}{r!{\huxvb[190, 190, 190]{1}}}{\huxtpad{2pt}\raggedleft {\fontsize{10pt}{12pt}\selectfont 0}\huxbpad{2pt}} \tabularnewline[-0.5pt]


\hhline{>{\huxb{1}}|>{\huxb[255, 255, 255]{1}}->{\huxb{1}}|>{\huxb{1}}->{\huxb{1}}->{\huxb{1}}->{\huxb{1}}-}
\arrayrulecolor{black}

\multicolumn{1}{!{\huxvb{1}}l!{\huxvb[190, 190, 190]{1}}}{} &
\multicolumn{1}{l!{\huxvb[190, 190, 190]{1}}}{\huxtpad{2pt}\raggedright {\fontsize{10pt}{12pt}\selectfont Duster 360}\huxbpad{2pt}} &
\multicolumn{1}{r!{\huxvb[190, 190, 190]{1}}}{\huxtpad{2pt}\raggedleft {\fontsize{10pt}{12pt}\selectfont 14.3}\huxbpad{2pt}} &
\multicolumn{1}{r!{\huxvb[190, 190, 190]{1}}}{\huxtpad{2pt}\raggedleft {\fontsize{10pt}{12pt}\selectfont 8}\huxbpad{2pt}} &
\multicolumn{1}{r!{\huxvb[190, 190, 190]{1}}}{\huxtpad{2pt}\raggedleft {\fontsize{10pt}{12pt}\selectfont 0}\huxbpad{2pt}} \tabularnewline[-0.5pt]


\hhline{>{\huxb{1}}|>{\huxb[255, 255, 255]{1}}->{\huxb{1}}|>{\huxb{1}}->{\huxb{1}}->{\huxb{1}}->{\huxb{1}}-}
\arrayrulecolor{black}

\multicolumn{1}{!{\huxvb{1}}l!{\huxvb[190, 190, 190]{1}}}{} &
\multicolumn{1}{l!{\huxvb[190, 190, 190]{1}}}{\huxtpad{2pt}\raggedright {\fontsize{10pt}{12pt}\selectfont Merc 450SE}\huxbpad{2pt}} &
\multicolumn{1}{r!{\huxvb[190, 190, 190]{1}}}{\huxtpad{2pt}\raggedleft {\fontsize{10pt}{12pt}\selectfont 16.4}\huxbpad{2pt}} &
\multicolumn{1}{r!{\huxvb[190, 190, 190]{1}}}{\huxtpad{2pt}\raggedleft {\fontsize{10pt}{12pt}\selectfont 8}\huxbpad{2pt}} &
\multicolumn{1}{r!{\huxvb[190, 190, 190]{1}}}{\huxtpad{2pt}\raggedleft {\fontsize{10pt}{12pt}\selectfont 0}\huxbpad{2pt}} \tabularnewline[-0.5pt]


\hhline{>{\huxb{1}}|>{\huxb[255, 255, 255]{1}}->{\huxb{1}}|>{\huxb{1}}->{\huxb{1}}->{\huxb{1}}->{\huxb{1}}-}
\arrayrulecolor{black}

\multicolumn{1}{!{\huxvb{1}}l!{\huxvb[190, 190, 190]{1}}}{} &
\multicolumn{1}{l!{\huxvb[190, 190, 190]{1}}}{\huxtpad{2pt}\raggedright {\fontsize{10pt}{12pt}\selectfont Merc 450SL}\huxbpad{2pt}} &
\multicolumn{1}{r!{\huxvb[190, 190, 190]{1}}}{\huxtpad{2pt}\raggedleft {\fontsize{10pt}{12pt}\selectfont 17.3}\huxbpad{2pt}} &
\multicolumn{1}{r!{\huxvb[190, 190, 190]{1}}}{\huxtpad{2pt}\raggedleft {\fontsize{10pt}{12pt}\selectfont 8}\huxbpad{2pt}} &
\multicolumn{1}{r!{\huxvb[190, 190, 190]{1}}}{\huxtpad{2pt}\raggedleft {\fontsize{10pt}{12pt}\selectfont 0}\huxbpad{2pt}} \tabularnewline[-0.5pt]


\hhline{>{\huxb{1}}|>{\huxb[255, 255, 255]{1}}->{\huxb{1}}|>{\huxb{1}}->{\huxb{1}}->{\huxb{1}}->{\huxb{1}}-}
\arrayrulecolor{black}

\multicolumn{1}{!{\huxvb{1}}l!{\huxvb[190, 190, 190]{1}}}{} &
\multicolumn{1}{l!{\huxvb[190, 190, 190]{1}}}{\huxtpad{2pt}\raggedright {\fontsize{10pt}{12pt}\selectfont Merc 450SLC}\huxbpad{2pt}} &
\multicolumn{1}{r!{\huxvb[190, 190, 190]{1}}}{\huxtpad{2pt}\raggedleft {\fontsize{10pt}{12pt}\selectfont 15.2}\huxbpad{2pt}} &
\multicolumn{1}{r!{\huxvb[190, 190, 190]{1}}}{\huxtpad{2pt}\raggedleft {\fontsize{10pt}{12pt}\selectfont 8}\huxbpad{2pt}} &
\multicolumn{1}{r!{\huxvb[190, 190, 190]{1}}}{\huxtpad{2pt}\raggedleft {\fontsize{10pt}{12pt}\selectfont 0}\huxbpad{2pt}} \tabularnewline[-0.5pt]


\hhline{>{\huxb{1}}|>{\huxb[255, 255, 255]{1}}->{\huxb{1}}|>{\huxb{1}}->{\huxb{1}}->{\huxb{1}}->{\huxb{1}}-}
\arrayrulecolor{black}

\multicolumn{1}{!{\huxvb{1}}l!{\huxvb[190, 190, 190]{1}}}{} &
\multicolumn{1}{l!{\huxvb[190, 190, 190]{1}}}{\huxtpad{2pt}\raggedright {\fontsize{10pt}{12pt}\selectfont Cadillac Fleetwood}\huxbpad{2pt}} &
\multicolumn{1}{r!{\huxvb[190, 190, 190]{1}}}{\huxtpad{2pt}\raggedleft {\fontsize{10pt}{12pt}\selectfont 10.4}\huxbpad{2pt}} &
\multicolumn{1}{r!{\huxvb[190, 190, 190]{1}}}{\huxtpad{2pt}\raggedleft {\fontsize{10pt}{12pt}\selectfont 8}\huxbpad{2pt}} &
\multicolumn{1}{r!{\huxvb[190, 190, 190]{1}}}{\huxtpad{2pt}\raggedleft {\fontsize{10pt}{12pt}\selectfont 0}\huxbpad{2pt}} \tabularnewline[-0.5pt]


\hhline{>{\huxb{1}}|>{\huxb[255, 255, 255]{1}}->{\huxb{1}}|>{\huxb{1}}->{\huxb{1}}->{\huxb{1}}->{\huxb{1}}-}
\arrayrulecolor{black}

\multicolumn{1}{!{\huxvb{1}}l!{\huxvb[190, 190, 190]{1}}}{} &
\multicolumn{1}{l!{\huxvb[190, 190, 190]{1}}}{\huxtpad{2pt}\raggedright {\fontsize{10pt}{12pt}\selectfont Lincoln Continental}\huxbpad{2pt}} &
\multicolumn{1}{r!{\huxvb[190, 190, 190]{1}}}{\huxtpad{2pt}\raggedleft {\fontsize{10pt}{12pt}\selectfont 10.4}\huxbpad{2pt}} &
\multicolumn{1}{r!{\huxvb[190, 190, 190]{1}}}{\huxtpad{2pt}\raggedleft {\fontsize{10pt}{12pt}\selectfont 8}\huxbpad{2pt}} &
\multicolumn{1}{r!{\huxvb[190, 190, 190]{1}}}{\huxtpad{2pt}\raggedleft {\fontsize{10pt}{12pt}\selectfont 0}\huxbpad{2pt}} \tabularnewline[-0.5pt]


\hhline{>{\huxb{1}}|>{\huxb[255, 255, 255]{1}}->{\huxb{1}}|>{\huxb{1}}->{\huxb{1}}->{\huxb{1}}->{\huxb{1}}-}
\arrayrulecolor{black}

\multicolumn{1}{!{\huxvb{1}}l!{\huxvb[190, 190, 190]{1}}}{} &
\multicolumn{1}{l!{\huxvb[190, 190, 190]{1}}}{\huxtpad{2pt}\raggedright {\fontsize{10pt}{12pt}\selectfont Chrysler Imperial}\huxbpad{2pt}} &
\multicolumn{1}{r!{\huxvb[190, 190, 190]{1}}}{\huxtpad{2pt}\raggedleft {\fontsize{10pt}{12pt}\selectfont 14.7}\huxbpad{2pt}} &
\multicolumn{1}{r!{\huxvb[190, 190, 190]{1}}}{\huxtpad{2pt}\raggedleft {\fontsize{10pt}{12pt}\selectfont 8}\huxbpad{2pt}} &
\multicolumn{1}{r!{\huxvb[190, 190, 190]{1}}}{\huxtpad{2pt}\raggedleft {\fontsize{10pt}{12pt}\selectfont 0}\huxbpad{2pt}} \tabularnewline[-0.5pt]


\hhline{>{\huxb{1}}|>{\huxb[255, 255, 255]{1}}->{\huxb{1}}|>{\huxb{1}}->{\huxb{1}}->{\huxb{1}}->{\huxb{1}}-}
\arrayrulecolor{black}

\multicolumn{1}{!{\huxvb{1}}l!{\huxvb[190, 190, 190]{1}}}{} &
\multicolumn{1}{l!{\huxvb[190, 190, 190]{1}}}{\huxtpad{2pt}\raggedright {\fontsize{10pt}{12pt}\selectfont Dodge Challenger}\huxbpad{2pt}} &
\multicolumn{1}{r!{\huxvb[190, 190, 190]{1}}}{\huxtpad{2pt}\raggedleft {\fontsize{10pt}{12pt}\selectfont 15.5}\huxbpad{2pt}} &
\multicolumn{1}{r!{\huxvb[190, 190, 190]{1}}}{\huxtpad{2pt}\raggedleft {\fontsize{10pt}{12pt}\selectfont 8}\huxbpad{2pt}} &
\multicolumn{1}{r!{\huxvb[190, 190, 190]{1}}}{\huxtpad{2pt}\raggedleft {\fontsize{10pt}{12pt}\selectfont 0}\huxbpad{2pt}} \tabularnewline[-0.5pt]


\hhline{>{\huxb{1}}|>{\huxb[255, 255, 255]{1}}->{\huxb{1}}|>{\huxb{1}}->{\huxb{1}}->{\huxb{1}}->{\huxb{1}}-}
\arrayrulecolor{black}

\multicolumn{1}{!{\huxvb{1}}l!{\huxvb[190, 190, 190]{1}}}{} &
\multicolumn{1}{l!{\huxvb[190, 190, 190]{1}}}{\huxtpad{2pt}\raggedright {\fontsize{10pt}{12pt}\selectfont AMC Javelin}\huxbpad{2pt}} &
\multicolumn{1}{r!{\huxvb[190, 190, 190]{1}}}{\huxtpad{2pt}\raggedleft {\fontsize{10pt}{12pt}\selectfont 15.2}\huxbpad{2pt}} &
\multicolumn{1}{r!{\huxvb[190, 190, 190]{1}}}{\huxtpad{2pt}\raggedleft {\fontsize{10pt}{12pt}\selectfont 8}\huxbpad{2pt}} &
\multicolumn{1}{r!{\huxvb[190, 190, 190]{1}}}{\huxtpad{2pt}\raggedleft {\fontsize{10pt}{12pt}\selectfont 0}\huxbpad{2pt}} \tabularnewline[-0.5pt]


\hhline{>{\huxb{1}}|>{\huxb[255, 255, 255]{1}}->{\huxb{1}}|>{\huxb{1}}->{\huxb{1}}->{\huxb{1}}->{\huxb{1}}-}
\arrayrulecolor{black}

\multicolumn{1}{!{\huxvb{1}}l!{\huxvb[190, 190, 190]{1}}}{} &
\multicolumn{1}{l!{\huxvb[190, 190, 190]{1}}}{\huxtpad{2pt}\raggedright {\fontsize{10pt}{12pt}\selectfont Camaro Z28}\huxbpad{2pt}} &
\multicolumn{1}{r!{\huxvb[190, 190, 190]{1}}}{\huxtpad{2pt}\raggedleft {\fontsize{10pt}{12pt}\selectfont 13.3}\huxbpad{2pt}} &
\multicolumn{1}{r!{\huxvb[190, 190, 190]{1}}}{\huxtpad{2pt}\raggedleft {\fontsize{10pt}{12pt}\selectfont 8}\huxbpad{2pt}} &
\multicolumn{1}{r!{\huxvb[190, 190, 190]{1}}}{\huxtpad{2pt}\raggedleft {\fontsize{10pt}{12pt}\selectfont 0}\huxbpad{2pt}} \tabularnewline[-0.5pt]


\hhline{>{\huxb{1}}|>{\huxb[255, 255, 255]{1}}->{\huxb{1}}|>{\huxb{1}}->{\huxb{1}}->{\huxb{1}}->{\huxb{1}}-}
\arrayrulecolor{black}

\multicolumn{1}{!{\huxvb{1}}l!{\huxvb[190, 190, 190]{1}}}{} &
\multicolumn{1}{l!{\huxvb[190, 190, 190]{1}}}{\huxtpad{2pt}\raggedright {\fontsize{10pt}{12pt}\selectfont Pontiac Firebird}\huxbpad{2pt}} &
\multicolumn{1}{r!{\huxvb[190, 190, 190]{1}}}{\huxtpad{2pt}\raggedleft {\fontsize{10pt}{12pt}\selectfont 19.2}\huxbpad{2pt}} &
\multicolumn{1}{r!{\huxvb[190, 190, 190]{1}}}{\huxtpad{2pt}\raggedleft {\fontsize{10pt}{12pt}\selectfont 8}\huxbpad{2pt}} &
\multicolumn{1}{r!{\huxvb[190, 190, 190]{1}}}{\huxtpad{2pt}\raggedleft {\fontsize{10pt}{12pt}\selectfont 0}\huxbpad{2pt}} \tabularnewline[-0.5pt]


\hhline{>{\huxb{1}}|>{\huxb[255, 255, 255]{1}}->{\huxb{1}}|>{\huxb{1}}->{\huxb{1}}->{\huxb{1}}->{\huxb{1}}-}
\arrayrulecolor{black}

\multicolumn{1}{!{\huxvb{1}}l!{\huxvb[190, 190, 190]{1}}}{} &
\multicolumn{1}{l!{\huxvb[190, 190, 190]{1}}}{\huxtpad{2pt}\raggedright {\fontsize{10pt}{12pt}\selectfont Ford Pantera L}\huxbpad{2pt}} &
\multicolumn{1}{r!{\huxvb[190, 190, 190]{1}}}{\huxtpad{2pt}\raggedleft {\fontsize{10pt}{12pt}\selectfont 15.8}\huxbpad{2pt}} &
\multicolumn{1}{r!{\huxvb[190, 190, 190]{1}}}{\huxtpad{2pt}\raggedleft {\fontsize{10pt}{12pt}\selectfont 8}\huxbpad{2pt}} &
\multicolumn{1}{r!{\huxvb[190, 190, 190]{1}}}{\huxtpad{2pt}\raggedleft {\fontsize{10pt}{12pt}\selectfont 1}\huxbpad{2pt}} \tabularnewline[-0.5pt]


\hhline{>{\huxb{1}}|>{\huxb[255, 255, 255]{1}}->{\huxb{1}}|>{\huxb{1}}->{\huxb{1}}->{\huxb{1}}->{\huxb{1}}-}
\arrayrulecolor{black}

\multicolumn{1}{!{\huxvb{1}}l!{\huxvb[190, 190, 190]{1}}}{\multirow{-14}{*}[0ex]{\huxtpad{2pt}\raggedright \textbf{{\fontsize{10pt}{12pt}\selectfont Eight cylinders}}\huxbpad{2pt}}} &
\multicolumn{1}{l!{\huxvb[190, 190, 190]{1}}}{\huxtpad{2pt}\raggedright {\fontsize{10pt}{12pt}\selectfont Maserati Bora}\huxbpad{2pt}} &
\multicolumn{1}{r!{\huxvb[190, 190, 190]{1}}}{\huxtpad{2pt}\raggedleft {\fontsize{10pt}{12pt}\selectfont 15~~}\huxbpad{2pt}} &
\multicolumn{1}{r!{\huxvb[190, 190, 190]{1}}}{\huxtpad{2pt}\raggedleft {\fontsize{10pt}{12pt}\selectfont 8}\huxbpad{2pt}} &
\multicolumn{1}{r!{\huxvb[190, 190, 190]{1}}}{\huxtpad{2pt}\raggedleft {\fontsize{10pt}{12pt}\selectfont 1}\huxbpad{2pt}} \tabularnewline[-0.5pt]


\hhline{>{\huxb{1}}->{\huxb{1}}->{\huxb{1}}->{\huxb{1}}->{\huxb{1}}-}
\arrayrulecolor{black}
\end{tabularx}
\end{table}
 

\FloatBarrier

\texttt{merge\_cells} works by setting the top left cell's
\texttt{colspan} and \texttt{rowspan} properties. If you know HTML
tables, then these will be familiar to you. \texttt{colspan} sets how
many columns the cell covers, and \texttt{rowspan} sets how many rows
the cell covers. If you prefer, you can set these directly:

\begin{Shaded}
\begin{Highlighting}[]
\KeywordTok{colspan}\NormalTok{(cars_mpg)[}\DecValTok{1}\NormalTok{, }\DecValTok{1}\NormalTok{] <-}\StringTok{ }\DecValTok{5}
\end{Highlighting}
\end{Shaded}

\FloatBarrier

\hypertarget{quick-themes}{%
\subsection{Quick themes}\label{quick-themes}}

Huxtable comes with some predefined themes for formatting.

\begin{Shaded}
\begin{Highlighting}[]
\KeywordTok{theme_plain}\NormalTok{(car_ht)}
\end{Highlighting}
\end{Shaded}

 \begin{table}[h]
\begin{raggedright}
    \providecommand{\huxb}[2][0,0,0]{\arrayrulecolor[RGB]{#1}\global\arrayrulewidth=#2pt}
    \providecommand{\huxvb}[2][0,0,0]{\color[RGB]{#1}\vrule width #2pt}
    \providecommand{\huxtpad}[1]{\rule{0pt}{\baselineskip+#1}}
    \providecommand{\huxbpad}[1]{\rule[-#1]{0pt}{#1}}
  \begin{tabularx}{0.6\textwidth}{p{0.21\textwidth} p{0.09\textwidth} p{0.09\textwidth} p{0.09\textwidth} p{0.12\textwidth}}


\hhline{>{\huxb{0.4}}->{\huxb{0.4}}->{\huxb{0.4}}->{\huxb{0.4}}->{\huxb{0.4}}-}
\arrayrulecolor{black}

\multicolumn{1}{!{\huxvb{0.4}}l!{\huxvb[190, 190, 190]{0.4}}}{\huxtpad{4pt}\raggedright \textbf{Car}\huxbpad{4pt}} &
\multicolumn{1}{l!{\huxvb[190, 190, 190]{0.4}}}{\huxtpad{4pt}\raggedright \textbf{MPG}\huxbpad{4pt}} &
\multicolumn{1}{l!{\huxvb[190, 190, 190]{0.4}}}{\huxtpad{4pt}\raggedright \textbf{Cylinders}\huxbpad{4pt}} &
\multicolumn{1}{l!{\huxvb[190, 190, 190]{0.4}}}{\huxtpad{4pt}\raggedright \textbf{Horsepower}\huxbpad{4pt}} &
\multicolumn{1}{l!{\huxvb[190, 190, 190]{0.4}}}{\huxtpad{4pt}\raggedright \textbf{kml}\huxbpad{4pt}} \tabularnewline[-0.5pt]


\hhline{>{\huxb{0.4}}->{\huxb{0.4}}->{\huxb{0.4}}->{\huxb{0.4}}->{\huxb{0.4}}-}
\arrayrulecolor{black}

\multicolumn{1}{!{\huxvb{0.4}}l!{\huxvb[190, 190, 190]{0.4}}}{\cellcolor[RGB]{242, 242, 242}\huxtpad{4pt}\raggedright Valiant\huxbpad{4pt}} &
\multicolumn{1}{r!{\huxvb[190, 190, 190]{0.4}}}{\cellcolor[RGB]{242, 242, 242}\huxtpad{4pt}\raggedleft 18.1\huxbpad{4pt}} &
\multicolumn{1}{r!{\huxvb[190, 190, 190]{0.4}}}{\cellcolor[RGB]{242, 242, 242}\huxtpad{4pt}\raggedleft 6\huxbpad{4pt}} &
\multicolumn{1}{r!{\huxvb[190, 190, 190]{0.4}}}{\cellcolor[RGB]{242, 242, 242}\huxtpad{4pt}\raggedleft 105\huxbpad{4pt}} &
\multicolumn{1}{r!{\huxvb[190, 190, 190]{0.4}}}{\cellcolor[RGB]{242, 242, 242}\huxtpad{4pt}\raggedleft 6.42\huxbpad{4pt}} \tabularnewline[-0.5pt]


\hhline{>{\huxb{0.4}}|>{\huxb[190, 190, 190]{0.4}}|>{\huxb[190, 190, 190]{0.4}}|>{\huxb[190, 190, 190]{0.4}}|>{\huxb[190, 190, 190]{0.4}}|>{\huxb[190, 190, 190]{0.4}}|}
\arrayrulecolor{black}

\multicolumn{1}{!{\huxvb{0.4}}l!{\huxvb[190, 190, 190]{0.4}}}{\huxtpad{4pt}\raggedright Mazda RX4\huxbpad{4pt}} &
\multicolumn{1}{r!{\huxvb[190, 190, 190]{0.4}}}{\huxtpad{4pt}\raggedleft 21~~\huxbpad{4pt}} &
\multicolumn{1}{r!{\huxvb[190, 190, 190]{0.4}}}{\huxtpad{4pt}\raggedleft 6\huxbpad{4pt}} &
\multicolumn{1}{r!{\huxvb[190, 190, 190]{0.4}}}{\huxtpad{4pt}\raggedleft 110\huxbpad{4pt}} &
\multicolumn{1}{r!{\huxvb[190, 190, 190]{0.4}}}{\huxtpad{4pt}\raggedleft 7.45\huxbpad{4pt}} \tabularnewline[-0.5pt]


\hhline{>{\huxb{0.4}}|>{\huxb[190, 190, 190]{0.4}}|>{\huxb[190, 190, 190]{0.4}}|>{\huxb[190, 190, 190]{0.4}}|>{\huxb[190, 190, 190]{0.4}}|>{\huxb[190, 190, 190]{0.4}}|}
\arrayrulecolor{black}

\multicolumn{1}{!{\huxvb{0.4}}l!{\huxvb[190, 190, 190]{0.4}}}{\cellcolor[RGB]{242, 242, 242}\huxtpad{4pt}\raggedright Mazda RX4 Wag\huxbpad{4pt}} &
\multicolumn{1}{r!{\huxvb[190, 190, 190]{0.4}}}{\cellcolor[RGB]{242, 242, 242}\huxtpad{4pt}\raggedleft 21~~\huxbpad{4pt}} &
\multicolumn{1}{r!{\huxvb[190, 190, 190]{0.4}}}{\cellcolor[RGB]{242, 242, 242}\huxtpad{4pt}\raggedleft 6\huxbpad{4pt}} &
\multicolumn{1}{r!{\huxvb[190, 190, 190]{0.4}}}{\cellcolor[RGB]{242, 242, 242}\huxtpad{4pt}\raggedleft 110\huxbpad{4pt}} &
\multicolumn{1}{r!{\huxvb[190, 190, 190]{0.4}}}{\cellcolor[RGB]{242, 242, 242}\huxtpad{4pt}\raggedleft 7.45\huxbpad{4pt}} \tabularnewline[-0.5pt]


\hhline{>{\huxb{0.4}}|>{\huxb[190, 190, 190]{0.4}}|>{\huxb[190, 190, 190]{0.4}}|>{\huxb[190, 190, 190]{0.4}}|>{\huxb[190, 190, 190]{0.4}}|>{\huxb[190, 190, 190]{0.4}}|}
\arrayrulecolor{black}

\multicolumn{1}{!{\huxvb{0.4}}l!{\huxvb[190, 190, 190]{0.4}}}{\huxtpad{4pt}\raggedright Hornet 4 Drive\huxbpad{4pt}} &
\multicolumn{1}{r!{\huxvb[190, 190, 190]{0.4}}}{\huxtpad{4pt}\raggedleft 21.4\huxbpad{4pt}} &
\multicolumn{1}{r!{\huxvb[190, 190, 190]{0.4}}}{\huxtpad{4pt}\raggedleft 6\huxbpad{4pt}} &
\multicolumn{1}{r!{\huxvb[190, 190, 190]{0.4}}}{\huxtpad{4pt}\raggedleft 110\huxbpad{4pt}} &
\multicolumn{1}{r!{\huxvb[190, 190, 190]{0.4}}}{\huxtpad{4pt}\raggedleft 7.59\huxbpad{4pt}} \tabularnewline[-0.5pt]


\hhline{>{\huxb{0.4}}|>{\huxb[190, 190, 190]{0.4}}|>{\huxb[190, 190, 190]{0.4}}|>{\huxb[190, 190, 190]{0.4}}|>{\huxb[190, 190, 190]{0.4}}|>{\huxb[190, 190, 190]{0.4}}|}
\arrayrulecolor{black}

\multicolumn{1}{!{\huxvb{0.4}}l!{\huxvb[190, 190, 190]{0.4}}}{\cellcolor[RGB]{242, 242, 242}\huxtpad{4pt}\raggedright Merc 280\huxbpad{4pt}} &
\multicolumn{1}{r!{\huxvb[190, 190, 190]{0.4}}}{\cellcolor[RGB]{242, 242, 242}\huxtpad{4pt}\raggedleft 19.2\huxbpad{4pt}} &
\multicolumn{1}{r!{\huxvb[190, 190, 190]{0.4}}}{\cellcolor[RGB]{242, 242, 242}\huxtpad{4pt}\raggedleft 6\huxbpad{4pt}} &
\multicolumn{1}{r!{\huxvb[190, 190, 190]{0.4}}}{\cellcolor[RGB]{242, 242, 242}\huxtpad{4pt}\raggedleft 123\huxbpad{4pt}} &
\multicolumn{1}{r!{\huxvb[190, 190, 190]{0.4}}}{\cellcolor[RGB]{242, 242, 242}\huxtpad{4pt}\raggedleft 6.81\huxbpad{4pt}} \tabularnewline[-0.5pt]


\hhline{>{\huxb{0.4}}|>{\huxb[190, 190, 190]{0.4}}|>{\huxb[190, 190, 190]{0.4}}|>{\huxb[190, 190, 190]{0.4}}|>{\huxb[190, 190, 190]{0.4}}|>{\huxb[190, 190, 190]{0.4}}|}
\arrayrulecolor{black}

\multicolumn{1}{!{\huxvb{0.4}}l!{\huxvb[190, 190, 190]{0.4}}}{\huxtpad{4pt}\raggedright Hornet Sportabout\huxbpad{4pt}} &
\multicolumn{1}{r!{\huxvb[190, 190, 190]{0.4}}}{\huxtpad{4pt}\raggedleft 18.7\huxbpad{4pt}} &
\multicolumn{1}{r!{\huxvb[190, 190, 190]{0.4}}}{\huxtpad{4pt}\raggedleft 8\huxbpad{4pt}} &
\multicolumn{1}{r!{\huxvb[190, 190, 190]{0.4}}}{\huxtpad{4pt}\raggedleft 175\huxbpad{4pt}} &
\multicolumn{1}{r!{\huxvb[190, 190, 190]{0.4}}}{\huxtpad{4pt}\raggedleft 6.63\huxbpad{4pt}} \tabularnewline[-0.5pt]


\hhline{>{\huxb{0.4}}|>{\huxb[190, 190, 190]{0.4}}|>{\huxb[190, 190, 190]{0.4}}|>{\huxb[190, 190, 190]{0.4}}|>{\huxb[190, 190, 190]{0.4}}|>{\huxb[190, 190, 190]{0.4}}|}
\arrayrulecolor{black}

\multicolumn{1}{!{\huxvb{0.4}}l!{\huxvb[190, 190, 190]{0.4}}}{\cellcolor[RGB]{242, 242, 242}\huxtpad{4pt}\raggedright Duster 360\huxbpad{4pt}} &
\multicolumn{1}{r!{\huxvb[190, 190, 190]{0.4}}}{\cellcolor[RGB]{242, 242, 242}\huxtpad{4pt}\raggedleft 14.3\huxbpad{4pt}} &
\multicolumn{1}{r!{\huxvb[190, 190, 190]{0.4}}}{\cellcolor[RGB]{242, 242, 242}\huxtpad{4pt}\raggedleft 8\huxbpad{4pt}} &
\multicolumn{1}{r!{\huxvb[190, 190, 190]{0.4}}}{\cellcolor[RGB]{242, 242, 242}\huxtpad{4pt}\raggedleft 245\huxbpad{4pt}} &
\multicolumn{1}{r!{\huxvb[190, 190, 190]{0.4}}}{\cellcolor[RGB]{242, 242, 242}\huxtpad{4pt}\raggedleft 5.07\huxbpad{4pt}} \tabularnewline[-0.5pt]


\hhline{>{\huxb{0.4}}->{\huxb{0.4}}->{\huxb{0.4}}->{\huxb{0.4}}->{\huxb{0.4}}-}
\arrayrulecolor{black}
\end{tabularx}\par\end{raggedright}
\end{table}
 

\FloatBarrier

\hypertarget{selecting-rows-columns-and-cells}{%
\section{Selecting rows, columns and
cells}\label{selecting-rows-columns-and-cells}}

\hypertarget{row-and-column-functions}{%
\subsection{Row and column functions}\label{row-and-column-functions}}

If you use the \texttt{set\_*} style functions, huxtable has some
convenience functions for selecting rows and columns.

To select all rows, or all columns, use \texttt{everywhere} in the row
or column specification. To select just even or odd-numbered rows or
columns, use \texttt{evens} or \texttt{odds}. To select the last
\texttt{n} rows or columns, use \texttt{final(n)}. To select every
\emph{n}th row, use \texttt{every(n)} and to do this starting from row
\emph{m} use \texttt{every(n,\ from\ =\ m)}.

With these functions it is easy to add striped backgrounds to tables:

\begin{Shaded}
\begin{Highlighting}[]
\NormalTok{car_ht                                                 }\OperatorTok\StringTok{ }
\StringTok{      }\KeywordTok{set_background_color}\NormalTok{(evens, everywhere, }\StringTok{"wheat"}\NormalTok{) }\OperatorTok\StringTok{ }
\StringTok{      }\KeywordTok{set_background_color}\NormalTok{(odds, everywhere, }\KeywordTok{grey}\NormalTok{(.}\DecValTok{9}\NormalTok{)) }\OperatorTok\StringTok{ }
\StringTok{      }\KeywordTok{set_bold}\NormalTok{(}\DecValTok{1}\NormalTok{, everywhere, }\OtherTok{TRUE}\NormalTok{)}
\end{Highlighting}
\end{Shaded}

 \begin{table}[h]
\centering
    \providecommand{\huxb}[2][0,0,0]{\arrayrulecolor[RGB]{#1}\global\arrayrulewidth=#2pt}
    \providecommand{\huxvb}[2][0,0,0]{\color[RGB]{#1}\vrule width #2pt}
    \providecommand{\huxtpad}[1]{\rule{0pt}{\baselineskip+#1}}
    \providecommand{\huxbpad}[1]{\rule[-#1]{0pt}{#1}}
  \begin{tabularx}{0.6\textwidth}{p{0.21\textwidth} p{0.09\textwidth} p{0.09\textwidth} p{0.09\textwidth} p{0.12\textwidth}}


\hhline{>{\huxb[190, 190, 190]{0.4}}|>{\huxb[190, 190, 190]{0.4}}|>{\huxb[190, 190, 190]{0.4}}|>{\huxb[190, 190, 190]{0.4}}|>{\huxb[190, 190, 190]{0.4}}|}
\arrayrulecolor{black}

\multicolumn{1}{!{\huxvb{0}}l!{\huxvb[190, 190, 190]{0.4}}}{\cellcolor[RGB]{230, 230, 230}\huxtpad{4pt}\raggedright \textbf{Car}\huxbpad{4pt}} &
\multicolumn{1}{l!{\huxvb[190, 190, 190]{0.4}}}{\cellcolor[RGB]{230, 230, 230}\huxtpad{4pt}\raggedright \textbf{MPG}\huxbpad{4pt}} &
\multicolumn{1}{l!{\huxvb[190, 190, 190]{0.4}}}{\cellcolor[RGB]{230, 230, 230}\huxtpad{4pt}\raggedright \textbf{Cylinders}\huxbpad{4pt}} &
\multicolumn{1}{l!{\huxvb[190, 190, 190]{0.4}}}{\cellcolor[RGB]{230, 230, 230}\huxtpad{4pt}\raggedright \textbf{Horsepower}\huxbpad{4pt}} &
\multicolumn{1}{l!{\huxvb[190, 190, 190]{0.4}}}{\cellcolor[RGB]{230, 230, 230}\huxtpad{4pt}\raggedright \textbf{kml}\huxbpad{4pt}} \tabularnewline[-0.5pt]


\hhline{>{\huxb[190, 190, 190]{0.4}}|>{\huxb[190, 190, 190]{0.4}}|>{\huxb[190, 190, 190]{0.4}}|>{\huxb[190, 190, 190]{0.4}}|>{\huxb[190, 190, 190]{0.4}}|}
\arrayrulecolor{black}

\multicolumn{1}{!{\huxvb{0}}l!{\huxvb[190, 190, 190]{0.4}}}{\cellcolor[RGB]{245, 222, 179}\huxtpad{4pt}\raggedright Valiant\huxbpad{4pt}} &
\multicolumn{1}{r!{\huxvb[190, 190, 190]{0.4}}}{\cellcolor[RGB]{245, 222, 179}\huxtpad{4pt}\raggedleft 18.1\huxbpad{4pt}} &
\multicolumn{1}{r!{\huxvb[190, 190, 190]{0.4}}}{\cellcolor[RGB]{245, 222, 179}\huxtpad{4pt}\raggedleft 6\huxbpad{4pt}} &
\multicolumn{1}{r!{\huxvb[190, 190, 190]{0.4}}}{\cellcolor[RGB]{245, 222, 179}\huxtpad{4pt}\raggedleft 105\huxbpad{4pt}} &
\multicolumn{1}{r!{\huxvb[190, 190, 190]{0.4}}}{\cellcolor[RGB]{245, 222, 179}\huxtpad{4pt}\raggedleft 6.42\huxbpad{4pt}} \tabularnewline[-0.5pt]


\hhline{>{\huxb[190, 190, 190]{0.4}}|>{\huxb[190, 190, 190]{0.4}}|>{\huxb[190, 190, 190]{0.4}}|>{\huxb[190, 190, 190]{0.4}}|>{\huxb[190, 190, 190]{0.4}}|}
\arrayrulecolor{black}

\multicolumn{1}{!{\huxvb{0}}l!{\huxvb[190, 190, 190]{0.4}}}{\cellcolor[RGB]{230, 230, 230}\huxtpad{4pt}\raggedright Mazda RX4\huxbpad{4pt}} &
\multicolumn{1}{r!{\huxvb[190, 190, 190]{0.4}}}{\cellcolor[RGB]{230, 230, 230}\huxtpad{4pt}\raggedleft 21~~\huxbpad{4pt}} &
\multicolumn{1}{r!{\huxvb[190, 190, 190]{0.4}}}{\cellcolor[RGB]{230, 230, 230}\huxtpad{4pt}\raggedleft 6\huxbpad{4pt}} &
\multicolumn{1}{r!{\huxvb[190, 190, 190]{0.4}}}{\cellcolor[RGB]{230, 230, 230}\huxtpad{4pt}\raggedleft 110\huxbpad{4pt}} &
\multicolumn{1}{r!{\huxvb[190, 190, 190]{0.4}}}{\cellcolor[RGB]{230, 230, 230}\huxtpad{4pt}\raggedleft 7.45\huxbpad{4pt}} \tabularnewline[-0.5pt]


\hhline{>{\huxb[190, 190, 190]{0.4}}|>{\huxb[190, 190, 190]{0.4}}|>{\huxb[190, 190, 190]{0.4}}|>{\huxb[190, 190, 190]{0.4}}|>{\huxb[190, 190, 190]{0.4}}|}
\arrayrulecolor{black}

\multicolumn{1}{!{\huxvb{0}}l!{\huxvb[190, 190, 190]{0.4}}}{\cellcolor[RGB]{245, 222, 179}\huxtpad{4pt}\raggedright Mazda RX4 Wag\huxbpad{4pt}} &
\multicolumn{1}{r!{\huxvb[190, 190, 190]{0.4}}}{\cellcolor[RGB]{245, 222, 179}\huxtpad{4pt}\raggedleft 21~~\huxbpad{4pt}} &
\multicolumn{1}{r!{\huxvb[190, 190, 190]{0.4}}}{\cellcolor[RGB]{245, 222, 179}\huxtpad{4pt}\raggedleft 6\huxbpad{4pt}} &
\multicolumn{1}{r!{\huxvb[190, 190, 190]{0.4}}}{\cellcolor[RGB]{245, 222, 179}\huxtpad{4pt}\raggedleft 110\huxbpad{4pt}} &
\multicolumn{1}{r!{\huxvb[190, 190, 190]{0.4}}}{\cellcolor[RGB]{245, 222, 179}\huxtpad{4pt}\raggedleft 7.45\huxbpad{4pt}} \tabularnewline[-0.5pt]


\hhline{>{\huxb[190, 190, 190]{0.4}}|>{\huxb[190, 190, 190]{0.4}}|>{\huxb[190, 190, 190]{0.4}}|>{\huxb[190, 190, 190]{0.4}}|>{\huxb[190, 190, 190]{0.4}}|}
\arrayrulecolor{black}

\multicolumn{1}{!{\huxvb{0}}l!{\huxvb[190, 190, 190]{0.4}}}{\cellcolor[RGB]{230, 230, 230}\huxtpad{4pt}\raggedright Hornet 4 Drive\huxbpad{4pt}} &
\multicolumn{1}{r!{\huxvb[190, 190, 190]{0.4}}}{\cellcolor[RGB]{230, 230, 230}\huxtpad{4pt}\raggedleft 21.4\huxbpad{4pt}} &
\multicolumn{1}{r!{\huxvb[190, 190, 190]{0.4}}}{\cellcolor[RGB]{230, 230, 230}\huxtpad{4pt}\raggedleft 6\huxbpad{4pt}} &
\multicolumn{1}{r!{\huxvb[190, 190, 190]{0.4}}}{\cellcolor[RGB]{230, 230, 230}\huxtpad{4pt}\raggedleft 110\huxbpad{4pt}} &
\multicolumn{1}{r!{\huxvb[190, 190, 190]{0.4}}}{\cellcolor[RGB]{230, 230, 230}\huxtpad{4pt}\raggedleft 7.59\huxbpad{4pt}} \tabularnewline[-0.5pt]


\hhline{>{\huxb[190, 190, 190]{0.4}}|>{\huxb[190, 190, 190]{0.4}}|>{\huxb[190, 190, 190]{0.4}}|>{\huxb[190, 190, 190]{0.4}}|>{\huxb[190, 190, 190]{0.4}}|}
\arrayrulecolor{black}

\multicolumn{1}{!{\huxvb{0}}l!{\huxvb[190, 190, 190]{0.4}}}{\cellcolor[RGB]{245, 222, 179}\huxtpad{4pt}\raggedright Merc 280\huxbpad{4pt}} &
\multicolumn{1}{r!{\huxvb[190, 190, 190]{0.4}}}{\cellcolor[RGB]{245, 222, 179}\huxtpad{4pt}\raggedleft 19.2\huxbpad{4pt}} &
\multicolumn{1}{r!{\huxvb[190, 190, 190]{0.4}}}{\cellcolor[RGB]{245, 222, 179}\huxtpad{4pt}\raggedleft 6\huxbpad{4pt}} &
\multicolumn{1}{r!{\huxvb[190, 190, 190]{0.4}}}{\cellcolor[RGB]{245, 222, 179}\huxtpad{4pt}\raggedleft 123\huxbpad{4pt}} &
\multicolumn{1}{r!{\huxvb[190, 190, 190]{0.4}}}{\cellcolor[RGB]{245, 222, 179}\huxtpad{4pt}\raggedleft 6.81\huxbpad{4pt}} \tabularnewline[-0.5pt]


\hhline{>{\huxb[190, 190, 190]{0.4}}|>{\huxb[190, 190, 190]{0.4}}|>{\huxb[190, 190, 190]{0.4}}|>{\huxb[190, 190, 190]{0.4}}|>{\huxb[190, 190, 190]{0.4}}|}
\arrayrulecolor{black}

\multicolumn{1}{!{\huxvb{0}}l!{\huxvb[190, 190, 190]{0.4}}}{\cellcolor[RGB]{230, 230, 230}\huxtpad{4pt}\raggedright Hornet Sportabout\huxbpad{4pt}} &
\multicolumn{1}{r!{\huxvb[190, 190, 190]{0.4}}}{\cellcolor[RGB]{230, 230, 230}\huxtpad{4pt}\raggedleft 18.7\huxbpad{4pt}} &
\multicolumn{1}{r!{\huxvb[190, 190, 190]{0.4}}}{\cellcolor[RGB]{230, 230, 230}\huxtpad{4pt}\raggedleft 8\huxbpad{4pt}} &
\multicolumn{1}{r!{\huxvb[190, 190, 190]{0.4}}}{\cellcolor[RGB]{230, 230, 230}\huxtpad{4pt}\raggedleft 175\huxbpad{4pt}} &
\multicolumn{1}{r!{\huxvb[190, 190, 190]{0.4}}}{\cellcolor[RGB]{230, 230, 230}\huxtpad{4pt}\raggedleft 6.63\huxbpad{4pt}} \tabularnewline[-0.5pt]


\hhline{>{\huxb[190, 190, 190]{0.4}}|>{\huxb[190, 190, 190]{0.4}}|>{\huxb[190, 190, 190]{0.4}}|>{\huxb[190, 190, 190]{0.4}}|>{\huxb[190, 190, 190]{0.4}}|}
\arrayrulecolor{black}

\multicolumn{1}{!{\huxvb{0}}l!{\huxvb[190, 190, 190]{0.4}}}{\cellcolor[RGB]{245, 222, 179}\huxtpad{4pt}\raggedright Duster 360\huxbpad{4pt}} &
\multicolumn{1}{r!{\huxvb[190, 190, 190]{0.4}}}{\cellcolor[RGB]{245, 222, 179}\huxtpad{4pt}\raggedleft 14.3\huxbpad{4pt}} &
\multicolumn{1}{r!{\huxvb[190, 190, 190]{0.4}}}{\cellcolor[RGB]{245, 222, 179}\huxtpad{4pt}\raggedleft 8\huxbpad{4pt}} &
\multicolumn{1}{r!{\huxvb[190, 190, 190]{0.4}}}{\cellcolor[RGB]{245, 222, 179}\huxtpad{4pt}\raggedleft 245\huxbpad{4pt}} &
\multicolumn{1}{r!{\huxvb[190, 190, 190]{0.4}}}{\cellcolor[RGB]{245, 222, 179}\huxtpad{4pt}\raggedleft 5.07\huxbpad{4pt}} \tabularnewline[-0.5pt]


\hhline{>{\huxb[190, 190, 190]{0.4}}|>{\huxb[190, 190, 190]{0.4}}|>{\huxb[190, 190, 190]{0.4}}|>{\huxb[190, 190, 190]{0.4}}|>{\huxb[190, 190, 190]{0.4}}|}
\arrayrulecolor{black}
\end{tabularx}
\end{table}
 

\FloatBarrier

Of course you could also just do \texttt{1:nrow(car\_ht)}, but, in the
middle of a dplyr pipe, you may not know exactly how many rows or
columns you have. Also, these functions make your code easy to read.

You can also use \texttt{dplyr} functions like \texttt{starts\_with()},
\texttt{contains()}, and \texttt{matches()} to specify columns by column
name. For a full list of these functions, see \texttt{?select\_helpers}.

\begin{Shaded}
\begin{Highlighting}[]
\NormalTok{car_ht }\OperatorTok\StringTok{ }\KeywordTok{set_background_color}\NormalTok{(everywhere, }\KeywordTok{starts_with}\NormalTok{(}\StringTok{"C"}\NormalTok{), }\StringTok{"orange"}\NormalTok{)}
\end{Highlighting}
\end{Shaded}

 \begin{table}[h]
\centering
    \providecommand{\huxb}[2][0,0,0]{\arrayrulecolor[RGB]{#1}\global\arrayrulewidth=#2pt}
    \providecommand{\huxvb}[2][0,0,0]{\color[RGB]{#1}\vrule width #2pt}
    \providecommand{\huxtpad}[1]{\rule{0pt}{\baselineskip+#1}}
    \providecommand{\huxbpad}[1]{\rule[-#1]{0pt}{#1}}
  \begin{tabularx}{0.6\textwidth}{p{0.21\textwidth} p{0.09\textwidth} p{0.09\textwidth} p{0.09\textwidth} p{0.12\textwidth}}


\hhline{>{\huxb[190, 190, 190]{0.4}}|>{\huxb[190, 190, 190]{0.4}}|>{\huxb[190, 190, 190]{0.4}}|>{\huxb[190, 190, 190]{0.4}}|>{\huxb[190, 190, 190]{0.4}}|}
\arrayrulecolor{black}

\multicolumn{1}{!{\huxvb{0}}l!{\huxvb[190, 190, 190]{0.4}}}{\cellcolor[RGB]{255, 165, 0}\huxtpad{4pt}\raggedright Car\huxbpad{4pt}} &
\multicolumn{1}{l!{\huxvb[190, 190, 190]{0.4}}}{\huxtpad{4pt}\raggedright MPG\huxbpad{4pt}} &
\multicolumn{1}{l!{\huxvb[190, 190, 190]{0.4}}}{\cellcolor[RGB]{255, 165, 0}\huxtpad{4pt}\raggedright Cylinders\huxbpad{4pt}} &
\multicolumn{1}{l!{\huxvb[190, 190, 190]{0.4}}}{\huxtpad{4pt}\raggedright Horsepower\huxbpad{4pt}} &
\multicolumn{1}{l!{\huxvb[190, 190, 190]{0.4}}}{\huxtpad{4pt}\raggedright kml\huxbpad{4pt}} \tabularnewline[-0.5pt]


\hhline{>{\huxb[190, 190, 190]{0.4}}|>{\huxb[190, 190, 190]{0.4}}|>{\huxb[190, 190, 190]{0.4}}|>{\huxb[190, 190, 190]{0.4}}|>{\huxb[190, 190, 190]{0.4}}|}
\arrayrulecolor{black}

\multicolumn{1}{!{\huxvb{0}}l!{\huxvb[190, 190, 190]{0.4}}}{\cellcolor[RGB]{255, 165, 0}\huxtpad{4pt}\raggedright Valiant\huxbpad{4pt}} &
\multicolumn{1}{r!{\huxvb[190, 190, 190]{0.4}}}{\huxtpad{4pt}\raggedleft 18.1\huxbpad{4pt}} &
\multicolumn{1}{r!{\huxvb[190, 190, 190]{0.4}}}{\cellcolor[RGB]{255, 165, 0}\huxtpad{4pt}\raggedleft 6\huxbpad{4pt}} &
\multicolumn{1}{r!{\huxvb[190, 190, 190]{0.4}}}{\huxtpad{4pt}\raggedleft 105\huxbpad{4pt}} &
\multicolumn{1}{r!{\huxvb[190, 190, 190]{0.4}}}{\huxtpad{4pt}\raggedleft 6.42\huxbpad{4pt}} \tabularnewline[-0.5pt]


\hhline{>{\huxb[190, 190, 190]{0.4}}|>{\huxb[190, 190, 190]{0.4}}|>{\huxb[190, 190, 190]{0.4}}|>{\huxb[190, 190, 190]{0.4}}|>{\huxb[190, 190, 190]{0.4}}|}
\arrayrulecolor{black}

\multicolumn{1}{!{\huxvb{0}}l!{\huxvb[190, 190, 190]{0.4}}}{\cellcolor[RGB]{255, 165, 0}\huxtpad{4pt}\raggedright Mazda RX4\huxbpad{4pt}} &
\multicolumn{1}{r!{\huxvb[190, 190, 190]{0.4}}}{\huxtpad{4pt}\raggedleft 21~~\huxbpad{4pt}} &
\multicolumn{1}{r!{\huxvb[190, 190, 190]{0.4}}}{\cellcolor[RGB]{255, 165, 0}\huxtpad{4pt}\raggedleft 6\huxbpad{4pt}} &
\multicolumn{1}{r!{\huxvb[190, 190, 190]{0.4}}}{\huxtpad{4pt}\raggedleft 110\huxbpad{4pt}} &
\multicolumn{1}{r!{\huxvb[190, 190, 190]{0.4}}}{\huxtpad{4pt}\raggedleft 7.45\huxbpad{4pt}} \tabularnewline[-0.5pt]


\hhline{>{\huxb[190, 190, 190]{0.4}}|>{\huxb[190, 190, 190]{0.4}}|>{\huxb[190, 190, 190]{0.4}}|>{\huxb[190, 190, 190]{0.4}}|>{\huxb[190, 190, 190]{0.4}}|}
\arrayrulecolor{black}

\multicolumn{1}{!{\huxvb{0}}l!{\huxvb[190, 190, 190]{0.4}}}{\cellcolor[RGB]{255, 165, 0}\huxtpad{4pt}\raggedright Mazda RX4 Wag\huxbpad{4pt}} &
\multicolumn{1}{r!{\huxvb[190, 190, 190]{0.4}}}{\huxtpad{4pt}\raggedleft 21~~\huxbpad{4pt}} &
\multicolumn{1}{r!{\huxvb[190, 190, 190]{0.4}}}{\cellcolor[RGB]{255, 165, 0}\huxtpad{4pt}\raggedleft 6\huxbpad{4pt}} &
\multicolumn{1}{r!{\huxvb[190, 190, 190]{0.4}}}{\huxtpad{4pt}\raggedleft 110\huxbpad{4pt}} &
\multicolumn{1}{r!{\huxvb[190, 190, 190]{0.4}}}{\huxtpad{4pt}\raggedleft 7.45\huxbpad{4pt}} \tabularnewline[-0.5pt]


\hhline{>{\huxb[190, 190, 190]{0.4}}|>{\huxb[190, 190, 190]{0.4}}|>{\huxb[190, 190, 190]{0.4}}|>{\huxb[190, 190, 190]{0.4}}|>{\huxb[190, 190, 190]{0.4}}|}
\arrayrulecolor{black}

\multicolumn{1}{!{\huxvb{0}}l!{\huxvb[190, 190, 190]{0.4}}}{\cellcolor[RGB]{255, 165, 0}\huxtpad{4pt}\raggedright Hornet 4 Drive\huxbpad{4pt}} &
\multicolumn{1}{r!{\huxvb[190, 190, 190]{0.4}}}{\huxtpad{4pt}\raggedleft 21.4\huxbpad{4pt}} &
\multicolumn{1}{r!{\huxvb[190, 190, 190]{0.4}}}{\cellcolor[RGB]{255, 165, 0}\huxtpad{4pt}\raggedleft 6\huxbpad{4pt}} &
\multicolumn{1}{r!{\huxvb[190, 190, 190]{0.4}}}{\huxtpad{4pt}\raggedleft 110\huxbpad{4pt}} &
\multicolumn{1}{r!{\huxvb[190, 190, 190]{0.4}}}{\huxtpad{4pt}\raggedleft 7.59\huxbpad{4pt}} \tabularnewline[-0.5pt]


\hhline{>{\huxb[190, 190, 190]{0.4}}|>{\huxb[190, 190, 190]{0.4}}|>{\huxb[190, 190, 190]{0.4}}|>{\huxb[190, 190, 190]{0.4}}|>{\huxb[190, 190, 190]{0.4}}|}
\arrayrulecolor{black}

\multicolumn{1}{!{\huxvb{0}}l!{\huxvb[190, 190, 190]{0.4}}}{\cellcolor[RGB]{255, 165, 0}\huxtpad{4pt}\raggedright Merc 280\huxbpad{4pt}} &
\multicolumn{1}{r!{\huxvb[190, 190, 190]{0.4}}}{\huxtpad{4pt}\raggedleft 19.2\huxbpad{4pt}} &
\multicolumn{1}{r!{\huxvb[190, 190, 190]{0.4}}}{\cellcolor[RGB]{255, 165, 0}\huxtpad{4pt}\raggedleft 6\huxbpad{4pt}} &
\multicolumn{1}{r!{\huxvb[190, 190, 190]{0.4}}}{\huxtpad{4pt}\raggedleft 123\huxbpad{4pt}} &
\multicolumn{1}{r!{\huxvb[190, 190, 190]{0.4}}}{\huxtpad{4pt}\raggedleft 6.81\huxbpad{4pt}} \tabularnewline[-0.5pt]


\hhline{>{\huxb[190, 190, 190]{0.4}}|>{\huxb[190, 190, 190]{0.4}}|>{\huxb[190, 190, 190]{0.4}}|>{\huxb[190, 190, 190]{0.4}}|>{\huxb[190, 190, 190]{0.4}}|}
\arrayrulecolor{black}

\multicolumn{1}{!{\huxvb{0}}l!{\huxvb[190, 190, 190]{0.4}}}{\cellcolor[RGB]{255, 165, 0}\huxtpad{4pt}\raggedright Hornet Sportabout\huxbpad{4pt}} &
\multicolumn{1}{r!{\huxvb[190, 190, 190]{0.4}}}{\huxtpad{4pt}\raggedleft 18.7\huxbpad{4pt}} &
\multicolumn{1}{r!{\huxvb[190, 190, 190]{0.4}}}{\cellcolor[RGB]{255, 165, 0}\huxtpad{4pt}\raggedleft 8\huxbpad{4pt}} &
\multicolumn{1}{r!{\huxvb[190, 190, 190]{0.4}}}{\huxtpad{4pt}\raggedleft 175\huxbpad{4pt}} &
\multicolumn{1}{r!{\huxvb[190, 190, 190]{0.4}}}{\huxtpad{4pt}\raggedleft 6.63\huxbpad{4pt}} \tabularnewline[-0.5pt]


\hhline{>{\huxb[190, 190, 190]{0.4}}|>{\huxb[190, 190, 190]{0.4}}|>{\huxb[190, 190, 190]{0.4}}|>{\huxb[190, 190, 190]{0.4}}|>{\huxb[190, 190, 190]{0.4}}|}
\arrayrulecolor{black}

\multicolumn{1}{!{\huxvb{0}}l!{\huxvb[190, 190, 190]{0.4}}}{\cellcolor[RGB]{255, 165, 0}\huxtpad{4pt}\raggedright Duster 360\huxbpad{4pt}} &
\multicolumn{1}{r!{\huxvb[190, 190, 190]{0.4}}}{\huxtpad{4pt}\raggedleft 14.3\huxbpad{4pt}} &
\multicolumn{1}{r!{\huxvb[190, 190, 190]{0.4}}}{\cellcolor[RGB]{255, 165, 0}\huxtpad{4pt}\raggedleft 8\huxbpad{4pt}} &
\multicolumn{1}{r!{\huxvb[190, 190, 190]{0.4}}}{\huxtpad{4pt}\raggedleft 245\huxbpad{4pt}} &
\multicolumn{1}{r!{\huxvb[190, 190, 190]{0.4}}}{\huxtpad{4pt}\raggedleft 5.07\huxbpad{4pt}} \tabularnewline[-0.5pt]


\hhline{>{\huxb[190, 190, 190]{0.4}}|>{\huxb[190, 190, 190]{0.4}}|>{\huxb[190, 190, 190]{0.4}}|>{\huxb[190, 190, 190]{0.4}}|>{\huxb[190, 190, 190]{0.4}}|}
\arrayrulecolor{black}
\end{tabularx}
\end{table}
 

\begin{Shaded}
\begin{Highlighting}[]
\NormalTok{car_ht }\OperatorTok\StringTok{ }\KeywordTok{set_italic}\NormalTok{(everywhere, dplyr}\OperatorTok{::}\KeywordTok{matches}\NormalTok{(}\StringTok{"[aeiou]"}\NormalTok{), }\OtherTok{TRUE}\NormalTok{)}
\end{Highlighting}
\end{Shaded}

 \begin{table}[h]
\centering
    \providecommand{\huxb}[2][0,0,0]{\arrayrulecolor[RGB]{#1}\global\arrayrulewidth=#2pt}
    \providecommand{\huxvb}[2][0,0,0]{\color[RGB]{#1}\vrule width #2pt}
    \providecommand{\huxtpad}[1]{\rule{0pt}{\baselineskip+#1}}
    \providecommand{\huxbpad}[1]{\rule[-#1]{0pt}{#1}}
  \begin{tabularx}{0.6\textwidth}{p{0.21\textwidth} p{0.09\textwidth} p{0.09\textwidth} p{0.09\textwidth} p{0.12\textwidth}}


\hhline{>{\huxb[190, 190, 190]{0.4}}|>{\huxb[190, 190, 190]{0.4}}|>{\huxb[190, 190, 190]{0.4}}|>{\huxb[190, 190, 190]{0.4}}|>{\huxb[190, 190, 190]{0.4}}|}
\arrayrulecolor{black}

\multicolumn{1}{!{\huxvb{0}}l!{\huxvb[190, 190, 190]{0.4}}}{\huxtpad{4pt}\raggedright \textit{Car}\huxbpad{4pt}} &
\multicolumn{1}{l!{\huxvb[190, 190, 190]{0.4}}}{\huxtpad{4pt}\raggedright MPG\huxbpad{4pt}} &
\multicolumn{1}{l!{\huxvb[190, 190, 190]{0.4}}}{\huxtpad{4pt}\raggedright \textit{Cylinders}\huxbpad{4pt}} &
\multicolumn{1}{l!{\huxvb[190, 190, 190]{0.4}}}{\huxtpad{4pt}\raggedright \textit{Horsepower}\huxbpad{4pt}} &
\multicolumn{1}{l!{\huxvb[190, 190, 190]{0.4}}}{\huxtpad{4pt}\raggedright kml\huxbpad{4pt}} \tabularnewline[-0.5pt]


\hhline{>{\huxb[190, 190, 190]{0.4}}|>{\huxb[190, 190, 190]{0.4}}|>{\huxb[190, 190, 190]{0.4}}|>{\huxb[190, 190, 190]{0.4}}|>{\huxb[190, 190, 190]{0.4}}|}
\arrayrulecolor{black}

\multicolumn{1}{!{\huxvb{0}}l!{\huxvb[190, 190, 190]{0.4}}}{\huxtpad{4pt}\raggedright \textit{Valiant}\huxbpad{4pt}} &
\multicolumn{1}{r!{\huxvb[190, 190, 190]{0.4}}}{\huxtpad{4pt}\raggedleft 18.1\huxbpad{4pt}} &
\multicolumn{1}{r!{\huxvb[190, 190, 190]{0.4}}}{\huxtpad{4pt}\raggedleft \textit{6}\huxbpad{4pt}} &
\multicolumn{1}{r!{\huxvb[190, 190, 190]{0.4}}}{\huxtpad{4pt}\raggedleft \textit{105}\huxbpad{4pt}} &
\multicolumn{1}{r!{\huxvb[190, 190, 190]{0.4}}}{\huxtpad{4pt}\raggedleft 6.42\huxbpad{4pt}} \tabularnewline[-0.5pt]


\hhline{>{\huxb[190, 190, 190]{0.4}}|>{\huxb[190, 190, 190]{0.4}}|>{\huxb[190, 190, 190]{0.4}}|>{\huxb[190, 190, 190]{0.4}}|>{\huxb[190, 190, 190]{0.4}}|}
\arrayrulecolor{black}

\multicolumn{1}{!{\huxvb{0}}l!{\huxvb[190, 190, 190]{0.4}}}{\huxtpad{4pt}\raggedright \textit{Mazda RX4}\huxbpad{4pt}} &
\multicolumn{1}{r!{\huxvb[190, 190, 190]{0.4}}}{\huxtpad{4pt}\raggedleft 21~~\huxbpad{4pt}} &
\multicolumn{1}{r!{\huxvb[190, 190, 190]{0.4}}}{\huxtpad{4pt}\raggedleft \textit{6}\huxbpad{4pt}} &
\multicolumn{1}{r!{\huxvb[190, 190, 190]{0.4}}}{\huxtpad{4pt}\raggedleft \textit{110}\huxbpad{4pt}} &
\multicolumn{1}{r!{\huxvb[190, 190, 190]{0.4}}}{\huxtpad{4pt}\raggedleft 7.45\huxbpad{4pt}} \tabularnewline[-0.5pt]


\hhline{>{\huxb[190, 190, 190]{0.4}}|>{\huxb[190, 190, 190]{0.4}}|>{\huxb[190, 190, 190]{0.4}}|>{\huxb[190, 190, 190]{0.4}}|>{\huxb[190, 190, 190]{0.4}}|}
\arrayrulecolor{black}

\multicolumn{1}{!{\huxvb{0}}l!{\huxvb[190, 190, 190]{0.4}}}{\huxtpad{4pt}\raggedright \textit{Mazda RX4 Wag}\huxbpad{4pt}} &
\multicolumn{1}{r!{\huxvb[190, 190, 190]{0.4}}}{\huxtpad{4pt}\raggedleft 21~~\huxbpad{4pt}} &
\multicolumn{1}{r!{\huxvb[190, 190, 190]{0.4}}}{\huxtpad{4pt}\raggedleft \textit{6}\huxbpad{4pt}} &
\multicolumn{1}{r!{\huxvb[190, 190, 190]{0.4}}}{\huxtpad{4pt}\raggedleft \textit{110}\huxbpad{4pt}} &
\multicolumn{1}{r!{\huxvb[190, 190, 190]{0.4}}}{\huxtpad{4pt}\raggedleft 7.45\huxbpad{4pt}} \tabularnewline[-0.5pt]


\hhline{>{\huxb[190, 190, 190]{0.4}}|>{\huxb[190, 190, 190]{0.4}}|>{\huxb[190, 190, 190]{0.4}}|>{\huxb[190, 190, 190]{0.4}}|>{\huxb[190, 190, 190]{0.4}}|}
\arrayrulecolor{black}

\multicolumn{1}{!{\huxvb{0}}l!{\huxvb[190, 190, 190]{0.4}}}{\huxtpad{4pt}\raggedright \textit{Hornet 4 Drive}\huxbpad{4pt}} &
\multicolumn{1}{r!{\huxvb[190, 190, 190]{0.4}}}{\huxtpad{4pt}\raggedleft 21.4\huxbpad{4pt}} &
\multicolumn{1}{r!{\huxvb[190, 190, 190]{0.4}}}{\huxtpad{4pt}\raggedleft \textit{6}\huxbpad{4pt}} &
\multicolumn{1}{r!{\huxvb[190, 190, 190]{0.4}}}{\huxtpad{4pt}\raggedleft \textit{110}\huxbpad{4pt}} &
\multicolumn{1}{r!{\huxvb[190, 190, 190]{0.4}}}{\huxtpad{4pt}\raggedleft 7.59\huxbpad{4pt}} \tabularnewline[-0.5pt]


\hhline{>{\huxb[190, 190, 190]{0.4}}|>{\huxb[190, 190, 190]{0.4}}|>{\huxb[190, 190, 190]{0.4}}|>{\huxb[190, 190, 190]{0.4}}|>{\huxb[190, 190, 190]{0.4}}|}
\arrayrulecolor{black}

\multicolumn{1}{!{\huxvb{0}}l!{\huxvb[190, 190, 190]{0.4}}}{\huxtpad{4pt}\raggedright \textit{Merc 280}\huxbpad{4pt}} &
\multicolumn{1}{r!{\huxvb[190, 190, 190]{0.4}}}{\huxtpad{4pt}\raggedleft 19.2\huxbpad{4pt}} &
\multicolumn{1}{r!{\huxvb[190, 190, 190]{0.4}}}{\huxtpad{4pt}\raggedleft \textit{6}\huxbpad{4pt}} &
\multicolumn{1}{r!{\huxvb[190, 190, 190]{0.4}}}{\huxtpad{4pt}\raggedleft \textit{123}\huxbpad{4pt}} &
\multicolumn{1}{r!{\huxvb[190, 190, 190]{0.4}}}{\huxtpad{4pt}\raggedleft 6.81\huxbpad{4pt}} \tabularnewline[-0.5pt]


\hhline{>{\huxb[190, 190, 190]{0.4}}|>{\huxb[190, 190, 190]{0.4}}|>{\huxb[190, 190, 190]{0.4}}|>{\huxb[190, 190, 190]{0.4}}|>{\huxb[190, 190, 190]{0.4}}|}
\arrayrulecolor{black}

\multicolumn{1}{!{\huxvb{0}}l!{\huxvb[190, 190, 190]{0.4}}}{\huxtpad{4pt}\raggedright \textit{Hornet Sportabout}\huxbpad{4pt}} &
\multicolumn{1}{r!{\huxvb[190, 190, 190]{0.4}}}{\huxtpad{4pt}\raggedleft 18.7\huxbpad{4pt}} &
\multicolumn{1}{r!{\huxvb[190, 190, 190]{0.4}}}{\huxtpad{4pt}\raggedleft \textit{8}\huxbpad{4pt}} &
\multicolumn{1}{r!{\huxvb[190, 190, 190]{0.4}}}{\huxtpad{4pt}\raggedleft \textit{175}\huxbpad{4pt}} &
\multicolumn{1}{r!{\huxvb[190, 190, 190]{0.4}}}{\huxtpad{4pt}\raggedleft 6.63\huxbpad{4pt}} \tabularnewline[-0.5pt]


\hhline{>{\huxb[190, 190, 190]{0.4}}|>{\huxb[190, 190, 190]{0.4}}|>{\huxb[190, 190, 190]{0.4}}|>{\huxb[190, 190, 190]{0.4}}|>{\huxb[190, 190, 190]{0.4}}|}
\arrayrulecolor{black}

\multicolumn{1}{!{\huxvb{0}}l!{\huxvb[190, 190, 190]{0.4}}}{\huxtpad{4pt}\raggedright \textit{Duster 360}\huxbpad{4pt}} &
\multicolumn{1}{r!{\huxvb[190, 190, 190]{0.4}}}{\huxtpad{4pt}\raggedleft 14.3\huxbpad{4pt}} &
\multicolumn{1}{r!{\huxvb[190, 190, 190]{0.4}}}{\huxtpad{4pt}\raggedleft \textit{8}\huxbpad{4pt}} &
\multicolumn{1}{r!{\huxvb[190, 190, 190]{0.4}}}{\huxtpad{4pt}\raggedleft \textit{245}\huxbpad{4pt}} &
\multicolumn{1}{r!{\huxvb[190, 190, 190]{0.4}}}{\huxtpad{4pt}\raggedleft 5.07\huxbpad{4pt}} \tabularnewline[-0.5pt]


\hhline{>{\huxb[190, 190, 190]{0.4}}|>{\huxb[190, 190, 190]{0.4}}|>{\huxb[190, 190, 190]{0.4}}|>{\huxb[190, 190, 190]{0.4}}|>{\huxb[190, 190, 190]{0.4}}|}
\arrayrulecolor{black}
\end{tabularx}
\end{table}
 

\FloatBarrier

Note that unlike in \texttt{dplyr}'s \texttt{select} function, you have
to specify rows as well as columns.

Lastly, remember that you can set a property for every cell by simply
omitting the \texttt{row} and \texttt{col} arguments, like this:
\texttt{set\_background\_color(ht,\ \textquotesingle{}orange\textquotesingle{})}.

\hypertarget{conditional-formatting}{%
\subsection{Conditional formatting}\label{conditional-formatting}}

You may want to apply conditional formatting to cells, based on their
contents. Suppose we want to display a table of correlations, and to
highlight ones which are significant. We can use the \texttt{where()}
function to select those cells.

\begin{Shaded}
\begin{Highlighting}[]
\KeywordTok{library}\NormalTok{(psych)}
\KeywordTok{data}\NormalTok{(attitude)}
\NormalTok{att_corr <-}\StringTok{ }\KeywordTok{corr.test}\NormalTok{(}\KeywordTok{as.matrix}\NormalTok{(attitude))}

\NormalTok{att_hux <-}\StringTok{ }\KeywordTok{as_hux}\NormalTok{(att_corr}\OperatorTok{$}\NormalTok{r)                                           }\OperatorTok\StringTok{ }
\StringTok{      }\CommentTok{# selects cells with p < 0.05:}
\StringTok{      }\KeywordTok{set_background_color}\NormalTok{(}\KeywordTok{where}\NormalTok{(att_corr}\OperatorTok{$}\NormalTok{p }\OperatorTok{<}\StringTok{ }\FloatTok{0.05}\NormalTok{), }\StringTok{"yellow"}\NormalTok{)          }\OperatorTok\StringTok{ }
\StringTok{      }\CommentTok{# selects cells with p < 0.01:}
\StringTok{      }\KeywordTok{set_background_color}\NormalTok{(}\KeywordTok{where}\NormalTok{(att_corr}\OperatorTok{$}\NormalTok{p }\OperatorTok{<}\StringTok{ }\FloatTok{0.01}\NormalTok{), }\StringTok{"orange"}\NormalTok{)          }\OperatorTok\StringTok{ }
\StringTok{      }\KeywordTok{set_text_color}\NormalTok{(}\KeywordTok{where}\NormalTok{(}\KeywordTok{row}\NormalTok{(att_corr}\OperatorTok{$}\NormalTok{r) }\OperatorTok{==}\StringTok{ }\KeywordTok{col}\NormalTok{(att_corr}\OperatorTok{$}\NormalTok{r)), }\StringTok{"grey"}\NormalTok{) }


\NormalTok{att_hux <-}\StringTok{ }\NormalTok{att_hux                                                      }\OperatorTok\StringTok{ }
\StringTok{      }\NormalTok{huxtable}\OperatorTok{::}\KeywordTok{add_rownames}\NormalTok{()                                          }\OperatorTok\StringTok{ }
\StringTok{      }\NormalTok{huxtable}\OperatorTok{::}\KeywordTok{add_colnames}\NormalTok{()                                          }\OperatorTok
\StringTok{      }\KeywordTok{set_caption}\NormalTok{(}\StringTok{'Correlations in attitudes among 30 departments'}\NormalTok{)     }\OperatorTok\StringTok{ }
\StringTok{      }\KeywordTok{set_bold}\NormalTok{(}\DecValTok{1}\NormalTok{, everywhere, }\OtherTok{TRUE}\NormalTok{)                                     }\OperatorTok\StringTok{ }
\StringTok{      }\KeywordTok{set_bold}\NormalTok{(everywhere, }\DecValTok{1}\NormalTok{, }\OtherTok{TRUE}\NormalTok{)                                     }\OperatorTok\StringTok{ }
\StringTok{      }\KeywordTok{set_all_borders}\NormalTok{(}\DecValTok{1}\NormalTok{)                                                }\OperatorTok
\StringTok{      }\KeywordTok{set_number_format}\NormalTok{(}\DecValTok{2}\NormalTok{)                                              }\OperatorTok\StringTok{ }
\StringTok{      }\KeywordTok{set_position}\NormalTok{(}\StringTok{'left'}\NormalTok{)}

\NormalTok{att_hux}
\end{Highlighting}
\end{Shaded}

 \begin{table}[h]
\begin{raggedright}\captionsetup{justification=raggedright,singlelinecheck=off}
\caption{Correlations in attitudes among 30 departments}

    \providecommand{\huxb}[2][0,0,0]{\arrayrulecolor[RGB]{#1}\global\arrayrulewidth=#2pt}
    \providecommand{\huxvb}[2][0,0,0]{\color[RGB]{#1}\vrule width #2pt}
    \providecommand{\huxtpad}[1]{\rule{0pt}{\baselineskip+#1}}
    \providecommand{\huxbpad}[1]{\rule[-#1]{0pt}{#1}}
  \begin{tabularx}{0.5\textwidth}{p{0.0625\textwidth} p{0.0625\textwidth} p{0.0625\textwidth} p{0.0625\textwidth} p{0.0625\textwidth} p{0.0625\textwidth} p{0.0625\textwidth} p{0.0625\textwidth}}


\hhline{>{\huxb{1}}->{\huxb{1}}->{\huxb{1}}->{\huxb{1}}->{\huxb{1}}->{\huxb{1}}->{\huxb{1}}->{\huxb{1}}-}
\arrayrulecolor{black}

\multicolumn{1}{!{\huxvb{1}}l!{\huxvb{1}}}{\huxtpad{4pt}\raggedright \textbf{rownames}\huxbpad{4pt}} &
\multicolumn{1}{l!{\huxvb{1}}}{\huxtpad{4pt}\raggedright \textbf{rating}\huxbpad{4pt}} &
\multicolumn{1}{l!{\huxvb{1}}}{\huxtpad{4pt}\raggedright \textbf{complaints}\huxbpad{4pt}} &
\multicolumn{1}{l!{\huxvb{1}}}{\huxtpad{4pt}\raggedright \textbf{privileges}\huxbpad{4pt}} &
\multicolumn{1}{l!{\huxvb{1}}}{\huxtpad{4pt}\raggedright \textbf{learning}\huxbpad{4pt}} &
\multicolumn{1}{l!{\huxvb{1}}}{\huxtpad{4pt}\raggedright \textbf{raises}\huxbpad{4pt}} &
\multicolumn{1}{l!{\huxvb{1}}}{\huxtpad{4pt}\raggedright \textbf{critical}\huxbpad{4pt}} &
\multicolumn{1}{l!{\huxvb{1}}}{\huxtpad{4pt}\raggedright \textbf{advance}\huxbpad{4pt}} \tabularnewline[-0.5pt]


\hhline{>{\huxb{1}}->{\huxb{1}}->{\huxb{1}}->{\huxb{1}}->{\huxb{1}}->{\huxb{1}}->{\huxb{1}}->{\huxb{1}}-}
\arrayrulecolor{black}

\multicolumn{1}{!{\huxvb{1}}l!{\huxvb{1}}}{\huxtpad{4pt}\raggedright \textbf{rating}\huxbpad{4pt}} &
\multicolumn{1}{r!{\huxvb{1}}}{\cellcolor[RGB]{255, 165, 0}\huxtpad{4pt}\raggedleft \textcolor[RGB]{190, 190, 190}{1.00}\huxbpad{4pt}} &
\multicolumn{1}{r!{\huxvb{1}}}{\cellcolor[RGB]{255, 165, 0}\huxtpad{4pt}\raggedleft 0.83\huxbpad{4pt}} &
\multicolumn{1}{r!{\huxvb{1}}}{\huxtpad{4pt}\raggedleft 0.43\huxbpad{4pt}} &
\multicolumn{1}{r!{\huxvb{1}}}{\cellcolor[RGB]{255, 165, 0}\huxtpad{4pt}\raggedleft 0.62\huxbpad{4pt}} &
\multicolumn{1}{r!{\huxvb{1}}}{\cellcolor[RGB]{255, 165, 0}\huxtpad{4pt}\raggedleft 0.59\huxbpad{4pt}} &
\multicolumn{1}{r!{\huxvb{1}}}{\huxtpad{4pt}\raggedleft 0.16\huxbpad{4pt}} &
\multicolumn{1}{r!{\huxvb{1}}}{\huxtpad{4pt}\raggedleft 0.16\huxbpad{4pt}} \tabularnewline[-0.5pt]


\hhline{>{\huxb{1}}->{\huxb{1}}->{\huxb{1}}->{\huxb{1}}->{\huxb{1}}->{\huxb{1}}->{\huxb{1}}->{\huxb{1}}-}
\arrayrulecolor{black}

\multicolumn{1}{!{\huxvb{1}}l!{\huxvb{1}}}{\huxtpad{4pt}\raggedright \textbf{complaints}\huxbpad{4pt}} &
\multicolumn{1}{r!{\huxvb{1}}}{\cellcolor[RGB]{255, 165, 0}\huxtpad{4pt}\raggedleft 0.83\huxbpad{4pt}} &
\multicolumn{1}{r!{\huxvb{1}}}{\cellcolor[RGB]{255, 165, 0}\huxtpad{4pt}\raggedleft \textcolor[RGB]{190, 190, 190}{1.00}\huxbpad{4pt}} &
\multicolumn{1}{r!{\huxvb{1}}}{\cellcolor[RGB]{255, 255, 0}\huxtpad{4pt}\raggedleft 0.56\huxbpad{4pt}} &
\multicolumn{1}{r!{\huxvb{1}}}{\cellcolor[RGB]{255, 165, 0}\huxtpad{4pt}\raggedleft 0.60\huxbpad{4pt}} &
\multicolumn{1}{r!{\huxvb{1}}}{\cellcolor[RGB]{255, 165, 0}\huxtpad{4pt}\raggedleft 0.67\huxbpad{4pt}} &
\multicolumn{1}{r!{\huxvb{1}}}{\huxtpad{4pt}\raggedleft 0.19\huxbpad{4pt}} &
\multicolumn{1}{r!{\huxvb{1}}}{\huxtpad{4pt}\raggedleft 0.22\huxbpad{4pt}} \tabularnewline[-0.5pt]


\hhline{>{\huxb{1}}->{\huxb{1}}->{\huxb{1}}->{\huxb{1}}->{\huxb{1}}->{\huxb{1}}->{\huxb{1}}->{\huxb{1}}-}
\arrayrulecolor{black}

\multicolumn{1}{!{\huxvb{1}}l!{\huxvb{1}}}{\huxtpad{4pt}\raggedright \textbf{privileges}\huxbpad{4pt}} &
\multicolumn{1}{r!{\huxvb{1}}}{\cellcolor[RGB]{255, 255, 0}\huxtpad{4pt}\raggedleft 0.43\huxbpad{4pt}} &
\multicolumn{1}{r!{\huxvb{1}}}{\cellcolor[RGB]{255, 165, 0}\huxtpad{4pt}\raggedleft 0.56\huxbpad{4pt}} &
\multicolumn{1}{r!{\huxvb{1}}}{\cellcolor[RGB]{255, 165, 0}\huxtpad{4pt}\raggedleft \textcolor[RGB]{190, 190, 190}{1.00}\huxbpad{4pt}} &
\multicolumn{1}{r!{\huxvb{1}}}{\huxtpad{4pt}\raggedleft 0.49\huxbpad{4pt}} &
\multicolumn{1}{r!{\huxvb{1}}}{\huxtpad{4pt}\raggedleft 0.45\huxbpad{4pt}} &
\multicolumn{1}{r!{\huxvb{1}}}{\huxtpad{4pt}\raggedleft 0.15\huxbpad{4pt}} &
\multicolumn{1}{r!{\huxvb{1}}}{\huxtpad{4pt}\raggedleft 0.34\huxbpad{4pt}} \tabularnewline[-0.5pt]


\hhline{>{\huxb{1}}->{\huxb{1}}->{\huxb{1}}->{\huxb{1}}->{\huxb{1}}->{\huxb{1}}->{\huxb{1}}->{\huxb{1}}-}
\arrayrulecolor{black}

\multicolumn{1}{!{\huxvb{1}}l!{\huxvb{1}}}{\huxtpad{4pt}\raggedright \textbf{learning}\huxbpad{4pt}} &
\multicolumn{1}{r!{\huxvb{1}}}{\cellcolor[RGB]{255, 165, 0}\huxtpad{4pt}\raggedleft 0.62\huxbpad{4pt}} &
\multicolumn{1}{r!{\huxvb{1}}}{\cellcolor[RGB]{255, 165, 0}\huxtpad{4pt}\raggedleft 0.60\huxbpad{4pt}} &
\multicolumn{1}{r!{\huxvb{1}}}{\cellcolor[RGB]{255, 165, 0}\huxtpad{4pt}\raggedleft 0.49\huxbpad{4pt}} &
\multicolumn{1}{r!{\huxvb{1}}}{\cellcolor[RGB]{255, 165, 0}\huxtpad{4pt}\raggedleft \textcolor[RGB]{190, 190, 190}{1.00}\huxbpad{4pt}} &
\multicolumn{1}{r!{\huxvb{1}}}{\cellcolor[RGB]{255, 165, 0}\huxtpad{4pt}\raggedleft 0.64\huxbpad{4pt}} &
\multicolumn{1}{r!{\huxvb{1}}}{\huxtpad{4pt}\raggedleft 0.12\huxbpad{4pt}} &
\multicolumn{1}{r!{\huxvb{1}}}{\cellcolor[RGB]{255, 255, 0}\huxtpad{4pt}\raggedleft 0.53\huxbpad{4pt}} \tabularnewline[-0.5pt]


\hhline{>{\huxb{1}}->{\huxb{1}}->{\huxb{1}}->{\huxb{1}}->{\huxb{1}}->{\huxb{1}}->{\huxb{1}}->{\huxb{1}}-}
\arrayrulecolor{black}

\multicolumn{1}{!{\huxvb{1}}l!{\huxvb{1}}}{\huxtpad{4pt}\raggedright \textbf{raises}\huxbpad{4pt}} &
\multicolumn{1}{r!{\huxvb{1}}}{\cellcolor[RGB]{255, 165, 0}\huxtpad{4pt}\raggedleft 0.59\huxbpad{4pt}} &
\multicolumn{1}{r!{\huxvb{1}}}{\cellcolor[RGB]{255, 165, 0}\huxtpad{4pt}\raggedleft 0.67\huxbpad{4pt}} &
\multicolumn{1}{r!{\huxvb{1}}}{\cellcolor[RGB]{255, 255, 0}\huxtpad{4pt}\raggedleft 0.45\huxbpad{4pt}} &
\multicolumn{1}{r!{\huxvb{1}}}{\cellcolor[RGB]{255, 165, 0}\huxtpad{4pt}\raggedleft 0.64\huxbpad{4pt}} &
\multicolumn{1}{r!{\huxvb{1}}}{\cellcolor[RGB]{255, 165, 0}\huxtpad{4pt}\raggedleft \textcolor[RGB]{190, 190, 190}{1.00}\huxbpad{4pt}} &
\multicolumn{1}{r!{\huxvb{1}}}{\huxtpad{4pt}\raggedleft 0.38\huxbpad{4pt}} &
\multicolumn{1}{r!{\huxvb{1}}}{\cellcolor[RGB]{255, 255, 0}\huxtpad{4pt}\raggedleft 0.57\huxbpad{4pt}} \tabularnewline[-0.5pt]


\hhline{>{\huxb{1}}->{\huxb{1}}->{\huxb{1}}->{\huxb{1}}->{\huxb{1}}->{\huxb{1}}->{\huxb{1}}->{\huxb{1}}-}
\arrayrulecolor{black}

\multicolumn{1}{!{\huxvb{1}}l!{\huxvb{1}}}{\huxtpad{4pt}\raggedright \textbf{critical}\huxbpad{4pt}} &
\multicolumn{1}{r!{\huxvb{1}}}{\huxtpad{4pt}\raggedleft 0.16\huxbpad{4pt}} &
\multicolumn{1}{r!{\huxvb{1}}}{\huxtpad{4pt}\raggedleft 0.19\huxbpad{4pt}} &
\multicolumn{1}{r!{\huxvb{1}}}{\huxtpad{4pt}\raggedleft 0.15\huxbpad{4pt}} &
\multicolumn{1}{r!{\huxvb{1}}}{\huxtpad{4pt}\raggedleft 0.12\huxbpad{4pt}} &
\multicolumn{1}{r!{\huxvb{1}}}{\cellcolor[RGB]{255, 255, 0}\huxtpad{4pt}\raggedleft 0.38\huxbpad{4pt}} &
\multicolumn{1}{r!{\huxvb{1}}}{\cellcolor[RGB]{255, 165, 0}\huxtpad{4pt}\raggedleft \textcolor[RGB]{190, 190, 190}{1.00}\huxbpad{4pt}} &
\multicolumn{1}{r!{\huxvb{1}}}{\huxtpad{4pt}\raggedleft 0.28\huxbpad{4pt}} \tabularnewline[-0.5pt]


\hhline{>{\huxb{1}}->{\huxb{1}}->{\huxb{1}}->{\huxb{1}}->{\huxb{1}}->{\huxb{1}}->{\huxb{1}}->{\huxb{1}}-}
\arrayrulecolor{black}

\multicolumn{1}{!{\huxvb{1}}l!{\huxvb{1}}}{\huxtpad{4pt}\raggedright \textbf{advance}\huxbpad{4pt}} &
\multicolumn{1}{r!{\huxvb{1}}}{\huxtpad{4pt}\raggedleft 0.16\huxbpad{4pt}} &
\multicolumn{1}{r!{\huxvb{1}}}{\huxtpad{4pt}\raggedleft 0.22\huxbpad{4pt}} &
\multicolumn{1}{r!{\huxvb{1}}}{\huxtpad{4pt}\raggedleft 0.34\huxbpad{4pt}} &
\multicolumn{1}{r!{\huxvb{1}}}{\cellcolor[RGB]{255, 165, 0}\huxtpad{4pt}\raggedleft 0.53\huxbpad{4pt}} &
\multicolumn{1}{r!{\huxvb{1}}}{\cellcolor[RGB]{255, 165, 0}\huxtpad{4pt}\raggedleft 0.57\huxbpad{4pt}} &
\multicolumn{1}{r!{\huxvb{1}}}{\huxtpad{4pt}\raggedleft 0.28\huxbpad{4pt}} &
\multicolumn{1}{r!{\huxvb{1}}}{\cellcolor[RGB]{255, 165, 0}\huxtpad{4pt}\raggedleft \textcolor[RGB]{190, 190, 190}{1.00}\huxbpad{4pt}} \tabularnewline[-0.5pt]


\hhline{>{\huxb{1}}->{\huxb{1}}->{\huxb{1}}->{\huxb{1}}->{\huxb{1}}->{\huxb{1}}->{\huxb{1}}->{\huxb{1}}-}
\arrayrulecolor{black}
\end{tabularx}\par\end{raggedright}
\end{table}
 

\FloatBarrier

We have now seen three ways to call \texttt{set\_*} functions in
huxtable:

\begin{itemize}
\tightlist
\item
  With four arguments, like
  \texttt{set\_property(hux\_object,\ rows,\ cols,\ value)};
\item
  With two arguments, like \texttt{set\_property(hux\_object,\ value)}
  to set a property everywhere;
\item
  With three arguments, like
  \texttt{set\_property(hux\_object,\ where(condition),\ value)} to set
  a property for specific cells.
\end{itemize}

The second argument of the three-argument version must return a 2-column
matrix. Each row of the matrix gives one cell. \texttt{where()} does
this for you: it takes a logical matrix argument and returns the rows
and columns where a condition is \texttt{TRUE}. It's easiest to show
this with an example:

\begin{Shaded}
\begin{Highlighting}[]
\NormalTok{m <-}\StringTok{ }\KeywordTok{matrix}\NormalTok{(}\KeywordTok{c}\NormalTok{(}\StringTok{'dog'}\NormalTok{, }\StringTok{'cat'}\NormalTok{, }\StringTok{'dog'}\NormalTok{, }\StringTok{'dog'}\NormalTok{, }\StringTok{'cat'}\NormalTok{, }\StringTok{'cat'}\NormalTok{, }\StringTok{'cat'}\NormalTok{, }\StringTok{'dog'}\NormalTok{), }\DecValTok{4}\NormalTok{, }\DecValTok{2}\NormalTok{)}
\NormalTok{m}
\end{Highlighting}
\end{Shaded}

\begin{verbatim}
##      [,1]  [,2] 
## [1,] "dog" "cat"
## [2,] "cat" "cat"
## [3,] "dog" "cat"
## [4,] "dog" "dog"
\end{verbatim}

\begin{Shaded}
\begin{Highlighting}[]
\KeywordTok{where}\NormalTok{(m }\OperatorTok{==}\StringTok{ 'dog'}\NormalTok{) }\CommentTok{# m is equal to 'dog' in cells (1, 1), (3, 1), (4, 1) and (4, 2):}
\end{Highlighting}
\end{Shaded}

\begin{verbatim}
##      row col
## [1,]   1   1
## [2,]   3   1
## [3,]   4   1
## [4,]   4   2
\end{verbatim}

\FloatBarrier

\texttt{set\_*} functions have one more optional argument, the
\texttt{byrow} argument, which is \texttt{FALSE} by default. If you set
a single pattern for many cells, you may want the pattern to fill the
matrix by column or by row. The default fills the pattern in going down
columns. If you set \texttt{byrow\ =\ TRUE}, the pattern goes across
rows instead. (This is a bit confusing: typically,
\texttt{byrow\ =\ TRUE} means that the \emph{columns} will all look the
same. But it works the same way as the \texttt{byrow} argument to
\texttt{matrix()}.)

\begin{Shaded}
\begin{Highlighting}[]
\NormalTok{color_demo <-}\StringTok{ }\KeywordTok{matrix}\NormalTok{(}\StringTok{'text'}\NormalTok{, }\DecValTok{7}\NormalTok{, }\DecValTok{7}\NormalTok{)}
\NormalTok{rainbow <-}\StringTok{ }\KeywordTok{c}\NormalTok{(}\StringTok{'red'}\NormalTok{, }\StringTok{'orange'}\NormalTok{, }\StringTok{'yellow'}\NormalTok{, }\StringTok{'green'}\NormalTok{, }\StringTok{'blue'}\NormalTok{, }\StringTok{'turquoise'}\NormalTok{, }\StringTok{'violet'}\NormalTok{)}
\NormalTok{color_demo <-}\StringTok{ }\KeywordTok{as_hux}\NormalTok{(color_demo)                  }\OperatorTok\StringTok{ }
\StringTok{      }\KeywordTok{set_text_color}\NormalTok{(rainbow)                     }\OperatorTok\StringTok{ }\CommentTok{# text rainbow down columns}
\StringTok{      }\KeywordTok{set_background_color}\NormalTok{(rainbow, }\DataTypeTok{byrow =} \OtherTok{TRUE}\NormalTok{) }\OperatorTok\StringTok{ }\CommentTok{# background color rainbow along rows}
\StringTok{      }\KeywordTok{set_all_borders}\NormalTok{(}\DecValTok{1}\NormalTok{)                          }\OperatorTok\StringTok{ }
\StringTok{      }\KeywordTok{set_all_border_colors}\NormalTok{(}\StringTok{'white'}\NormalTok{)}
\NormalTok{color_demo}
\end{Highlighting}
\end{Shaded}

 \begin{table}[h]
\centering
    \providecommand{\huxb}[2][0,0,0]{\arrayrulecolor[RGB]{#1}\global\arrayrulewidth=#2pt}
    \providecommand{\huxvb}[2][0,0,0]{\color[RGB]{#1}\vrule width #2pt}
    \providecommand{\huxtpad}[1]{\rule{0pt}{\baselineskip+#1}}
    \providecommand{\huxbpad}[1]{\rule[-#1]{0pt}{#1}}
  \begin{tabularx}{0.5\textwidth}{p{0.0714285714285714\textwidth} p{0.0714285714285714\textwidth} p{0.0714285714285714\textwidth} p{0.0714285714285714\textwidth} p{0.0714285714285714\textwidth} p{0.0714285714285714\textwidth} p{0.0714285714285714\textwidth}}


\hhline{>{\huxb[255, 255, 255]{1}}->{\huxb[255, 255, 255]{1}}->{\huxb[255, 255, 255]{1}}->{\huxb[255, 255, 255]{1}}->{\huxb[255, 255, 255]{1}}->{\huxb[255, 255, 255]{1}}->{\huxb[255, 255, 255]{1}}-}
\arrayrulecolor{black}

\multicolumn{1}{!{\huxvb[255, 255, 255]{1}}l!{\huxvb[255, 255, 255]{1}}}{\cellcolor[RGB]{255, 0, 0}\huxtpad{4pt}\raggedright \textcolor[RGB]{255, 0, 0}{text}\huxbpad{4pt}} &
\multicolumn{1}{l!{\huxvb[255, 255, 255]{1}}}{\cellcolor[RGB]{255, 165, 0}\huxtpad{4pt}\raggedright \textcolor[RGB]{255, 0, 0}{text}\huxbpad{4pt}} &
\multicolumn{1}{l!{\huxvb[255, 255, 255]{1}}}{\cellcolor[RGB]{255, 255, 0}\huxtpad{4pt}\raggedright \textcolor[RGB]{255, 0, 0}{text}\huxbpad{4pt}} &
\multicolumn{1}{l!{\huxvb[255, 255, 255]{1}}}{\cellcolor[RGB]{0, 255, 0}\huxtpad{4pt}\raggedright \textcolor[RGB]{255, 0, 0}{text}\huxbpad{4pt}} &
\multicolumn{1}{l!{\huxvb[255, 255, 255]{1}}}{\cellcolor[RGB]{0, 0, 255}\huxtpad{4pt}\raggedright \textcolor[RGB]{255, 0, 0}{text}\huxbpad{4pt}} &
\multicolumn{1}{l!{\huxvb[255, 255, 255]{1}}}{\cellcolor[RGB]{64, 224, 208}\huxtpad{4pt}\raggedright \textcolor[RGB]{255, 0, 0}{text}\huxbpad{4pt}} &
\multicolumn{1}{l!{\huxvb[255, 255, 255]{1}}}{\cellcolor[RGB]{238, 130, 238}\huxtpad{4pt}\raggedright \textcolor[RGB]{255, 0, 0}{text}\huxbpad{4pt}} \tabularnewline[-0.5pt]


\hhline{>{\huxb[255, 255, 255]{1}}->{\huxb[255, 255, 255]{1}}->{\huxb[255, 255, 255]{1}}->{\huxb[255, 255, 255]{1}}->{\huxb[255, 255, 255]{1}}->{\huxb[255, 255, 255]{1}}->{\huxb[255, 255, 255]{1}}-}
\arrayrulecolor{black}

\multicolumn{1}{!{\huxvb[255, 255, 255]{1}}l!{\huxvb[255, 255, 255]{1}}}{\cellcolor[RGB]{255, 0, 0}\huxtpad{4pt}\raggedright \textcolor[RGB]{255, 165, 0}{text}\huxbpad{4pt}} &
\multicolumn{1}{l!{\huxvb[255, 255, 255]{1}}}{\cellcolor[RGB]{255, 165, 0}\huxtpad{4pt}\raggedright \textcolor[RGB]{255, 165, 0}{text}\huxbpad{4pt}} &
\multicolumn{1}{l!{\huxvb[255, 255, 255]{1}}}{\cellcolor[RGB]{255, 255, 0}\huxtpad{4pt}\raggedright \textcolor[RGB]{255, 165, 0}{text}\huxbpad{4pt}} &
\multicolumn{1}{l!{\huxvb[255, 255, 255]{1}}}{\cellcolor[RGB]{0, 255, 0}\huxtpad{4pt}\raggedright \textcolor[RGB]{255, 165, 0}{text}\huxbpad{4pt}} &
\multicolumn{1}{l!{\huxvb[255, 255, 255]{1}}}{\cellcolor[RGB]{0, 0, 255}\huxtpad{4pt}\raggedright \textcolor[RGB]{255, 165, 0}{text}\huxbpad{4pt}} &
\multicolumn{1}{l!{\huxvb[255, 255, 255]{1}}}{\cellcolor[RGB]{64, 224, 208}\huxtpad{4pt}\raggedright \textcolor[RGB]{255, 165, 0}{text}\huxbpad{4pt}} &
\multicolumn{1}{l!{\huxvb[255, 255, 255]{1}}}{\cellcolor[RGB]{238, 130, 238}\huxtpad{4pt}\raggedright \textcolor[RGB]{255, 165, 0}{text}\huxbpad{4pt}} \tabularnewline[-0.5pt]


\hhline{>{\huxb[255, 255, 255]{1}}->{\huxb[255, 255, 255]{1}}->{\huxb[255, 255, 255]{1}}->{\huxb[255, 255, 255]{1}}->{\huxb[255, 255, 255]{1}}->{\huxb[255, 255, 255]{1}}->{\huxb[255, 255, 255]{1}}-}
\arrayrulecolor{black}

\multicolumn{1}{!{\huxvb[255, 255, 255]{1}}l!{\huxvb[255, 255, 255]{1}}}{\cellcolor[RGB]{255, 0, 0}\huxtpad{4pt}\raggedright \textcolor[RGB]{255, 255, 0}{text}\huxbpad{4pt}} &
\multicolumn{1}{l!{\huxvb[255, 255, 255]{1}}}{\cellcolor[RGB]{255, 165, 0}\huxtpad{4pt}\raggedright \textcolor[RGB]{255, 255, 0}{text}\huxbpad{4pt}} &
\multicolumn{1}{l!{\huxvb[255, 255, 255]{1}}}{\cellcolor[RGB]{255, 255, 0}\huxtpad{4pt}\raggedright \textcolor[RGB]{255, 255, 0}{text}\huxbpad{4pt}} &
\multicolumn{1}{l!{\huxvb[255, 255, 255]{1}}}{\cellcolor[RGB]{0, 255, 0}\huxtpad{4pt}\raggedright \textcolor[RGB]{255, 255, 0}{text}\huxbpad{4pt}} &
\multicolumn{1}{l!{\huxvb[255, 255, 255]{1}}}{\cellcolor[RGB]{0, 0, 255}\huxtpad{4pt}\raggedright \textcolor[RGB]{255, 255, 0}{text}\huxbpad{4pt}} &
\multicolumn{1}{l!{\huxvb[255, 255, 255]{1}}}{\cellcolor[RGB]{64, 224, 208}\huxtpad{4pt}\raggedright \textcolor[RGB]{255, 255, 0}{text}\huxbpad{4pt}} &
\multicolumn{1}{l!{\huxvb[255, 255, 255]{1}}}{\cellcolor[RGB]{238, 130, 238}\huxtpad{4pt}\raggedright \textcolor[RGB]{255, 255, 0}{text}\huxbpad{4pt}} \tabularnewline[-0.5pt]


\hhline{>{\huxb[255, 255, 255]{1}}->{\huxb[255, 255, 255]{1}}->{\huxb[255, 255, 255]{1}}->{\huxb[255, 255, 255]{1}}->{\huxb[255, 255, 255]{1}}->{\huxb[255, 255, 255]{1}}->{\huxb[255, 255, 255]{1}}-}
\arrayrulecolor{black}

\multicolumn{1}{!{\huxvb[255, 255, 255]{1}}l!{\huxvb[255, 255, 255]{1}}}{\cellcolor[RGB]{255, 0, 0}\huxtpad{4pt}\raggedright \textcolor[RGB]{0, 255, 0}{text}\huxbpad{4pt}} &
\multicolumn{1}{l!{\huxvb[255, 255, 255]{1}}}{\cellcolor[RGB]{255, 165, 0}\huxtpad{4pt}\raggedright \textcolor[RGB]{0, 255, 0}{text}\huxbpad{4pt}} &
\multicolumn{1}{l!{\huxvb[255, 255, 255]{1}}}{\cellcolor[RGB]{255, 255, 0}\huxtpad{4pt}\raggedright \textcolor[RGB]{0, 255, 0}{text}\huxbpad{4pt}} &
\multicolumn{1}{l!{\huxvb[255, 255, 255]{1}}}{\cellcolor[RGB]{0, 255, 0}\huxtpad{4pt}\raggedright \textcolor[RGB]{0, 255, 0}{text}\huxbpad{4pt}} &
\multicolumn{1}{l!{\huxvb[255, 255, 255]{1}}}{\cellcolor[RGB]{0, 0, 255}\huxtpad{4pt}\raggedright \textcolor[RGB]{0, 255, 0}{text}\huxbpad{4pt}} &
\multicolumn{1}{l!{\huxvb[255, 255, 255]{1}}}{\cellcolor[RGB]{64, 224, 208}\huxtpad{4pt}\raggedright \textcolor[RGB]{0, 255, 0}{text}\huxbpad{4pt}} &
\multicolumn{1}{l!{\huxvb[255, 255, 255]{1}}}{\cellcolor[RGB]{238, 130, 238}\huxtpad{4pt}\raggedright \textcolor[RGB]{0, 255, 0}{text}\huxbpad{4pt}} \tabularnewline[-0.5pt]


\hhline{>{\huxb[255, 255, 255]{1}}->{\huxb[255, 255, 255]{1}}->{\huxb[255, 255, 255]{1}}->{\huxb[255, 255, 255]{1}}->{\huxb[255, 255, 255]{1}}->{\huxb[255, 255, 255]{1}}->{\huxb[255, 255, 255]{1}}-}
\arrayrulecolor{black}

\multicolumn{1}{!{\huxvb[255, 255, 255]{1}}l!{\huxvb[255, 255, 255]{1}}}{\cellcolor[RGB]{255, 0, 0}\huxtpad{4pt}\raggedright \textcolor[RGB]{0, 0, 255}{text}\huxbpad{4pt}} &
\multicolumn{1}{l!{\huxvb[255, 255, 255]{1}}}{\cellcolor[RGB]{255, 165, 0}\huxtpad{4pt}\raggedright \textcolor[RGB]{0, 0, 255}{text}\huxbpad{4pt}} &
\multicolumn{1}{l!{\huxvb[255, 255, 255]{1}}}{\cellcolor[RGB]{255, 255, 0}\huxtpad{4pt}\raggedright \textcolor[RGB]{0, 0, 255}{text}\huxbpad{4pt}} &
\multicolumn{1}{l!{\huxvb[255, 255, 255]{1}}}{\cellcolor[RGB]{0, 255, 0}\huxtpad{4pt}\raggedright \textcolor[RGB]{0, 0, 255}{text}\huxbpad{4pt}} &
\multicolumn{1}{l!{\huxvb[255, 255, 255]{1}}}{\cellcolor[RGB]{0, 0, 255}\huxtpad{4pt}\raggedright \textcolor[RGB]{0, 0, 255}{text}\huxbpad{4pt}} &
\multicolumn{1}{l!{\huxvb[255, 255, 255]{1}}}{\cellcolor[RGB]{64, 224, 208}\huxtpad{4pt}\raggedright \textcolor[RGB]{0, 0, 255}{text}\huxbpad{4pt}} &
\multicolumn{1}{l!{\huxvb[255, 255, 255]{1}}}{\cellcolor[RGB]{238, 130, 238}\huxtpad{4pt}\raggedright \textcolor[RGB]{0, 0, 255}{text}\huxbpad{4pt}} \tabularnewline[-0.5pt]


\hhline{>{\huxb[255, 255, 255]{1}}->{\huxb[255, 255, 255]{1}}->{\huxb[255, 255, 255]{1}}->{\huxb[255, 255, 255]{1}}->{\huxb[255, 255, 255]{1}}->{\huxb[255, 255, 255]{1}}->{\huxb[255, 255, 255]{1}}-}
\arrayrulecolor{black}

\multicolumn{1}{!{\huxvb[255, 255, 255]{1}}l!{\huxvb[255, 255, 255]{1}}}{\cellcolor[RGB]{255, 0, 0}\huxtpad{4pt}\raggedright \textcolor[RGB]{64, 224, 208}{text}\huxbpad{4pt}} &
\multicolumn{1}{l!{\huxvb[255, 255, 255]{1}}}{\cellcolor[RGB]{255, 165, 0}\huxtpad{4pt}\raggedright \textcolor[RGB]{64, 224, 208}{text}\huxbpad{4pt}} &
\multicolumn{1}{l!{\huxvb[255, 255, 255]{1}}}{\cellcolor[RGB]{255, 255, 0}\huxtpad{4pt}\raggedright \textcolor[RGB]{64, 224, 208}{text}\huxbpad{4pt}} &
\multicolumn{1}{l!{\huxvb[255, 255, 255]{1}}}{\cellcolor[RGB]{0, 255, 0}\huxtpad{4pt}\raggedright \textcolor[RGB]{64, 224, 208}{text}\huxbpad{4pt}} &
\multicolumn{1}{l!{\huxvb[255, 255, 255]{1}}}{\cellcolor[RGB]{0, 0, 255}\huxtpad{4pt}\raggedright \textcolor[RGB]{64, 224, 208}{text}\huxbpad{4pt}} &
\multicolumn{1}{l!{\huxvb[255, 255, 255]{1}}}{\cellcolor[RGB]{64, 224, 208}\huxtpad{4pt}\raggedright \textcolor[RGB]{64, 224, 208}{text}\huxbpad{4pt}} &
\multicolumn{1}{l!{\huxvb[255, 255, 255]{1}}}{\cellcolor[RGB]{238, 130, 238}\huxtpad{4pt}\raggedright \textcolor[RGB]{64, 224, 208}{text}\huxbpad{4pt}} \tabularnewline[-0.5pt]


\hhline{>{\huxb[255, 255, 255]{1}}->{\huxb[255, 255, 255]{1}}->{\huxb[255, 255, 255]{1}}->{\huxb[255, 255, 255]{1}}->{\huxb[255, 255, 255]{1}}->{\huxb[255, 255, 255]{1}}->{\huxb[255, 255, 255]{1}}-}
\arrayrulecolor{black}

\multicolumn{1}{!{\huxvb[255, 255, 255]{1}}l!{\huxvb[255, 255, 255]{1}}}{\cellcolor[RGB]{255, 0, 0}\huxtpad{4pt}\raggedright \textcolor[RGB]{238, 130, 238}{text}\huxbpad{4pt}} &
\multicolumn{1}{l!{\huxvb[255, 255, 255]{1}}}{\cellcolor[RGB]{255, 165, 0}\huxtpad{4pt}\raggedright \textcolor[RGB]{238, 130, 238}{text}\huxbpad{4pt}} &
\multicolumn{1}{l!{\huxvb[255, 255, 255]{1}}}{\cellcolor[RGB]{255, 255, 0}\huxtpad{4pt}\raggedright \textcolor[RGB]{238, 130, 238}{text}\huxbpad{4pt}} &
\multicolumn{1}{l!{\huxvb[255, 255, 255]{1}}}{\cellcolor[RGB]{0, 255, 0}\huxtpad{4pt}\raggedright \textcolor[RGB]{238, 130, 238}{text}\huxbpad{4pt}} &
\multicolumn{1}{l!{\huxvb[255, 255, 255]{1}}}{\cellcolor[RGB]{0, 0, 255}\huxtpad{4pt}\raggedright \textcolor[RGB]{238, 130, 238}{text}\huxbpad{4pt}} &
\multicolumn{1}{l!{\huxvb[255, 255, 255]{1}}}{\cellcolor[RGB]{64, 224, 208}\huxtpad{4pt}\raggedright \textcolor[RGB]{238, 130, 238}{text}\huxbpad{4pt}} &
\multicolumn{1}{l!{\huxvb[255, 255, 255]{1}}}{\cellcolor[RGB]{238, 130, 238}\huxtpad{4pt}\raggedright \textcolor[RGB]{238, 130, 238}{text}\huxbpad{4pt}} \tabularnewline[-0.5pt]


\hhline{>{\huxb[255, 255, 255]{1}}->{\huxb[255, 255, 255]{1}}->{\huxb[255, 255, 255]{1}}->{\huxb[255, 255, 255]{1}}->{\huxb[255, 255, 255]{1}}->{\huxb[255, 255, 255]{1}}->{\huxb[255, 255, 255]{1}}-}
\arrayrulecolor{black}
\end{tabularx}
\end{table}
 

\FloatBarrier

\hypertarget{creating-a-regression-table}{%
\section{Creating a regression
table}\label{creating-a-regression-table}}

A common task for scientists is to create a table of regressions. The
function \texttt{huxreg} does this for you. Here's a quick example:

\begin{Shaded}
\begin{Highlighting}[]
\KeywordTok{data}\NormalTok{(diamonds, }\DataTypeTok{package =} \StringTok{'ggplot2'}\NormalTok{)}

\NormalTok{lm1 <-}\StringTok{ }\KeywordTok{lm}\NormalTok{(price }\OperatorTok{~}\StringTok{ }\NormalTok{carat, diamonds)}
\NormalTok{lm2 <-}\StringTok{ }\KeywordTok{lm}\NormalTok{(price }\OperatorTok{~}\StringTok{ }\NormalTok{depth, diamonds)}
\NormalTok{lm3 <-}\StringTok{ }\KeywordTok{lm}\NormalTok{(price }\OperatorTok{~}\StringTok{ }\NormalTok{carat }\OperatorTok{+}\StringTok{ }\NormalTok{depth, diamonds)}

\KeywordTok{huxreg}\NormalTok{(lm1, lm2, lm3)}
\end{Highlighting}
\end{Shaded}

 \begin{table}[h]
\centering
    \providecommand{\huxb}[2][0,0,0]{\arrayrulecolor[RGB]{#1}\global\arrayrulewidth=#2pt}
    \providecommand{\huxvb}[2][0,0,0]{\color[RGB]{#1}\vrule width #2pt}
    \providecommand{\huxtpad}[1]{\rule{0pt}{\baselineskip+#1}}
    \providecommand{\huxbpad}[1]{\rule[-#1]{0pt}{#1}}
  \begin{tabularx}{0.5\textwidth}{p{0.125\textwidth} p{0.125\textwidth} p{0.125\textwidth} p{0.125\textwidth}}


\hhline{>{\huxb{0.8}}->{\huxb{0.8}}->{\huxb{0.8}}->{\huxb{0.8}}-}
\arrayrulecolor{black}

\multicolumn{1}{!{\huxvb{0}}c!{\huxvb{0}}}{\huxtpad{4pt}\centering \huxbpad{4pt}} &
\multicolumn{1}{c!{\huxvb{0}}}{\huxtpad{4pt}\centering (1)\huxbpad{4pt}} &
\multicolumn{1}{c!{\huxvb{0}}}{\huxtpad{4pt}\centering (2)\huxbpad{4pt}} &
\multicolumn{1}{c!{\huxvb{0}}}{\huxtpad{4pt}\centering (3)\huxbpad{4pt}} \tabularnewline[-0.5pt]


\hhline{>{\huxb[255, 255, 255]{0.4}}->{\huxb{0.4}}->{\huxb{0.4}}->{\huxb{0.4}}-}
\arrayrulecolor{black}

\multicolumn{1}{!{\huxvb{0}}l!{\huxvb{0}}}{\huxtpad{4pt}\raggedright (Intercept)\huxbpad{4pt}} &
\multicolumn{1}{r!{\huxvb{0}}}{\huxtpad{4pt}\raggedleft -2256.361 ***\huxbpad{4pt}} &
\multicolumn{1}{r!{\huxvb{0}}}{\huxtpad{4pt}\raggedleft 5763.668 ***\huxbpad{4pt}} &
\multicolumn{1}{r!{\huxvb{0}}}{\huxtpad{4pt}\raggedleft 4045.333 ***\huxbpad{4pt}} \tabularnewline[-0.5pt]


\hhline{}
\arrayrulecolor{black}

\multicolumn{1}{!{\huxvb{0}}l!{\huxvb{0}}}{\huxtpad{4pt}\raggedright \huxbpad{4pt}} &
\multicolumn{1}{r!{\huxvb{0}}}{\huxtpad{4pt}\raggedleft (13.055)~~~\huxbpad{4pt}} &
\multicolumn{1}{r!{\huxvb{0}}}{\huxtpad{4pt}\raggedleft (740.556)~~~\huxbpad{4pt}} &
\multicolumn{1}{r!{\huxvb{0}}}{\huxtpad{4pt}\raggedleft (286.205)~~~\huxbpad{4pt}} \tabularnewline[-0.5pt]


\hhline{}
\arrayrulecolor{black}

\multicolumn{1}{!{\huxvb{0}}l!{\huxvb{0}}}{\huxtpad{4pt}\raggedright carat\huxbpad{4pt}} &
\multicolumn{1}{r!{\huxvb{0}}}{\huxtpad{4pt}\raggedleft 7756.426 ***\huxbpad{4pt}} &
\multicolumn{1}{r!{\huxvb{0}}}{\huxtpad{4pt}\raggedleft ~~~~~~~~\huxbpad{4pt}} &
\multicolumn{1}{r!{\huxvb{0}}}{\huxtpad{4pt}\raggedleft 7765.141 ***\huxbpad{4pt}} \tabularnewline[-0.5pt]


\hhline{}
\arrayrulecolor{black}

\multicolumn{1}{!{\huxvb{0}}l!{\huxvb{0}}}{\huxtpad{4pt}\raggedright \huxbpad{4pt}} &
\multicolumn{1}{r!{\huxvb{0}}}{\huxtpad{4pt}\raggedleft (14.067)~~~\huxbpad{4pt}} &
\multicolumn{1}{r!{\huxvb{0}}}{\huxtpad{4pt}\raggedleft ~~~~~~~~\huxbpad{4pt}} &
\multicolumn{1}{r!{\huxvb{0}}}{\huxtpad{4pt}\raggedleft (14.009)~~~\huxbpad{4pt}} \tabularnewline[-0.5pt]


\hhline{}
\arrayrulecolor{black}

\multicolumn{1}{!{\huxvb{0}}l!{\huxvb{0}}}{\huxtpad{4pt}\raggedright depth\huxbpad{4pt}} &
\multicolumn{1}{r!{\huxvb{0}}}{\huxtpad{4pt}\raggedleft ~~~~~~~~\huxbpad{4pt}} &
\multicolumn{1}{r!{\huxvb{0}}}{\huxtpad{4pt}\raggedleft -29.650 *~~\huxbpad{4pt}} &
\multicolumn{1}{r!{\huxvb{0}}}{\huxtpad{4pt}\raggedleft -102.165 ***\huxbpad{4pt}} \tabularnewline[-0.5pt]


\hhline{}
\arrayrulecolor{black}

\multicolumn{1}{!{\huxvb{0}}l!{\huxvb{0}}}{\huxtpad{4pt}\raggedright \huxbpad{4pt}} &
\multicolumn{1}{r!{\huxvb{0}}}{\huxtpad{4pt}\raggedleft ~~~~~~~~\huxbpad{4pt}} &
\multicolumn{1}{r!{\huxvb{0}}}{\huxtpad{4pt}\raggedleft (11.990)~~~\huxbpad{4pt}} &
\multicolumn{1}{r!{\huxvb{0}}}{\huxtpad{4pt}\raggedleft (4.635)~~~\huxbpad{4pt}} \tabularnewline[-0.5pt]


\hhline{>{\huxb[255, 255, 255]{0.4}}->{\huxb{0.4}}->{\huxb{0.4}}->{\huxb{0.4}}-}
\arrayrulecolor{black}

\multicolumn{1}{!{\huxvb{0}}l!{\huxvb{0}}}{\huxtpad{4pt}\raggedright N\huxbpad{4pt}} &
\multicolumn{1}{r!{\huxvb{0}}}{\huxtpad{4pt}\raggedleft 53940~~~~~~~~\huxbpad{4pt}} &
\multicolumn{1}{r!{\huxvb{0}}}{\huxtpad{4pt}\raggedleft 53940~~~~~~~~\huxbpad{4pt}} &
\multicolumn{1}{r!{\huxvb{0}}}{\huxtpad{4pt}\raggedleft 53940~~~~~~~~\huxbpad{4pt}} \tabularnewline[-0.5pt]


\hhline{}
\arrayrulecolor{black}

\multicolumn{1}{!{\huxvb{0}}l!{\huxvb{0}}}{\huxtpad{4pt}\raggedright R2\huxbpad{4pt}} &
\multicolumn{1}{r!{\huxvb{0}}}{\huxtpad{4pt}\raggedleft 0.849~~~~\huxbpad{4pt}} &
\multicolumn{1}{r!{\huxvb{0}}}{\huxtpad{4pt}\raggedleft 0.000~~~~\huxbpad{4pt}} &
\multicolumn{1}{r!{\huxvb{0}}}{\huxtpad{4pt}\raggedleft 0.851~~~~\huxbpad{4pt}} \tabularnewline[-0.5pt]


\hhline{}
\arrayrulecolor{black}

\multicolumn{1}{!{\huxvb{0}}l!{\huxvb{0}}}{\huxtpad{4pt}\raggedright logLik\huxbpad{4pt}} &
\multicolumn{1}{r!{\huxvb{0}}}{\huxtpad{4pt}\raggedleft -472730.266~~~~\huxbpad{4pt}} &
\multicolumn{1}{r!{\huxvb{0}}}{\huxtpad{4pt}\raggedleft -523772.431~~~~\huxbpad{4pt}} &
\multicolumn{1}{r!{\huxvb{0}}}{\huxtpad{4pt}\raggedleft -472488.441~~~~\huxbpad{4pt}} \tabularnewline[-0.5pt]


\hhline{}
\arrayrulecolor{black}

\multicolumn{1}{!{\huxvb{0}}l!{\huxvb{0}}}{\huxtpad{4pt}\raggedright AIC\huxbpad{4pt}} &
\multicolumn{1}{r!{\huxvb{0}}}{\huxtpad{4pt}\raggedleft 945466.532~~~~\huxbpad{4pt}} &
\multicolumn{1}{r!{\huxvb{0}}}{\huxtpad{4pt}\raggedleft 1047550.862~~~~\huxbpad{4pt}} &
\multicolumn{1}{r!{\huxvb{0}}}{\huxtpad{4pt}\raggedleft 944984.882~~~~\huxbpad{4pt}} \tabularnewline[-0.5pt]


\hhline{>{\huxb{0.8}}->{\huxb{0.8}}->{\huxb{0.8}}->{\huxb{0.8}}-}
\arrayrulecolor{black}

\multicolumn{4}{!{\huxvb{0}}p{0.5\textwidth+6\tabcolsep}!{\huxvb{0}}}{\parbox[b]{0.5\textwidth+6\tabcolsep-4pt-4pt}{\huxtpad{4pt}\raggedright  *** p $<$ 0.001;  ** p $<$ 0.01;  * p $<$ 0.05.\huxbpad{4pt}}} \tabularnewline[-0.5pt]


\hhline{}
\arrayrulecolor{black}
\end{tabularx}
\end{table}
 

\FloatBarrier

For more information see the \texttt{huxreg} vignette, available online
in \href{https://hughjonesd.github.io/huxtable/huxreg.html}{HTML} or
\href{https://hughjonesd.github.io/huxtable/huxreg.pdf}{PDF} or in R via
\texttt{vignette(\textquotesingle{}huxreg\textquotesingle{})}.

\hypertarget{output-to-different-formats}{%
\section{Output to different
formats}\label{output-to-different-formats}}

\hypertarget{automatic-pretty-printing-of-data-frames}{%
\subsection{Automatic pretty-printing of data
frames}\label{automatic-pretty-printing-of-data-frames}}

If you load huxtable within a knitr document, it will automatically
format data frames for you by installing a
\texttt{knit\_print.data\_frame} command.

\begin{Shaded}
\begin{Highlighting}[]
\KeywordTok{head}\NormalTok{(mtcars)}
\end{Highlighting}
\end{Shaded}

 \begin{table}[h]
\begin{raggedright}
    \providecommand{\huxb}[2][0,0,0]{\arrayrulecolor[RGB]{#1}\global\arrayrulewidth=#2pt}
    \providecommand{\huxvb}[2][0,0,0]{\color[RGB]{#1}\vrule width #2pt}
    \providecommand{\huxtpad}[1]{\rule{0pt}{\baselineskip+#1}}
    \providecommand{\huxbpad}[1]{\rule[-#1]{0pt}{#1}}
  \begin{tabularx}{0.322222222222222\textwidth}{p{0.0292929292929293\textwidth} p{0.0292929292929293\textwidth} p{0.0292929292929293\textwidth} p{0.0292929292929293\textwidth} p{0.0292929292929293\textwidth} p{0.0292929292929293\textwidth} p{0.0292929292929293\textwidth} p{0.0292929292929293\textwidth} p{0.0292929292929293\textwidth} p{0.0292929292929293\textwidth} p{0.0292929292929293\textwidth}}


\hhline{>{\huxb{0.4}}->{\huxb{0.4}}->{\huxb{0.4}}->{\huxb{0.4}}->{\huxb{0.4}}->{\huxb{0.4}}->{\huxb{0.4}}->{\huxb{0.4}}->{\huxb{0.4}}->{\huxb{0.4}}->{\huxb{0.4}}-}
\arrayrulecolor{black}

\multicolumn{1}{!{\huxvb{0.4}}r!{\huxvb{0}}}{\huxtpad{4pt}\raggedleft \textbf{mpg}\huxbpad{4pt}} &
\multicolumn{1}{r!{\huxvb{0}}}{\huxtpad{4pt}\raggedleft \textbf{cyl}\huxbpad{4pt}} &
\multicolumn{1}{r!{\huxvb{0}}}{\huxtpad{4pt}\raggedleft \textbf{disp}\huxbpad{4pt}} &
\multicolumn{1}{r!{\huxvb{0}}}{\huxtpad{4pt}\raggedleft \textbf{hp}\huxbpad{4pt}} &
\multicolumn{1}{r!{\huxvb{0}}}{\huxtpad{4pt}\raggedleft \textbf{drat}\huxbpad{4pt}} &
\multicolumn{1}{r!{\huxvb{0}}}{\huxtpad{4pt}\raggedleft \textbf{wt}\huxbpad{4pt}} &
\multicolumn{1}{r!{\huxvb{0}}}{\huxtpad{4pt}\raggedleft \textbf{qsec}\huxbpad{4pt}} &
\multicolumn{1}{r!{\huxvb{0}}}{\huxtpad{4pt}\raggedleft \textbf{vs}\huxbpad{4pt}} &
\multicolumn{1}{r!{\huxvb{0}}}{\huxtpad{4pt}\raggedleft \textbf{am}\huxbpad{4pt}} &
\multicolumn{1}{r!{\huxvb{0}}}{\huxtpad{4pt}\raggedleft \textbf{gear}\huxbpad{4pt}} &
\multicolumn{1}{r!{\huxvb{0.4}}}{\huxtpad{4pt}\raggedleft \textbf{carb}\huxbpad{4pt}} \tabularnewline[-0.5pt]


\hhline{>{\huxb{0.4}}->{\huxb{0.4}}->{\huxb{0.4}}->{\huxb{0.4}}->{\huxb{0.4}}->{\huxb{0.4}}->{\huxb{0.4}}->{\huxb{0.4}}->{\huxb{0.4}}->{\huxb{0.4}}->{\huxb{0.4}}-}
\arrayrulecolor{black}

\multicolumn{1}{!{\huxvb{0.4}}r!{\huxvb{0}}}{\cellcolor[RGB]{242, 242, 242}\huxtpad{4pt}\raggedleft 21~~\huxbpad{4pt}} &
\multicolumn{1}{r!{\huxvb{0}}}{\cellcolor[RGB]{242, 242, 242}\huxtpad{4pt}\raggedleft 6\huxbpad{4pt}} &
\multicolumn{1}{r!{\huxvb{0}}}{\cellcolor[RGB]{242, 242, 242}\huxtpad{4pt}\raggedleft 160\huxbpad{4pt}} &
\multicolumn{1}{r!{\huxvb{0}}}{\cellcolor[RGB]{242, 242, 242}\huxtpad{4pt}\raggedleft 110\huxbpad{4pt}} &
\multicolumn{1}{r!{\huxvb{0}}}{\cellcolor[RGB]{242, 242, 242}\huxtpad{4pt}\raggedleft 3.9~\huxbpad{4pt}} &
\multicolumn{1}{r!{\huxvb{0}}}{\cellcolor[RGB]{242, 242, 242}\huxtpad{4pt}\raggedleft 2.62\huxbpad{4pt}} &
\multicolumn{1}{r!{\huxvb{0}}}{\cellcolor[RGB]{242, 242, 242}\huxtpad{4pt}\raggedleft 16.5\huxbpad{4pt}} &
\multicolumn{1}{r!{\huxvb{0}}}{\cellcolor[RGB]{242, 242, 242}\huxtpad{4pt}\raggedleft 0\huxbpad{4pt}} &
\multicolumn{1}{r!{\huxvb{0}}}{\cellcolor[RGB]{242, 242, 242}\huxtpad{4pt}\raggedleft 1\huxbpad{4pt}} &
\multicolumn{1}{r!{\huxvb{0}}}{\cellcolor[RGB]{242, 242, 242}\huxtpad{4pt}\raggedleft 4\huxbpad{4pt}} &
\multicolumn{1}{r!{\huxvb{0.4}}}{\cellcolor[RGB]{242, 242, 242}\huxtpad{4pt}\raggedleft 4\huxbpad{4pt}} \tabularnewline[-0.5pt]


\hhline{>{\huxb{0.4}}|>{\huxb{0.4}}|}
\arrayrulecolor{black}

\multicolumn{1}{!{\huxvb{0.4}}r!{\huxvb{0}}}{\huxtpad{4pt}\raggedleft 21~~\huxbpad{4pt}} &
\multicolumn{1}{r!{\huxvb{0}}}{\huxtpad{4pt}\raggedleft 6\huxbpad{4pt}} &
\multicolumn{1}{r!{\huxvb{0}}}{\huxtpad{4pt}\raggedleft 160\huxbpad{4pt}} &
\multicolumn{1}{r!{\huxvb{0}}}{\huxtpad{4pt}\raggedleft 110\huxbpad{4pt}} &
\multicolumn{1}{r!{\huxvb{0}}}{\huxtpad{4pt}\raggedleft 3.9~\huxbpad{4pt}} &
\multicolumn{1}{r!{\huxvb{0}}}{\huxtpad{4pt}\raggedleft 2.88\huxbpad{4pt}} &
\multicolumn{1}{r!{\huxvb{0}}}{\huxtpad{4pt}\raggedleft 17~~\huxbpad{4pt}} &
\multicolumn{1}{r!{\huxvb{0}}}{\huxtpad{4pt}\raggedleft 0\huxbpad{4pt}} &
\multicolumn{1}{r!{\huxvb{0}}}{\huxtpad{4pt}\raggedleft 1\huxbpad{4pt}} &
\multicolumn{1}{r!{\huxvb{0}}}{\huxtpad{4pt}\raggedleft 4\huxbpad{4pt}} &
\multicolumn{1}{r!{\huxvb{0.4}}}{\huxtpad{4pt}\raggedleft 4\huxbpad{4pt}} \tabularnewline[-0.5pt]


\hhline{>{\huxb{0.4}}|>{\huxb{0.4}}|}
\arrayrulecolor{black}

\multicolumn{1}{!{\huxvb{0.4}}r!{\huxvb{0}}}{\cellcolor[RGB]{242, 242, 242}\huxtpad{4pt}\raggedleft 22.8\huxbpad{4pt}} &
\multicolumn{1}{r!{\huxvb{0}}}{\cellcolor[RGB]{242, 242, 242}\huxtpad{4pt}\raggedleft 4\huxbpad{4pt}} &
\multicolumn{1}{r!{\huxvb{0}}}{\cellcolor[RGB]{242, 242, 242}\huxtpad{4pt}\raggedleft 108\huxbpad{4pt}} &
\multicolumn{1}{r!{\huxvb{0}}}{\cellcolor[RGB]{242, 242, 242}\huxtpad{4pt}\raggedleft 93\huxbpad{4pt}} &
\multicolumn{1}{r!{\huxvb{0}}}{\cellcolor[RGB]{242, 242, 242}\huxtpad{4pt}\raggedleft 3.85\huxbpad{4pt}} &
\multicolumn{1}{r!{\huxvb{0}}}{\cellcolor[RGB]{242, 242, 242}\huxtpad{4pt}\raggedleft 2.32\huxbpad{4pt}} &
\multicolumn{1}{r!{\huxvb{0}}}{\cellcolor[RGB]{242, 242, 242}\huxtpad{4pt}\raggedleft 18.6\huxbpad{4pt}} &
\multicolumn{1}{r!{\huxvb{0}}}{\cellcolor[RGB]{242, 242, 242}\huxtpad{4pt}\raggedleft 1\huxbpad{4pt}} &
\multicolumn{1}{r!{\huxvb{0}}}{\cellcolor[RGB]{242, 242, 242}\huxtpad{4pt}\raggedleft 1\huxbpad{4pt}} &
\multicolumn{1}{r!{\huxvb{0}}}{\cellcolor[RGB]{242, 242, 242}\huxtpad{4pt}\raggedleft 4\huxbpad{4pt}} &
\multicolumn{1}{r!{\huxvb{0.4}}}{\cellcolor[RGB]{242, 242, 242}\huxtpad{4pt}\raggedleft 1\huxbpad{4pt}} \tabularnewline[-0.5pt]


\hhline{>{\huxb{0.4}}|>{\huxb{0.4}}|}
\arrayrulecolor{black}

\multicolumn{1}{!{\huxvb{0.4}}r!{\huxvb{0}}}{\huxtpad{4pt}\raggedleft 21.4\huxbpad{4pt}} &
\multicolumn{1}{r!{\huxvb{0}}}{\huxtpad{4pt}\raggedleft 6\huxbpad{4pt}} &
\multicolumn{1}{r!{\huxvb{0}}}{\huxtpad{4pt}\raggedleft 258\huxbpad{4pt}} &
\multicolumn{1}{r!{\huxvb{0}}}{\huxtpad{4pt}\raggedleft 110\huxbpad{4pt}} &
\multicolumn{1}{r!{\huxvb{0}}}{\huxtpad{4pt}\raggedleft 3.08\huxbpad{4pt}} &
\multicolumn{1}{r!{\huxvb{0}}}{\huxtpad{4pt}\raggedleft 3.21\huxbpad{4pt}} &
\multicolumn{1}{r!{\huxvb{0}}}{\huxtpad{4pt}\raggedleft 19.4\huxbpad{4pt}} &
\multicolumn{1}{r!{\huxvb{0}}}{\huxtpad{4pt}\raggedleft 1\huxbpad{4pt}} &
\multicolumn{1}{r!{\huxvb{0}}}{\huxtpad{4pt}\raggedleft 0\huxbpad{4pt}} &
\multicolumn{1}{r!{\huxvb{0}}}{\huxtpad{4pt}\raggedleft 3\huxbpad{4pt}} &
\multicolumn{1}{r!{\huxvb{0.4}}}{\huxtpad{4pt}\raggedleft 1\huxbpad{4pt}} \tabularnewline[-0.5pt]


\hhline{>{\huxb{0.4}}|>{\huxb{0.4}}|}
\arrayrulecolor{black}

\multicolumn{1}{!{\huxvb{0.4}}r!{\huxvb{0}}}{\cellcolor[RGB]{242, 242, 242}\huxtpad{4pt}\raggedleft 18.7\huxbpad{4pt}} &
\multicolumn{1}{r!{\huxvb{0}}}{\cellcolor[RGB]{242, 242, 242}\huxtpad{4pt}\raggedleft 8\huxbpad{4pt}} &
\multicolumn{1}{r!{\huxvb{0}}}{\cellcolor[RGB]{242, 242, 242}\huxtpad{4pt}\raggedleft 360\huxbpad{4pt}} &
\multicolumn{1}{r!{\huxvb{0}}}{\cellcolor[RGB]{242, 242, 242}\huxtpad{4pt}\raggedleft 175\huxbpad{4pt}} &
\multicolumn{1}{r!{\huxvb{0}}}{\cellcolor[RGB]{242, 242, 242}\huxtpad{4pt}\raggedleft 3.15\huxbpad{4pt}} &
\multicolumn{1}{r!{\huxvb{0}}}{\cellcolor[RGB]{242, 242, 242}\huxtpad{4pt}\raggedleft 3.44\huxbpad{4pt}} &
\multicolumn{1}{r!{\huxvb{0}}}{\cellcolor[RGB]{242, 242, 242}\huxtpad{4pt}\raggedleft 17~~\huxbpad{4pt}} &
\multicolumn{1}{r!{\huxvb{0}}}{\cellcolor[RGB]{242, 242, 242}\huxtpad{4pt}\raggedleft 0\huxbpad{4pt}} &
\multicolumn{1}{r!{\huxvb{0}}}{\cellcolor[RGB]{242, 242, 242}\huxtpad{4pt}\raggedleft 0\huxbpad{4pt}} &
\multicolumn{1}{r!{\huxvb{0}}}{\cellcolor[RGB]{242, 242, 242}\huxtpad{4pt}\raggedleft 3\huxbpad{4pt}} &
\multicolumn{1}{r!{\huxvb{0.4}}}{\cellcolor[RGB]{242, 242, 242}\huxtpad{4pt}\raggedleft 2\huxbpad{4pt}} \tabularnewline[-0.5pt]


\hhline{>{\huxb{0.4}}|>{\huxb{0.4}}|}
\arrayrulecolor{black}

\multicolumn{1}{!{\huxvb{0.4}}r!{\huxvb{0}}}{\huxtpad{4pt}\raggedleft 18.1\huxbpad{4pt}} &
\multicolumn{1}{r!{\huxvb{0}}}{\huxtpad{4pt}\raggedleft 6\huxbpad{4pt}} &
\multicolumn{1}{r!{\huxvb{0}}}{\huxtpad{4pt}\raggedleft 225\huxbpad{4pt}} &
\multicolumn{1}{r!{\huxvb{0}}}{\huxtpad{4pt}\raggedleft 105\huxbpad{4pt}} &
\multicolumn{1}{r!{\huxvb{0}}}{\huxtpad{4pt}\raggedleft 2.76\huxbpad{4pt}} &
\multicolumn{1}{r!{\huxvb{0}}}{\huxtpad{4pt}\raggedleft 3.46\huxbpad{4pt}} &
\multicolumn{1}{r!{\huxvb{0}}}{\huxtpad{4pt}\raggedleft 20.2\huxbpad{4pt}} &
\multicolumn{1}{r!{\huxvb{0}}}{\huxtpad{4pt}\raggedleft 1\huxbpad{4pt}} &
\multicolumn{1}{r!{\huxvb{0}}}{\huxtpad{4pt}\raggedleft 0\huxbpad{4pt}} &
\multicolumn{1}{r!{\huxvb{0}}}{\huxtpad{4pt}\raggedleft 3\huxbpad{4pt}} &
\multicolumn{1}{r!{\huxvb{0.4}}}{\huxtpad{4pt}\raggedleft 1\huxbpad{4pt}} \tabularnewline[-0.5pt]


\hhline{>{\huxb{0.4}}->{\huxb{0.4}}->{\huxb{0.4}}->{\huxb{0.4}}->{\huxb{0.4}}->{\huxb{0.4}}->{\huxb{0.4}}->{\huxb{0.4}}->{\huxb{0.4}}->{\huxb{0.4}}->{\huxb{0.4}}-}
\arrayrulecolor{black}
\end{tabularx}\par\end{raggedright}
\end{table}
 

\FloatBarrier

If you don't want this (e.g.~if you want to use \texttt{knitr::kable} or
the
\href{https://cran.r-project.org/package=printr/vignettes/printr.html}{printr
package}, then you can turn it off like this:

\begin{Shaded}
\begin{Highlighting}[]
\KeywordTok{options}\NormalTok{(}\DataTypeTok{huxtable.knit_print_df =} \OtherTok{FALSE}\NormalTok{)}

\KeywordTok{head}\NormalTok{(mtcars) }\CommentTok{# back to normal}
\end{Highlighting}
\end{Shaded}

\begin{verbatim}
##                    mpg cyl disp  hp drat    wt  qsec vs am gear carb
## Mazda RX4         21.0   6  160 110 3.90 2.620 16.46  0  1    4    4
## Mazda RX4 Wag     21.0   6  160 110 3.90 2.875 17.02  0  1    4    4
## Datsun 710        22.8   4  108  93 3.85 2.320 18.61  1  1    4    1
## Hornet 4 Drive    21.4   6  258 110 3.08 3.215 19.44  1  0    3    1
## Hornet Sportabout 18.7   8  360 175 3.15 3.440 17.02  0  0    3    2
## Valiant           18.1   6  225 105 2.76 3.460 20.22  1  0    3    1
\end{verbatim}

\begin{Shaded}
\begin{Highlighting}[]
\KeywordTok{options}\NormalTok{(}\DataTypeTok{huxtable.knit_print_df =} \OtherTok{TRUE}\NormalTok{)}
\end{Highlighting}
\end{Shaded}

\FloatBarrier

\hypertarget{using-huxtables-in-knitr-and-rmarkdown}{%
\subsection{Using huxtables in knitr and
rmarkdown}\label{using-huxtables-in-knitr-and-rmarkdown}}

If you use knitr and rmarkdown in RStudio, huxtable objects should
automatically display in the appropriate format (HTML, LaTeX or RTF).
You need to have some LaTeX packages installed for huxtable to work in
LaTeX. To find out what these are, you can call
\texttt{report\_latex\_dependencies()}. This will print out and/or
return a set of \texttt{usepackage\{...\}} statements. If you use Sweave
or knitr without rmarkdown, you can use this function in your LaTeX
preamble to load the packages you need. You can also automatically
install these packages using \texttt{install\_latex\_dependencies()}.

Rmarkdown exports to Word via Markdown. You can use huxtable to do this,
but since Markdown tables are rather basic, a lot of formatting will be
lost. If you want to create Word or Powerpoint documents directly,
install the
\href{https://cran.r-project.org/package=flextable}{flextable package}
from CRAN. You can then convert your huxtable objects to
\texttt{flextable} objects and include them in Word or Powerpoint
documents. Almost all formatting should work. See the \texttt{flextable}
and \texttt{officer} documentation and \texttt{?as\_flextable} for more
details.

Similarly, to create formatted reports in Excel, install the
\href{https://cran.r-project.org/package=openxlsx}{openxlsx package}.
You can then use \texttt{as\_Workbook} to convert your huxtables to
Workbook objects, and save them using \texttt{openxlsx::saveWorkbook}.

Sometimes you may want to select how huxtable objects are printed by
default. For example, in an RStudio notebook (a .Rmd document with
\texttt{output\_format\ =\ html\_notebook}), huxtable can't
automatically work out what format to use, as of the time of writing.
You can set it manually using
\texttt{options(huxtable.print\ =\ print\_notebook)} which prints out
HTML in an appropriate format.

You can print a huxtable on screen using \texttt{print\_screen} (or just
by typing its name at the command line.) Borders, column and row spans
and cell alignment are shown. If the
\href{https://cran.r-project.org/package=crayon}{crayon} package is
installed, and your terminal or R IDE supports it, border, text and
background colours are also displayed.

\begin{Shaded}
\begin{Highlighting}[]
\KeywordTok{print_screen}\NormalTok{(ht)}
\end{Highlighting}
\end{Shaded}

\begin{verbatim}
##          Employee table         
##   Employee              Salary  
## ────────────────────────────────
##   John Smith          50000.00  
##   Jane Jones          50000.00  
##   Hadley Wickham     100000.00  
##   David Hugh-Jones    40000.00  
## ────────────────────────────────
##   DHJ deserves a pay rise       
## 
## Column names: Employee, Salary
\end{verbatim}

\FloatBarrier

If you need to output to another format, file an
\href{https://github.com/hughjonesd/huxtable}{issue request} on Github.

\hypertarget{quick-output-commands}{%
\subsection{Quick output commands}\label{quick-output-commands}}

Sometimes you quickly want to get your data into a document. To do this
you can use huxtable functions starting with \texttt{quick\_}, shown
below.

 \begin{table}[h]
\begin{raggedright}
    \providecommand{\huxb}[2][0,0,0]{\arrayrulecolor[RGB]{#1}\global\arrayrulewidth=#2pt}
    \providecommand{\huxvb}[2][0,0,0]{\color[RGB]{#1}\vrule width #2pt}
    \providecommand{\huxtpad}[1]{\rule{0pt}{\baselineskip+#1}}
    \providecommand{\huxbpad}[1]{\rule[-#1]{0pt}{#1}}
  \begin{tabularx}{0.5\textwidth}{p{0.25\textwidth} p{0.25\textwidth}}


\hhline{>{\huxb{0.4}}->{\huxb{0.4}}-}
\arrayrulecolor{black}

\multicolumn{1}{!{\huxvb{0.4}}l!{\huxvb{0}}}{\huxtpad{4pt}\raggedright {\fontfamily{cmtt}\selectfont \textbf{Command}}\huxbpad{4pt}} &
\multicolumn{1}{l!{\huxvb{0.4}}}{\huxtpad{4pt}\raggedright \textbf{Output}\huxbpad{4pt}} \tabularnewline[-0.5pt]


\hhline{>{\huxb{0.4}}->{\huxb{0.4}}-}
\arrayrulecolor{black}

\multicolumn{1}{!{\huxvb{0.4}}l!{\huxvb{0}}}{\cellcolor[RGB]{242, 242, 242}\huxtpad{4pt}\raggedright {\fontfamily{cmtt}\selectfont quick\_pdf}\huxbpad{4pt}} &
\multicolumn{1}{l!{\huxvb{0.4}}}{\cellcolor[RGB]{242, 242, 242}\huxtpad{4pt}\raggedright PDF document\huxbpad{4pt}} \tabularnewline[-0.5pt]


\hhline{>{\huxb{0.4}}|>{\huxb{0.4}}|}
\arrayrulecolor{black}

\multicolumn{1}{!{\huxvb{0.4}}l!{\huxvb{0}}}{\huxtpad{4pt}\raggedright {\fontfamily{cmtt}\selectfont quick\_docx}\huxbpad{4pt}} &
\multicolumn{1}{l!{\huxvb{0.4}}}{\huxtpad{4pt}\raggedright Word document\huxbpad{4pt}} \tabularnewline[-0.5pt]


\hhline{>{\huxb{0.4}}|>{\huxb{0.4}}|}
\arrayrulecolor{black}

\multicolumn{1}{!{\huxvb{0.4}}l!{\huxvb{0}}}{\cellcolor[RGB]{242, 242, 242}\huxtpad{4pt}\raggedright {\fontfamily{cmtt}\selectfont quick\_html}\huxbpad{4pt}} &
\multicolumn{1}{l!{\huxvb{0.4}}}{\cellcolor[RGB]{242, 242, 242}\huxtpad{4pt}\raggedright HTML web page\huxbpad{4pt}} \tabularnewline[-0.5pt]


\hhline{>{\huxb{0.4}}|>{\huxb{0.4}}|}
\arrayrulecolor{black}

\multicolumn{1}{!{\huxvb{0.4}}l!{\huxvb{0}}}{\huxtpad{4pt}\raggedright {\fontfamily{cmtt}\selectfont quick\_xlsx}\huxbpad{4pt}} &
\multicolumn{1}{l!{\huxvb{0.4}}}{\huxtpad{4pt}\raggedright Excel spreadsheet\huxbpad{4pt}} \tabularnewline[-0.5pt]


\hhline{>{\huxb{0.4}}|>{\huxb{0.4}}|}
\arrayrulecolor{black}

\multicolumn{1}{!{\huxvb{0.4}}l!{\huxvb{0}}}{\cellcolor[RGB]{242, 242, 242}\huxtpad{4pt}\raggedright {\fontfamily{cmtt}\selectfont quick\_pptx}\huxbpad{4pt}} &
\multicolumn{1}{l!{\huxvb{0.4}}}{\cellcolor[RGB]{242, 242, 242}\huxtpad{4pt}\raggedright Powerpoint presentation\huxbpad{4pt}} \tabularnewline[-0.5pt]


\hhline{>{\huxb{0.4}}|>{\huxb{0.4}}|}
\arrayrulecolor{black}

\multicolumn{1}{!{\huxvb{0.4}}l!{\huxvb{0}}}{\huxtpad{4pt}\raggedright {\fontfamily{cmtt}\selectfont quick\_rtf}\huxbpad{4pt}} &
\multicolumn{1}{l!{\huxvb{0.4}}}{\huxtpad{4pt}\raggedright RTF document\huxbpad{4pt}} \tabularnewline[-0.5pt]


\hhline{>{\huxb{0.4}}->{\huxb{0.4}}-}
\arrayrulecolor{black}
\end{tabularx}\par\end{raggedright}
\end{table}
 

\FloatBarrier

These are called with one or more huxtable objects (or objects which can
be turned into a huxtable, such as data frames). A new document of the
appropriate type will be created and opened. By default the file will be
in the current directory, under a name like e.g.
\texttt{huxtable-output.pdf}. If the file already exists, you'll be
asked to confirm the overwrite. For non-interactive use, you must
specify a filename yourself explicitly. This keeps you from accidentally
trashing your files.

\begin{Shaded}
\begin{Highlighting}[]
\KeywordTok{quick_pdf}\NormalTok{(mtcars) }
\KeywordTok{quick_pdf}\NormalTok{(mtcars, }\DataTypeTok{file =} \StringTok{'motorcars data.pdf'}\NormalTok{)}
\end{Highlighting}
\end{Shaded}

\FloatBarrier

\hypertarget{end-matter}{%
\section{End matter}\label{end-matter}}

For more information, see the
\href{https://hughjonesd.github.io/huxtable}{website} or
\href{https://github.com/hughjonesd/huxtable}{github}.


\end{document}
